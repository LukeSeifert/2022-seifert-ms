
\section{Derivation of Batchwise Reprocessing Steady State Mass}

The first step of this derivation is to separate the $\lambda$ term into the continuous $\lambda_c$, which is the continuous losses in the Bateman equation, and the batchwise reprocessing $\lambda_r$.

\begin{equation} \hfill
\lambda = \lambda_c + \lambda_r
\hfill \end{equation}

Next, set up and solve the differential equation for the continuous portions of the process.

\begin{equation} \hfill
\frac{dm}{dt} = C - \lambda_c m
\hfill \end{equation}

\begin{equation} \hfill
m(t) = \left(m_0 - \frac{C}{\lambda_c} \right) e^{-\lambda_c t}  + \frac{C}{\lambda_c}
\hfill \end{equation}

Change the continuous process to one which updates at discrete time steps.

\begin{equation} \hfill
m_{n+1} = \left(m_n - \frac{C}{\lambda_c} \right) e^{-\lambda_c \Delta t}  + \frac{C}{\lambda_c}
\hfill \end{equation}

Apply the batchwise reprocessing after each new discrete time step, treating each time step evaluated as a depletion time step.

\begin{equation} \hfill
m_{n+1} = \left(\left(m_n - \frac{C}{\lambda_c} \right) e^{-\lambda_c \Delta t}  + \frac{C}{\lambda_c}\right) \left(1 - {\lambda_r \Delta t} \right)
\hfill \end{equation}

Define some arbitrary variables for ease of use.

\begin{equation} \hfill
\alpha = \frac{C}{\lambda_c}
\hfill \end{equation}

\begin{equation} \hfill
\epsilon = e^{-\lambda_c \Delta t}
\hfill \end{equation}

\begin{equation} \hfill
\gamma = \left(1 - {\lambda_r \Delta t} \right)
\hfill \end{equation}

Using these simplified variables, the mass can be calculated for several subsequent steps.

\begin{equation} \hfill
m_{n+1} = \left(\left(m_n - \alpha \right) \epsilon  + \alpha\right) \gamma
\hfill \end{equation}

\begin{equation} \hfill
m_0 = m_0
\hfill \end{equation}

\begin{equation} \hfill
m_1 = m_0 \epsilon \gamma - \alpha \epsilon \gamma + \alpha \gamma
\hfill \end{equation}

\begin{equation} \hfill
m_2 = m_0 \epsilon^2 \gamma^2 - \alpha \epsilon^2 \gamma^2 + \alpha \epsilon \gamma^2 - \alpha \epsilon \gamma + \alpha \gamma
\hfill \end{equation}

From these steps, a more general formula can be derived.

\begin{equation} \hfill
m_n = m_0 \epsilon^n \gamma^n + \alpha \sum_{j=1}^n \gamma^j (\epsilon^{j-1} - \epsilon^j)
\hfill \end{equation}

Moving towards the goal of the steady state mass, the value of $n$ is taken to be approaching infinity. Because the values of $\gamma$ and $\alpha$ are both between zero and one, simplifications can be made.

\begin{equation} \hfill
m_{ss} = m_0 \epsilon^{\infty} \gamma^{\infty} + \alpha \sum_{j=1}^{\infty} \gamma^j (\epsilon^{j-1} - \epsilon^j)
\hfill \end{equation}

\begin{equation} \hfill
m_{ss} = \alpha \sum_{j=1}^{\infty} \gamma^j (\epsilon^{j-1} - \epsilon^j)
\hfill \end{equation}

The next step is to simplify the summation term by changing $\epsilon$ back to its original form.

\begin{equation} \hfill
m_{ss} = \alpha \sum_{j=1}^{\infty} \gamma^j (e^{-\lambda_c \Delta t (j-1)} - e^{-\lambda_c \Delta t j})
\hfill \end{equation}

\begin{equation} \hfill
m_{ss} = \alpha \sum_{j=1}^{\infty} \gamma^j e^{-\lambda_c \Delta t j} (e^{\lambda_c \Delta t} - 1)
\hfill \end{equation}

\begin{equation} \hfill
m_{ss} = \alpha (e^{\lambda_c \Delta t} - 1) \sum_{j=1}^{\infty} \gamma^j e^{-\lambda_c \Delta t j} 
\hfill \end{equation}

Next, the infinite series can be solved by using the following equation.

\begin{equation} \hfill
\sum_{j=1}^{\infty} x^j = \frac{x}{1-x} \ni |x| < 1
\hfill \end{equation}

\begin{equation} \hfill
m_{ss} = \alpha (e^{\lambda_c \Delta t} - 1) \frac{\gamma^j e^{-\lambda_c \Delta t j}}{1 - \gamma^j e^{-\lambda_c \Delta t j}}
\hfill \end{equation}

Plugging in the variables $\alpha$ and $\gamma$ yields the full equation for the steady state mass with batchwise reprocessing.


