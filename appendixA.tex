\renewcommand{\theequation}{A\thechapter.\arabic{equation}}

This appendix serves as a supplement to the discussion in Section \ref{s:batch-sum}.
Here, I show the steady state mass of some arbitrary nuclide which is processed using Steady batchwise reprocessing.

The first step of this derivation is to separate the $\lambda$ term into the continuous $\lambda_c$, which is the continuous losses in the Bateman equation, and the batchwise reprocessing $\lambda_r$.
This is given by:

\begin{equation} \hfill
\lambda = \lambda_c + \lambda_r.
\hfill \end{equation}

Next, set up and solve the differential equation for the continuous portions of the process.
This is set as the rate of change of the mass over time is equivalent to the constant rate of gain, $C$, minus the continuous losses.
This is given by:

\begin{equation} \hfill
\frac{dm}{dt} = C - \lambda_c m,
\hfill \end{equation}

where

\begin{equation} \hfill
m(t) = \left(m_0 - \frac{C}{\lambda_c} \right) e^{-\lambda_c t}  + \frac{C}{\lambda_c}.
\hfill \end{equation}

Then, I modify the continuous equation to a process which updates at discrete time steps and is pre-reprocessing:

\begin{equation} \hfill
m_{n+1}^{pre} = \left(m_n - \frac{C}{\lambda_c} \right) e^{-\lambda_c \Delta t}  + \frac{C}{\lambda_c}.
\hfill \end{equation}

After each new discrete time step, I apply the batchwise reprocessing. 
Each time step is evaluated as a depletion time step.
This is given by:

\begin{equation} \hfill
m_{n+1} = \left(\left(m_n - \frac{C}{\lambda_c} \right) e^{-\lambda_c \Delta t}  + \frac{C}{\lambda_c}\right) \left(1 - {\lambda_r \Delta t} \right).
\hfill \label{eq:nplusone}\end{equation}

From here, I define some arbitrary variables for ease of use:

\begin{equation} \hfill
\alpha = \frac{C}{\lambda_c},
\hfill \end{equation}

\begin{equation} \hfill
\epsilon = e^{-\lambda_c \Delta t},
\hfill \end{equation}

\begin{equation} \hfill
\gamma = \left(1 - {\lambda_r \Delta t} \right).
\hfill \end{equation}

Using these in Equation \eqref{eq:nplusone}, the mass can be calculated for several subsequent steps.
The more compact form of Equation \eqref{eq:nplusone} is given by:

\begin{equation} \hfill
m_{n+1} = \left(\left(m_n - \alpha \right) \epsilon  + \alpha\right) \gamma.
\hfill \label{eq:one} \end{equation}

The initial condition is defined as:

\begin{equation} \hfill
m(t=0) = m_0.
\hfill \label{eq:two} \end{equation}

Combining Equations \eqref{eq:one} and \eqref{eq:two}, the first step is given by:

\begin{equation} \hfill
m_1 = m_0 \epsilon \gamma - \alpha \epsilon \gamma + \alpha \gamma,
\hfill \end{equation}

and the second step is given by:

\begin{equation} \hfill
m_2 = m_0 \epsilon^2 \gamma^2 - \alpha \epsilon^2 \gamma^2 + \alpha \epsilon \gamma^2 - \alpha \epsilon \gamma + \alpha \gamma.
\hfill \end{equation}

Finally, the n$^{th}$ step mass can be derived by following the pattern shown in the previous equations.
This is given by:

\begin{equation} \hfill
m_n = m_0 \epsilon^n \gamma^n + \alpha \sum_{j=1}^n \gamma^j (\epsilon^{j-1} - \epsilon^j).
\hfill \end{equation}

As the time approaches infinity, the steady state mass will be reached:

\begin{equation} \hfill
m_{ss} = \lim_{t \rightarrow \infty} m(t).
\hfill \end{equation}

Because this is an iterative process, this means $n$ will be set to approach infinity to reach steady state.
%Moving towards the goal of the steady state mass, the value of $n$ is taken to be approaching infinity.
This is given by:

\begin{equation} \hfill
m_{ss} = m_0 \epsilon^{\infty} \gamma^{\infty} + \alpha \sum_{j=1}^{\infty} \gamma^j (\epsilon^{j-1} - \epsilon^j).
\hfill \end{equation}

The values of $\gamma$ and $\epsilon$ are between zero and one, so raising them to a large power makes them approach zero. The equation simplifies to:
 
\begin{equation} \hfill
m_{ss} = \alpha \sum_{j=1}^{\infty} \gamma^j (\epsilon^{j-1} - \epsilon^j).
\hfill \end{equation}

The next step is to simplify the summation term by changing $\epsilon$ back to its original form.
This is given by:

\begin{equation} \hfill
m_{ss} = \alpha \sum_{j=1}^{\infty} \gamma^j (e^{-\lambda_c \Delta t (j-1)} - e^{-\lambda_c \Delta t j}).
\hfill \end{equation}

Next, I group the exponential terms together that have $j$ in the exponent, given by:

\begin{equation} \hfill
m_{ss} = \alpha \sum_{j=1}^{\infty} \gamma^j e^{-\lambda_c \Delta t j} (e^{\lambda_c \Delta t} - 1).
\hfill \end{equation}

The terms which multiply but do not contain $j$ from the summation can be pulled outside the summation, given by:

\begin{equation} \hfill
m_{ss} = \alpha (e^{\lambda_c \Delta t} - 1) \sum_{j=1}^{\infty} \gamma^j e^{-\lambda_c \Delta t j}.
\hfill \end{equation}

This is of the form of a generic geometric series, which has a known convergence criteria and solution:

\begin{equation} \hfill
\sum_{n=1}^{\infty} x^n = \frac{x}{1-x} \ni |x| < 1
\hfill \end{equation}

This yields the equation for the steady state mass:

\begin{equation} \hfill
m_{ss} = \alpha (e^{\lambda_c \Delta t} - 1) \frac{\gamma^j e^{-\lambda_c \Delta t j}}{1 - \gamma^j e^{-\lambda_c \Delta t j}}.
\hfill \end{equation}

Plugging in the variables $\alpha$ and $\gamma$ yields the full equation for the steady state mass with batchwise reprocessing:

\begin{equation} \hfill 
m_{ss} =  \frac{C}{\lambda - \lambda_r} \left(e^{(\lambda - \lambda_r) \Delta t} - 1 \right) \sum_{n=1}^{\infty} \left( \left(1 - \lambda_r \Delta t \right) e^{-(\lambda - \lambda_r) \Delta t} \right)^n.
\hfill \end{equation}
