%% Package and Class "uiucthesis2014" for use with LaTeX2e.
\documentclass[edeposit,fullpage]{uiucthesis2018}

\usepackage[utf8]{inputenc}
\usepackage{graphicx}
\usepackage{float}
\usepackage{amsmath}
\usepackage[version=4]{mhchem} % \ce{^{227}_{90}Th+}

\usepackage[acronym,toc]{glossaries}
\include{acros}

\usepackage{xspace}
\usepackage{graphics}
\usepackage{float} 
\newcommand{\Cycamore}{\textsc{Cycamore}\xspace}
\newcommand{\Cyclus}{\textsc{Cyclus}\xspace}


\usepackage{placeins}
\usepackage{booktabs} % nice rules (thick lines) for tables
\usepackage{microtype} % improves typography for PDF

\usepackage[hyphens]{url}
\usepackage{hyperref}
%\usepackage{subfig}
\usepackage{hhline}
\usepackage{amsmath}
\usepackage{color}
\usepackage{multirow}
\usepackage{siunitx}
\usepackage{subcaption}
\sisetup{
    input-decimal-markers = .,input-ignore = {,},table-number-alignment = right,
    group-separator={,}, group-four-digits = true
}
\usepackage{fourier}
\usepackage{booktabs}
\newcommand\tab[1][1cm]{\hspace*{#1}}

\usepackage{threeparttable, tablefootnote}

%tikzpicture fit to page width
\usepackage{environ}
\makeatletter
\newsavebox{\measure@tikzpicture}
\NewEnviron{scaletikzpicturetowidth}[1]{%
  \def\tikz@width{#1}%
  \def\tikzscale{1}\begin{lrbox}{\measure@tikzpicture}%
  \BODY
  \end{lrbox}
  \pgfmathparse{#1/\wd\measure@tikzpicture}%
  \edef\tikzscale{\pgfmathresult}%
  \BODY
}

\usepackage{tabularx}
\newcolumntype{b}{>{\hsize=1.0\hsize}X}
\newcolumntype{q}{>{\hsize=0.5\hsize}X}
\newcolumntype{R}{>{\raggedleft\arraybackslash\hsize=0.5\hsize}X}
\newcolumntype{z}{>{\hsize=0.75\hsize}X}
\newcolumntype{s}{>{\hsize=.5\hsize}X}
\newcolumntype{m}{>{\hsize=.75\hsize}X}

\usepackage{cleveref}
\usepackage{datatool}
\usepackage[numbers]{natbib}
\usepackage{notoccite}


\usepackage{tikz}
\usetikzlibrary{positioning, arrows, decorations, shapes}

\usetikzlibrary{shapes.geometric,arrows}
\tikzstyle{process} = [rectangle, rounded corners, minimum width=2.5cm, minimum height=1cm,text centered, draw=black, fill=blue!30]

\tikzstyle{object} = [ellipse, rounded corners, minimum width=3cm, minimum height=1cm,text centered, draw=black, fill=green!30]
\tikzstyle{objectr} = [ellipse, rounded corners, minimum width=3cm, minimum height=1cm,text centered, draw=black, fill=red!30]

\tikzstyle{empty} =  [rectangle, rounded corners, minimum width=2.5cm, minimum height=0.7cm,text centered, draw=black, fill=white!30]
\tikzstyle{arrow} = [thick,->,>=stealth]


\title{Analysis of Reprocessing Methods in the Molten Salt Breeder Reactor}
\author{Luke Seifert}
\department{Nuclear, Plasma, Radiological Engineering}
%\schools{B.S., University of Illinois - Urbana Champaign, 2017}
\schools{}
\msthesis
\advisor{Madicken Munk}
%\advisor{Madicken Munk, Kathryn D. Huff}
\degreeyear{2022}
\committee{Madicken Munk \\ Tomasz Kozlowski}
%\committee{Assistant Professor Kathryn D. Huff, Adviser \\ Professor Segundo Lector }


\begin{document}
\maketitle

\frontmatter
%% Create an abstract that can also be used for the ProQuest abstract.
%% Note that ProQuest truncates their abstracts at 350 words.
\begin{abstract}

Molten salt reactors have had many depletion models generated since being announced as a generation IV reactor. Among these models, one important aspect which is considered is the online reprocessing capabilities of a liquid fueled molten salt reactor. This includes adding fresh fuel and removing fission products from the reactor during operation, allowing for a more efficient neutron economy.
In order to simulate reprocessing, there are two different mathematical approaches which are implemented, called batchwise and continuous reprocessing. For online reprocessing, the continuous reprocessing method is more physically reflective of the operation in reality. However, there are many works which have implemented batchwise reprocessing to simulate a continuous reprocessing process.
This work shows that continuous reprocessing and batchwise reprocessing methods are not interchangeable, and there is a fairly large difference in the results when using the two different approaches for the same system. This work also shows the computational cost and a depletion step mesh refinement for both continuous and batchwise reprocessing. From those results, it is determined that continuous reprocessing allows for significantly larger steps without large increases in error, which enables drastically reduced computational cost on the order of one magnitude. The effect of density increase when using the Serpent2 code due to continuous reprocessing is also investigated, and it is determined that for typical operations, the density change is roughly 0.12\%.


\end{abstract}

\chapter*{Acknowledgments}

This material is based upon work supported under an Integrated University Program Graduate Fellowship.

%% The thesis format requires the Table of Contents to come
%% before any other major sections, all of these sections after
%% the Table of Contents must be listed therein (i.e., use \chapter,
%% not \chapter*).  Common sections to have between the Table of
%% Contents and the main text are:
%%
%% List of Tables
%% List of Figures
%% List Symbols and/or Abbreviations
%% etc.

\tableofcontents
\listoftables
\listoffigures

%% Create a List of Abbreviations. The left column
%% is 1 inch wide and left-justified
%\chapter{List of Abbreviations}
%\printglossaries
%% Create a List of Symbols. The left column
%% is 0.7 inch wide and centered

\pagebreak
\mainmatter

\chapter{Introduction}

\renewcommand*\descriptionlabel[1]{\hspace\leftmargin$#1$}
\setcounter{tocdepth}{5}
\setcounter{secnumdepth}{5}
\newcommand{\minus}{\scalebox{0.65}[1.0]{$-$}}

\section{Energy}

In modern society, the usefulness and convenience supplied by advancing technologies has allowed quality of life to increase drastically over time. Within these advancing technologies, ranging from air conditioning to food production, there is a common thread of energy requirement. To cool the air, energy is required in order to compress, pressurize, and cool the refrigerant. To produce food, energy is required to pump water, create fertilizer, and operate heavy machinery. Thus, the importance of energy is significant in maintaining and developing quality of life for society.

However, energy is not free, and it takes effort to harvest and convert energy into usable forms. For wind power, this means building large turbines which can be turned by the wind. For solar power, this means building solar panels, which convert photons from the sun into electrical energy. However, for most large-scale electrical generation needs, it is more efficient for the conversion aspect to use boiling water to turn a turbine, which then generates electricity, as can be seen in Figure \ref{fig:energy-gen-methods}. Regardless of how it is generated, electricity can then be transported or stored to be used by people for any variety of purpose. 

\begin{figure}[H]
  \centering
  \includegraphics[scale=0.7]{images/US-energy-prod.png}
  \caption{Energy generation in the United States in 2021 from the US Energy Information Administration CITE https://www.eia.gov/electricity/monthly/.}
   \label{fig:energy-gen-methods}
\end{figure}

Figure \ref{fig:energy-gen-methods} shows the portion of energy generation. From this figure, it can be seen that the methods which dominate energy contribution are natural gas, coal, and nuclear. Of the various methods to generate electricity, desirable characteristics would be safety, efficiency, consistency, availability, and economics. Nuclear energy is strong in these characteristics, though there is an economic problem due to high up-front costs, which are on the order of \$10 billion for a 1GW nuclear power plant \cite{du_update_2009}. 

\section{Nuclear Energy}

Part of the reason nuclear energy has high up-front costs is because the nuclear industry is highly safety conscious and has many regulations in place in order to minimize risks. Although this mindset increases costs, the effort has paid dividends, which is apparent when viewing the deaths associated with different energy generation methods, shown in Table \ref{tab:death-mega} \cite{brook_why_2014}.

\begin{table}[H]
\renewcommand{\arraystretch}{1.25}
\caption{Mortality rates for each energy source in deaths per billion kWh produced (recreated from \cite{brook_why_2014}).}
\label{tab:death-mega}
\begin{center}
\begin{tabular}{ | c | c | }
 \hline
 Energy Source & Mortality Rate (deaths per billion kWh)\\
 \hline
 \hline
 Coal (Global Average) & 100 \\
 Biofuel/Biomass & 24 \\
 Coal (U.S.) & 15 \\
 Oil & 36 \\
 Natural Gas & 4 \\
 Hydro (Global Average) & 1.4 \\
 Solar (Rooftop) & 0.44 \\
 Wind & 0.15 \\
 Nuclear (Global Average) & 0.04 \\
 
 \hline
\end{tabular}
\end{center}
\end{table}

Even though the up-front costs are high, nuclear power plants are licensed to operate for 40 years and can get extensions to extend that time by 10-20 years \cite{bredimas_international_2008}. This means that nuclear plants are a very long term investment. An investment at this time scale is difficult to make for a private industry though, as there is less risk when investing in cheaper plants, such as natural gas. 

In order to address the economic issue of nuclear power plants, there are many different approaches which are implemented. One of these is to build smaller, cheaper nuclear plants which are then more affordable and are have less up-front cost. Another approach is from the regulation perspective, and involves either taxing negative externalities more harshly or subsidizing carbon-free energy sources. There is also the approach of developing new generation IV reactor designs \cite{kelly_generation_2014}.

\section{Molten Salt Reactors}

One of the new generation IV reactor designs is the molten salt reactor, or MSR \cite{kelly_generation_2014}. The MSR potentially offers a lower cost of electricity when compared to coal and pressurized water reactors, which are a large portion of the nuclear fleet \cite{moir_cost_2002}. The liquid fueled MSR design is significantly different from the current light water reactor, or LWR, designs which dominate the nuclear fleet currently in use. Light water reactors include boiling water reactors (BWRs) and pressurized water reactors (PWRs), both of which are well understood reactor designs currently implemented. These LWRs use a solid fuel containing uranium-235 with a water coolant, which also serves as a moderator to slow down the neutrons to thermal energies. This fuel generates heat as the fissile uranium-235 fissions. After some lengthy amount of depletion, it can be reshuffled or removed from the reactor to be placed into a cooling pool.

Molten salt reactors, instead of a solid fuel, can use a liquid fuel composed of a molten salt mixed with some fissile isotope, such as uranium-235. Using a liquid fuel allows for online reprocessing, which is a continuous process which can involve adding fresh fuel to the system as well as chemically removing undesirable products which are generated through fission. The system also allows for batchwise reprocessing, which is a process which occurs in discrete steps, such as salt disposal. Typically, most reprocessing involved in an MSR is continuously performed while the reactor is online.

The usefulness of reprocessing, particularly online reprocessing, is that the reactor does not need excess reactivity in order to continue operating for a long period of time. In a typical LWR, burnable poisons and control rods are used to lower reactivity initially so the reactor can burn longer. In a liquid fueled MSR, the parasitic absorbers, such as xenon-135, can be removed from the reactor. Additionally, fresh fuel salt can be added to the reactor in order to maintain stable operation.

This fresh fuel feed can include fissile isotopes, such as uranium-233 or uranium-235, fertile isotopes, such as uranium-238 or thorium-232, or some combination of both. Adding a fissile isotope to the reactor will allow for the reactor to continue fission when the isotope is fissioned by a neutron. For the fertile isotopes, the neutron must first be absorbed and the product decays, leading to a fissile isotope. This process can be seen in Equation \eqref{eq:th-pa-chain}, which is the breeding process used in the Molten Salt Breeder Reactor.

\begin{equation} \hfill
\ce{^{232}_{90}Th} + \ce{^{1}_{0}n} \longrightarrow \ce{^{233}_{90}Th} \longrightarrow \ce{^{233}_{91}Pa} \longrightarrow \ce{^{233}_{92}U}
\hfill\label{eq:th-pa-chain} \end{equation}

The half life of thorium-233 via beta decay is on the order of tens of minutes, while the half life of protactinium-233 via beta decay is roughly 27 days. Without reprocessing, the protactinium-233 remaining in the core for such a period of time would be more likely to interact with neutrons, impacting the breeding efficiency. The Molten Salt Breeder Reactor continuously chemically extracts the protactinium from the fuel salt, allowing it to decay in a separate tank. While it decays, another continuous chemical process strips uranium which is formed and brings the uranium back into the reactor, continuing the process.

\section{Reprocessing Methods}

For molten salt reactor reprocessing, the two different physical approaches are continuous and batchwise. Additionally, these reprocessing schemes can be online, while the reactor is operating, or offline, when the reactor is shutdown. Of particular importance for MSR behaviour is online reprocessing, which changes the fuel composition while the reactor is operating.

In order to model online reprocessing, there are two different computational methods which can be implemented. These methods are named continuous and batchwise reprocessing methods, the same naming convention as the physical processes which they represent. However, it is fairly common for MSR analysis works which implement a phyiscally continuous reprocessing scheme to simulate this scheme using batchwise mathematical models.

Thus, this work seeks to investigate the continuous and batchwise computational reprocessing methods. This investigation involves effects of depletion step size, validity of batchwise methods approximating continuous methods, and different sub-methods within both batchwise and continuous methods which can be implemented.


\chapter{Literature Review}
\renewcommand*\descriptionlabel[1]{\hspace\leftmargin$#1$}
\setcounter{tocdepth}{5}
\setcounter{secnumdepth}{5}

In this chapter, I will provide an overview of MSR modeling in the literature. This includes a discussion on different reprocessing methods and treatment of delayed neutron precursors. I will then provide more general information on depletion codes and the MSBR. Finally, I will provide a set of works and a brief discussion of each in order to show the state of the literature in terms of handling reprocessing in depletion simulations. This set of works will also show the number of works which have employed batchwise methods to approximate continuous reprocessing schemes. This is important, because a key takeaway of this work is that this approximation causes a buildup of error in the material composition.

\section{Molten Salt Reactors}

Modeling and simulation of liquid fueled molten salt reactors (MSRs) differs from solid fueled reactors in two main areas. The first is in the fresh fuel feed and fission product removal streams used during operation in MSRs; which is referred to as online reprocessing. Modeling an MSR without online reprocessing will cause discrepancies that will increase as the net depletion time increases.
This is a result of neutronic differences that arise from a lack of fresh fuel and buildup of fission products. In order to simulate this reprocessing functionality, software can use batchwise and continuous reprocessing methods.

The second difference is in the movement of the fuel salt, which causes the fuel and, consequently, the delayed neutron precursors (DNPs) to move in the core. Because the DNPs have different half lives before the delayed neutrons are born, the effect of the fuel movement on each group varies. However, the overall result is that the delayed neutrons are birthed in less important regions, such as external piping, which means there is a reduced effective delayed neutron fraction in the core.
This fraction is important for safe operation and reactivity control, so modeling it accurately is important to advancing this technology.
%Modeling and simulation of liquid fueled molten salt reactors (MSRs) is different from modeling and simulation of solid fueled reactors in a few different areas. Two main areas of particular importance are the online reprocessing functionality and the movement of the fuel through the reactor. The online reprocessing functionality involves removing fission products from the fuel and adding fresh fuel during operation. The movement of the fuel causes a shift in the location of delayed neutron precursors (DNPs). In solid fueled reactors, the distribution of DNPs follows the flux profile, while in MSRs the DNPs move along with the fuel. This results in a reduction in the effective delayed neutron fraction due to decay of DNPs in less important regions.

\subsection{Online Reprocessing}

The online reprocessing functionality of MSRs is simulated in different ways depending on the particular software used, and depending on the methods employed by that software. The two main ways to approach online reprocessing is by using batchwise reprocessing and continuous reprocessing methods. %For these approaches, the physical process may represent either approach, such as continuously removing protactinium from the core, but adding in the uranium it decays into only after some time has elapsed.
A purely batchwise approach without a sufficiently short time step will not fit the reprocessing design used by an MSR, as online reprocessing is carried out continuously.
For the batchwise method to fit a continuous reprocessing scheme, the time step would need to be on the same time scale as the shortest cycle time, or time it takes for a given element to be completely removed.
For the MSBR, the shortest cycle time is 20 seconds, which would make a depletion simulation on the order of years very computationally costly.
The batchwise method can accurately model a reprocessing scheme which uses batchwise reprocessing, such as a scheme which has salt entirely replaced after some given amount of time. An example of this is a salt discard, which occurs after a set amount of time. An entirely batchwise reprocessing scheme for an MSR would be neglecting the benefit of a liquid fuel, so such a scheme is not common in the literature.
%An MSR using offline reprocessing, however, can be accurately represented using purely batchwise reprocessing.

%Codes implement batchwise reprocessing by running the simulation, stopping, and then adding and removing materials. The process is then iterated. This method is useful in that it is fairly straightforward to implement, can be calibrated to different removal efficiencies, and makes mass balancing of the core straightforward. Some issues with the batchwise approach are that using large time steps will make the results more inaccurate, it has increased computational cost with smaller time steps, and it is an approximation of the actual physical process of reprocessing. It is an approximation because, in online reprocessing of a reactor, the fission products are continuously removed while any feed flows are continuously fed into the reactor.
Codes such as SaltProc and older versions of ChemTriton implement batchwise reprocessing by running the simulation for a fixed amount of time, then stopping, and then adding and removing materials \cite{rykhlevskii_modeling_2019, betzler_molten_2017}. The code iterates through this process for each depletion step until it has reached the final time. This method is useful because it is fairly straightforward to implement, easy to customize, and makes mass balancing of the core straightforward. Some issues with the batchwise approach are that using large time steps will make the results less accurate; it has to rerun each time step for pre-reprocessing and post-reprocessing; and it is an approximation of the actual physical process of reprocessing.
In an operating MSR, the fission products are continuously removed while any feed flows are continuously fed into the reactor, not removed in batches.
The reason that large time steps make the results less accurate can be made clear by imagining the limiting case, a single very large depletion step.
Taking a very large time step would mean that all of the reprocessing that should have taken place during that time step does not.
For example, instead of removing the xenon every 20 seconds, there is only a single removal at the very end after potentially years.
This means that the parasitic absorption of neutrons by fission products is not at all reduced over the entire depletion step, but instead the composition is only adjusted at the very end.
This example makes clear why small depletion steps are required in order to generate useful results using the batchwise method.

%It is an approximation because, in online reprocessing of a reactor, the fission products are continuously removed while any feed flows are continuously fed into the reactor.

%Continuous reprocessing is implemented by adding terms to the Bateman equation similar to an extra decay term. The benefits of using continuous reprocessing are that the model is physically accurate, it doesn't impact computational cost as much as a batchwise process, and it allows for large time steps to be used. A note should be made, however, that the time step size is still limited since the simulation will need to update based on the new material composition and evolving neutron spectrum. However, there is less error from inaccurate material removal and addition. The main downside of the continuous removal approach is that it is more cumbersome to implement and configure due to the nature of how it operates.
Codes such as Serpent2 and newer versions of ChemTriton implement continuous reprocessing by adding terms to the Bateman equation similar to an extra decay term, as can be seen in Equation (\ref{eq:Bateman_type_one_decay_dupe}), recreated here from Equation (\ref{eq:Bateman_type_one_decay}) for convenience of reading \cite{aufiero_extended_2013, jr_vicente_valdez_modeling_2020}. The benefits of using continuous reprocessing are that the model is more physically accurate and it allows for larger time steps to be used without increasing error. However, the largest time step size cannot become too large, or the simulation will become less accurate.
The exact value of the largest time step while remaining sufficiently accurate is dependent upon the reactor, the rate at which the material composition changes, and the level of accuracy which is desired.
This is because the neutron spectrum and cross section data does not update to the new material compositions until a new depletion step begins.


\begin{equation} \hfill
%\frac{dN}{dt} = \sum_j \lambda_j N_j + \gamma \Sigma_f \phi + \Sigma_k \phi - \lambda N - \Sigma \phi - C
%\frac{dN_j}{dt} = \sum_{i \neq j} \left [ \left( \gamma_{i \rightarrow j} \sigma_{f, i} \Phi + \lambda _{i \rightarrow j} + \lambda _{reproc, i \rightarrow j} + \sigma_{i \rightarrow j} \Phi \right) N_i \right ] - \left ( \lambda_j + \lambda_{reproc, j} + \sigma_j \Phi \right ) N_j
%\frac{dN_j}{dt} = \sum_{i \neq j} \left [ \left( \gamma_{i \rightarrow j} \sigma_{f, i} \Phi + \lambda _{i \rightarrow j} + \lambda _{r, i \rightarrow j} + \sigma_{i \rightarrow j} \Phi \right) N_i \right ] - \left ( \lambda_j + \lambda_{r, j} + \sigma_j \Phi \right ) N_j + \sum_{mat} \lambda _{r, i \rightarrow j} N_i
%\frac{dN_j}{dt}_{net} = \frac{dN_j}{dt}_{base} -  \lambda_{r, j} N_j + \sum_{mat} \lambda _{r, i \rightarrow j} N_i
\frac{dN_j}{dt}_{net} =\left( \frac{dN_j}{dt} \right)_{no\text{ }reprocessing} -  \lambda_{r, j} N_j + \sum_{feed} \lambda _{f, j} N_j
\hfill\label{eq:Bateman_type_one_decay_dupe} \end{equation}

%The main downsides of the continuous reprocessing approach are that it is more cumbersome to implement and difficult to configure due to the nature of how it operates.

Another aspect of online reprocessing that must be considered is mass balancing.
%For batchwise reprocessing, mass balancing can be solved in several different ways.
As mentioned previously, mass balancing is straightforward for batchwise reprocessing and can be approached in several different ways.
One method is to have the net mass of the feed rate over some depletion step be equivalent to the mass removed over that same depletion step, which is the current approach used by SaltProc \cite{rykhlevskii_modeling_2019}. Alternatively, the volume of the system can be adjusted so that constant density is maintained in order to keep cross section data consistent \cite{ridley_method_2017}. Another approach is to move excess fuel salt into a bleed-off tank \cite{ridley_method_2017}.

For mass balancing using continuous reprocessing, the depletion time steps can be much larger than those used in batchwise reprocessing.
This can cause a mass balancing issue if the system has a large feed rate but a small removal rate.
This could cause the simulation to continuously increase in mass, thus containing more fuel salt than intended and generating invalid results.
One way this can be treated, assuming the net change in mass in non-negligible, is to use batchwise reprocessing to balance the mass while continuous reprocessing handles the actual reprocessing.
%Solutions to this issue are to use balanced feed and removal rates, check the mass of the simulated core to ensure it is balanced over time, or use batchwise reprocessing at certain times to rebalance the core.
Implementing batchwise reprocessing increases computational cost when performed in Serpent2. This is because using a single input script with depletion steps will cause each depletion step to be run once, while altering the input script by changing volumes or removing mass to a bleed-off tank will cause the time step to be run twice.

\subsubsection{Batchwise Reprocessing}

%Imagine a process exists in which $X$\% of some element is removed from the fuel salt in the system over some time period T. For a batchwise approach, a straightforward approach would be to run the system, deplete for some time step $\Delta t$, remove $\gamma$\%, and repeat. The scaling term $\gamma$ for the removal efficiency is a linear approximation of the removal efficiency based on the time step adjustment and can be seen in Equation \ref{eq:batchwise_rem_eff}. An alternative approach is to establish a depletion time step $\Delta t$ such that the efficiency based time value $T$ is a multiple of $\Delta t$ and only have the batchwise removal occur at the steps when $\Delta t$ is a multiple of $T$.

%\begin{equation} \hfill
%\gamma = X \frac{\Delta t}{T}
%\hfill\label{eq:batchwise_rem_eff} \end{equation}
Batchwise reprocessing starts with some removal rate where some fraction, $f$, of a given element is removed from the fuel salt in the system over some time period, $T_{cyc}$, referred to as the cycle time. Over a given depletion simulation time step, $\Delta t$, a different fraction, $\gamma$, of the element's mass is removed, where $\gamma$ corresponds to $f$ according to Equation (\ref{eq:batchwise_rem_eff}). The scaling term on $f$ for the removal efficiency is a linear approximation based on the time step adjustment in Equation (\ref{eq:batchwise_rem_eff}). If the time step used causes the fraction removed, $\gamma$, to be greater than 1, the amount removed remains at 100\% so as to not have negative mass. This method is called Steady batchwise reprocessing since it uses a steady removal amount during each depletion step.

An alternative approach is to establish a depletion time step, $\Delta t$, such that the cycle time, $T_{cyc}$, is a multiple of $\Delta t$. Then, the fractional batchwise removal, $f$, occurs at the steps when $\Delta t$ is a multiple of $T_{cyc}$, removing the need for scaling. This method is called Bulk batchwise reprocessing because it performs reprocessing all at once in a bulk manner.

\begin{equation} \hfill
\gamma = f \frac{\Delta t}{T_{cyc}}
\hfill\label{eq:batchwise_rem_eff} \end{equation}

One example of implementing batchwise reprocessing can be seen in the depletion simulation of the Molten Salt Breeder Reactor using SaltProc \cite{rykhlevskii_modeling_2019}. This simulation uses batchwise reprocessing every three days with a linear approximation of the cycle times given by Robertson et al in the ORNL report \cite{robertson_conceptual_1971, rykhlevskii_modeling_2019}. The xenon and krypton removals in this work use a gas separation system equation, while the protactinium uses a modeled liquid-liquid reductive extraction process. These efficiencies differ from the cycle time values, causing slight differences in those terms. The MSBR also has two varying feed inputs of \ce{^{232}Th} and \ce{^{233}U}.

Examples of the Steady batchwise method used in SaltProc can be seen in Equations \eqref{eq:SP_20s} and \eqref{eq:SP_60d}. Both examples show what the resulting mass removal per step, or $\gamma$, term should be for different cycle time values. The first shows a cycle time of 20s, which is shorter than the step time of three days. This results in a removal rate over 100\%, which is set to 100\%. The second shows a cycle time of 60 days, longer than the step size. This results in a 5\% removal per step. An alternative approach to the second example, which is implemented in the earlier versions of SaltProc, is to perform no removal until 60 days have elapsed and then remove 100\% of material.


\begin{equation} \hfill
f = 100\%, T = 20 s, \Delta t = 3 d \rightarrow \gamma > 100\% \rightarrow 100\%
\hfill\label{eq:SP_20s} \end{equation}

\begin{equation} \hfill
f = 100\%, T = 60 d, \Delta t = 3 d \rightarrow \gamma = 5\%
\hfill\label{eq:SP_60d} \end{equation}

\subsubsection{Continuous Reprocessing}

%For a continuous approach, removing some fraction is dependent upon the particular isotope since the approach involves adding a term to the Bateman equation. This is one of the methods used in Serpent2 continuous reprocessing, where separations are only performed based on the element, not each particular isotope, an extension developed by Aufiero et al \cite{aufiero_extended_2013}. Equation \ref{eq:Bateman_default} shows the generic form of the Bateman equation, while Equation \ref{eq:Bateman_type_one_decay} shows the terms added to Equation \ref{eq:Bateman_default}, which are the reprocessing constants, or $\lambda_{r}$ added.
The continuous reprocessing method involves directly modifying the Bateman equation, shown in Equation \eqref{eq:Bateman_default}. One of the continuous reprocessing methods used in Serpent2 continuous reprocessing, an extension developed by Aufiero et al, is henceforth refered to as "decay" continuous reprocessing \cite{aufiero_extended_2013}. Equation \eqref{eq:Bateman_default} shows the Bateman equation with production and loss exclusively from fission, absorption, and decay; and Equation \eqref{eq:Bateman_type_one_decay} shows the Bateman equation with the removal from reprocessing subtracted and addition from feed sources added.
%Equation (\ref{eq:Bateman_type_one_decay}) shows the terms added to Equation (\ref{eq:Bateman_default}), which are the reprocessing constants, $\lambda_{r}$, which is of the same form of radioactive decay, hence the name "decay" continuous reprocessing. These terms represent removal of the current isotope as well as feed of the current isotope from other sources.

\begin{equation} \hfill
\begin{split}
%\frac{dN}{dt} = \sum_j \lambda_j N_j + \gamma \Sigma_f \phi + \Sigma_k \phi - \lambda N - \Sigma \phi - C
\frac{dN_j}{dt}_{base} = \sum_{i \neq j}  & \left [ \left( \gamma_{i \rightarrow j} \sigma_{f, i} \Phi + \lambda _{i \rightarrow j} + \sigma_{i \rightarrow j} \Phi \right) N_i \right ]\\
 & - \left ( \lambda_j + \sigma_j \Phi \right ) N_j
\end{split}
\hfill\label{eq:Bateman_default} \end{equation}

\begin{equation} \hfill
%\frac{dN}{dt} = \sum_j \lambda_j N_j + \gamma \Sigma_f \phi + \Sigma_k \phi - \lambda N - \Sigma \phi - C
%\frac{dN_j}{dt} = \sum_{i \neq j} \left [ \left( \gamma_{i \rightarrow j} \sigma_{f, i} \Phi + \lambda _{i \rightarrow j} + \lambda _{reproc, i \rightarrow j} + \sigma_{i \rightarrow j} \Phi \right) N_i \right ] - \left ( \lambda_j + \lambda_{reproc, j} + \sigma_j \Phi \right ) N_j
%\frac{dN_j}{dt} = \sum_{i \neq j} \left [ \left( \gamma_{i \rightarrow j} \sigma_{f, i} \Phi + \lambda _{i \rightarrow j} + \lambda _{r, i \rightarrow j} + \sigma_{i \rightarrow j} \Phi \right) N_i \right ] - \left ( \lambda_j + \lambda_{r, j} + \sigma_j \Phi \right ) N_j + \sum_{mat} \lambda _{r, i \rightarrow j} N_i
%\frac{dN_j}{dt}_{net} = \frac{dN_j}{dt}_{base} -  \lambda_{r, j} N_j + \sum_{mat} \lambda _{r, i \rightarrow j} N_i
\frac{dN_j}{dt}_{net} = \frac{dN_j}{dt}_{base} -  \lambda_{r, j} N_j + \sum_{feed} \lambda _{f, j} N_j
\hfill\label{eq:Bateman_type_one_decay} \end{equation}

The symbols given in the equations are defined as follows \cite{leppanen_development_2007}:
\begin{description}
\item[N_j] is the atomic density of isotope j $\left[\frac{atoms}{cm^3}\right]$.
\item[\gamma_{i \rightarrow j}] is the fractional fission product yield of $j$ in the fission of isotope $i$.
\item[\sigma_{f, i}] is the microscopic fission cross section of isotope $i$ $\left[cm^2\right]$.
\item[\Phi] is the spectrum-averaged scalar flux in the fuel region $\left[\frac{neutrons}{cm^2 s}\right]$.
\item[\lambda _{i \rightarrow j}] is the decay constant of decay $i \rightarrow j$ $\left[s^{-1}\right]$.
%\item[\lambda _{r, i \rightarrow j}] is the reprocessing constant for feed of $i \rightarrow j$
\item[\sigma_{i \rightarrow j}] is the microscopic transmution cross section of reaction $i \rightarrow j$ $\left[cm^2\right]$.
\item[N_i] is the atomic density of isotope $i$ $\left[\frac{atoms}{cm^3}\right]$.
\item[\lambda_j] is the decay constant of isotope $j$ $\left[s^{-1}\right]$.
\item[\lambda_{r, j}] is the reprocessing constant for removal of isotope $j$ $\left[s^{-1}\right]$.
\item[\sigma_j] is the microscopic total transmutation cross section of isotope $j$ $\left[cm^2\right]$.
\item[\lambda _{f, j}] is the feed constant for feed of isotope $i$ $\left[s^{-1}\right]$.
\end{description}

%It can be seen visually that the reprocessing removal essentially operates as increasing the decay rate when applied in this manner. For example, increasing the decay rate of a particular element in a given material in Serpent2 also requires another material to gain a feed rate equivalent to that decay rate. The feed rate of a particular material is given by summing the removal rates of other materials for each isotope.
Equation \eqref{eq:Bateman_type_one_decay} shows that the reprocessing removal has the same mathematical operation as the decay rate. Unlike decay, the isotopes removed by reprocessing instead are transfered to a different material, which operates as a feed for that material. This can be seen in the summation term, which sums over the different feeds. Figure \ref{fig:msr-outline-fig-show} shows how these reprocessing and feed constants fit into the reactor operation. Additionally, the figure shows that these terms relate to reprocessing and feed mass flow rates of $\dot{m}_{r, j}$ and $\dot{m}_{f, j}$, respectively.


\begin{figure}[H]
  \centering
  \includegraphics[scale=0.5]{images/msr-feed-repr-fig.png}
  \caption{Simplified model of how Bateman equation relates to mass flow rates.}
   \label{fig:msr-outline-fig-show}
\end{figure}


Serpent2 has three different settings for continuous reprocessing which alter the effect on the Bateman equations. In the Serpent2 documentation, the different settings are referred to as the "continuous reprocessing modes" which can be selected from the "reprocessor" keyword. The zeroth setting does not conserve mass and will not be discussed in this work.

The first setting, which adds a decay-like term to the Bateman equation, is shown in Equation (\ref{eq:Bateman_type_one_decay}). This approach of adding a decay-like term to the Bateman equation has been done with other codes \cite{jr_vicente_valdez_modeling_2020, rodriguez-vieitez_transmutation_2002}. For this setting, three different approaches are implemented in order to compare their results, and are discussed later in this work. This setting is the most important for this work, and is the only setting implemented.


%The first approach is labeled the Cycle Time Decay (CTD) approach. It takes half the value of the cycle times and treats them as half life values. A similar method of treating removal periods for reprocessing can be seen in Brovchenko et al \cite{brovchenko_neutronic_2019}. Equation (\ref{eq:CTDex}) shows how this approach works with a cycle time of 20 seconds. The operation cuts the cycle time in half and then calculates the decay constant, or reprocessing constant, using the half life from the cycle time.

%\begin{equation} \hfill
%20s \rightarrow \Delta t = 10s \rightarrow \lambda_{r} = \frac{ln(2)}{\Delta t} \approx 6.93E\minus 2
%\hfill\label{eq:CTDex} \end{equation}

%The next approach is labeled the SaltProc Cycle Rate (SPCR) approach. It takes the same efficiency rates used by SaltProc, which uses 3 day depletion steps for the MSBR, and directly converts those rates to continuous reprocessing rates. Taking the inverse of the cycle time is a common way to generate the removal rate \cite{rykhlevskii_modeling_2019, nuttin_potential_2005}. Equation (\ref{eq:SPCRex}) shows how this approach works with a cycle time of 20s. This operation uses the SaltProc value, which in this case becomes 100\% removal over 3 days, converts it to be per second, and then converts that to the rate form usable in Serpent2.

%\begin{equation} \hfill
%20s \rightarrow \frac{1}{3 d} \rightarrow X = \frac{4E \minus 6}{1 s} \rightarrow \lambda_r = ln \left( \frac{1}{1 - X} \right) \approx 4E \minus 6
%\hfill\label{eq:SPCRex} \end{equation}

%This natural logarithm in this approach comes from the differential equation solution based on the removal rate provided. The method used to derive this equation can be seen in Equations \eqref{eq:nat_log_1}, \eqref{eq:nat_log_2}, and \eqref{eq:nat_log_3}. These equations assume that the $X$ term, or the fractional removal rate, is in units of per second.

%\begin{equation} \hfill
%\frac{dN}{dt} = -\lambda_r N
%\hfill\label{eq:nat_log_1} \end{equation}

%\begin{equation} \hfill
%N(t) = (1-X) N_{cur} = N_{cur} e^{-\lambda_r}
%\hfill\label{eq:nat_log_2} \end{equation}

%\begin{equation} \hfill
%-ln(1-X)  = \lambda_r = ln\left( \frac{1}{1-X} \right)
%\hfill\label{eq:nat_log_3} \end{equation}

%The final approach is labeled the Cycle Rate (CR) approach. It takes the cycle times and uses a linear approximation to generate the efficiency rate directly. Equation (\ref{eq:CRex}) shows how this approach works using a cycle time of 20 seconds. This operation takes the cycle time, determines the efficiency rate for 100\% removal over 20 seconds, and then uses that value to calculate the Serpent2 reprocessing constant.

%\begin{equation} \hfill
%20s \rightarrow X = \frac{1}{20 s} \rightarrow \lambda_r = ln \left( \frac{1}{1 - X} \right) \approx 5.12E\minus 2
%\hfill\label{eq:CRex} \end{equation}

%Equations \ref{eq:Bateman_diff_removed} and \ref{eq:Bateman_X_removed} show the relationship between the net removal amount and the efficiency. Equations \ref{eq:Bateman_diff_control} through \ref{eq:Bateman_soln_control} show what the $\lambda$, or reprocessing constant, value would need to be set to such that, at the end of the simulation time $t_f$, the overall removal fraction is equal to $X$. These equations are a demonstration for a case in which there is no decay or production, only some initial amount that is removed continuously.

%\begin{equation} \hfill
%\frac{dN_{removed}}{dt} = \lambda N
%\hfill\label{eq:Bateman_diff_removed} \end{equation}

%\begin{equation} \hfill
%X = \frac{N_{removed}}{N_0 + N_{produced}}
%\hfill\label{eq:Bateman_X_precalc} \end{equation}


%\begin{equation} \hfill
%\frac{dN}{dt} = -\lambda N
%\hfill\label{eq:Bateman_diff_control} \end{equation}

%\begin{equation} \hfill
%N(t) = N_0 e^{-\lambda t}
%\hfill\label{eq:Bateman_eqn_control} \end{equation}

%\begin{equation} \hfill
%X = \frac{N_{removed}}{N_0 + N_{produced}} = 1 - \frac{N_f}{N_0}
%\hfill\label{eq:Bateman_X_removed} \end{equation}

%\begin{equation} \hfill
%\lambda = \frac{ln(\frac{1}{1-X})}{t_f}
%\hfill\label{eq:Bateman_soln_control} \end{equation}

%\begin{equation} \hfill
%X = 1 - e^{-\lambda t_f}
%\hfill\label{eq:Bateman_soln_control} \end{equation}


%A different isotope which naturally decays with a decay constant of $\lambda_d$ will have a different set of equations, which can be seen in Equations \ref{eq:Bateman_diff_decay} through \ref{eq:Bateman_eqn_decay_X}. It can be seen that if a value of 0 is used in Equation \ref{eq:Bateman_eqn_decay_X} for the decay constant, then the value matches the result in Equation \ref{eq:Bateman_soln_control}.

%\begin{equation} \hfill
%\frac{dN}{dt} = -\lambda N - \lambda_d N
%\hfill\label{eq:Bateman_diff_decay} \end{equation}

%\begin{equation} \hfill
%N(t) = N_0 e^{-(\lambda + \lambda_d) t}
%\hfill\label{eq:Bateman_eqn_decay} \end{equation}


%\begin{equation} \hfill
%\frac{dN_{removed}}{dt} = \lambda N = \lambda N_0 e^{-(\lambda + \lambda_d) t}
%\hfill\label{eq:Bateman_diff_decay_removed} \end{equation}

%\begin{equation} \hfill
%N_{removed}(t) = \frac{\lambda N_0}{\lambda + \lambda_d} \left( 1 - e^{-(\lambda + \lambda_d) t} \right)
%\hfill\label{eq:Bateman_eqn_decay_removed} \end{equation}

%\begin{equation} \hfill
%X = \frac{N_{removed}}{N_0 + N_{produced}} = \frac{\lambda}{\lambda + \lambda_d} \left( 1 - e^{-(\lambda + \lambda_d) t} \right)
%\hfill\label{eq:Bateman_eqn_decay_X} \end{equation}

%Although the results match when the decay is removed, while the decay is present, the results cannot be equivalent. This demonstrates one of the flaws of using continuous reprocessing, which is that the effects of the same reprocessing efficiency, $X$, on different isotopes of the same element requires unique reprocessing constants, $\lambda$, for each isotope in order for the removal efficiencies to be equivalent. Therefore, operating a continuous reprocessing scheme using a single removal constant for an element will fundamentally be non-physical due to the variances in removal efficiencies. The current system in Serpent2 allows for a removal constant to be applied to all isotopes of that element, which can circumvent this issue.

The second continuous reprocessing setting used in Serpent2 removes a constant value from each isotopic Bateman equation. Since the chemistry provides a removal rate for elements, the straightforward approach would be to generate a removal rate for each element. This approach can be seen in Equation \ref{eq:Bateman_diff_second_example}, where the $C$ term represents the constant value being removed by continuous reprocessing.

\begin{equation} \hfill
%\frac{dN}{dt} = \sum_j \lambda_j N_j + \gamma \Sigma_f \phi + \Sigma_k \phi - \lambda N - \Sigma \phi - C
\frac{dN_j}{dt} = \sum_{i \neq j} \left [ \left( \gamma_{i \rightarrow j} \sigma_{f, i} \Phi + \lambda _{i \rightarrow j} + \sigma_{i \rightarrow j} \Phi \right) N_i \right ] - \left ( \lambda_j + \sigma_j \Phi \right ) N_j - C
\hfill\label{eq:Bateman_diff_second_example} \end{equation}

This approach has a flaw which can be directly demonstrated through an example if the reprocessing constants are defined for an element and not for each isotope. Imagine an element exists with two isotopes $\alpha$ and $\beta$. There is a removal rate of 10\% per second defined for the element. A batchwise process could remove 10\% of each isotopes relative abundance after a second by using a one second time step, which is physical in terms of fractional removal. This can be seen directly in Table \ref{tab:cont_repr_appr}, where 10\% of both isotopes and the net count are removed.

\begin{table}[ht]
\renewcommand{\arraystretch}{1.25}
\caption{Serpent2 Second Setting Approaches}
\label{tab:cont_repr_appr}
\begin{center}
\begin{tabular}{ c | c | c | c | c | c | c }
 \hline
 Labels & Initial & Batch & Both Fit & $\alpha$ Fit & $\beta$ Fit & Zeroing Fit\\
 \hline
 \hline
$\alpha$ [g] & 1000 & 900 & 949.5 & 900 & 999 & 990\\
$\beta$ [g] & 10 & 9 & -40.5 & -90 & 9 & 0\\
 \hline
 Net [g] & 1010 & 909 & 909 & 810 & 1008 & 990 \\
 \hline
\end{tabular}
\end{center}
\end{table}

For the continuous reprocessing, both isotopes use the same constant removal rate $C$.%however, the same constant removal rate, $C$, is defined for both of the isotopes and is the same value.
To perform reprocessing, the first attempt might be to try and remove 10\% of the net by splitting it amongst the isotopes evenly. The issue with this approach is immediately noticeable, which is that the concentration of isotope $\beta$ becomes negative, as seen in the "Both Fit" column. A logical next approach would be to try and make one of the isotopes exactly correct while bringing the other along with it, which can be seen in the $\alpha$ and $\beta$ Fit columns. The results of this approach are seen in Table \ref{tab:cont_repr_appr}, and it can be seen that only by looking at the isotope with the smallest concentration and using that to determine the amount to remove can the result be ensured to be non-negative. One final approach to consider is to set the smallest concentration to 0, and bring the other isotopes along. This approach can be seen in the "Zeroing Fit" column, and it also does not work very well.

This analysis shows that, for this reprocessing setting to function properly, the reprocessing constants would need to be generated for every isotope individually in order to establish an online reprocessing scheme. However, this would require collecting data on the quantity of each isotope and calibrating reprocessing constants at every depletion time step. Stopping the Serpent2 depletion run to collect this data and then restarting would cause this approach to have similar computational cost to a batchwise reprocessing method, as Serpent would need to run each time point twice due to the restarts.

%Previously, this continuous reprocessing functionality in Serpent2 was undocumented, which led to difficulties in its usage by users \cite{rykhlevskii_modeling_2019}. Recently, however, the Serpent2 documentation was updated by this work as well as work performed at The University of Tennessee with Dr. Chvala and Alex Wheeler.

%Overall, it can be seen that in the two different approaches used in Serpent2, neither is able to maintain physicality like batchwise reprocessing does. However, different approximations can be used in order to apply the Serpent2 continuous reprocessing. Additionally, the way continuous reprocessing is applied can vary. Different uses could be using batchwise reprocessing for a fine mesh and continuous reprocessing for a coarse mesh, where the fine mesh is applied at beginning of cycle (BOC) and end of cycle (EOC) while the coarse mesh is used during steady-state (SS) operations. Another alternative is that the continuous reprocessing could be focused on certain isotopes, such as xenon-135, in order to allow for larger batchwise steps to be used.

\subsection{DNP Movement}

The movement of DNPs in liquid fueled molten salt reactors impacts the operation of the reactor \cite{wooten_review_2018, aufiero_calculating_2014}. %directly since the decay of the precursors occurs at a different location compared to a static reactor \cite{wooten_review_2018, aufiero_calculating_2014}.
There are several effects which this movement can have on the reactor. %Since the delayed neutron precursors move,
There is a probability of decay in non-core regions of the reactor, meaning there are fewer delayed neutrons in the core. This results in a reduced effective delayed neutron fraction, which reduces controllability of the reactor. Additionally, since delayed neutrons have a softer spectrum than prompt neutrons, there is a difference in spectrum in a reactor which has no flow against a reactor which has a circulating fuel.
In a solid fueled reactor, the flux and the concentration of delayed neutron precursors have the same profile. However, a moving fluid fuel causes the precursors to have a shifted profile based on the fluid flow rate, mixing, and decay rate of the precursors. This was shown by Jun Shi and Massimiliano Fratoni in their work, which can be seen in Figure \ref{fig:genfoam_dnp_locations} \cite{shi_gen-foam_2021}.

%\begin{figure}[H]
%  \centering
%  \includegraphics[scale=0.25]{images/dnp1.PNG}
%  \includegraphics[scale=0.25]{images/dnp3.PNG}
%  \includegraphics[scale=0.25]{images/dnp6.PNG}
%  \caption{Plots of DNP concentrations in the molten salt reactor experiment \cite{shi_gen-%foam_2021}. The left side of each image shows concentration with no flow, while the right shows the %concentration with a 1200 gallon per minute flow rate. Top Left) Longest lived DNP group. Top Right) %Third longest lived DNP group. Bottom) Shortest lived DNP group.}
%   \label{fig:genfoam_dnp_locations}
%\end{figure}

\begin{figure}[H]
\centering
\begin{subfigure}{0.4\textwidth}
    \includegraphics[scale=0.25]{images/dnp1.PNG}
    \caption{Longest lived delayed neutron precursor group.}
    \label{fig:first}
\end{subfigure}
\hfill
\centering
\begin{subfigure}{0.4\textwidth}
    \includegraphics[scale=0.25]{images/dnp3.PNG}
    \caption{Third longest lived delayed neutron precursor group.}
    \label{fig:second}
\end{subfigure}
\hfill
\centering
\begin{subfigure}{0.4\textwidth}
    \includegraphics[scale=0.25]{images/dnp6.PNG}
    \caption{Shortest lived delayed neutron precursor group.}
    \label{fig:third}
\end{subfigure}

\caption{Plots of DNP concentrations in the molten salt reactor experiment \cite{shi_gen-foam_2021}. The left side of each image shows concentration with no flow, while the right shows the concentration with a 1200 gallon per minute flow rate.}
\label{fig:genfoam_dnp_locations}
\end{figure}



Figure \ref{fig:genfoam_dnp_locations} shows that for the shortest lived DNPs, the movement of the fuel does not have any significant impact upon the DNP distribution when compared to static fuel. This is reasonable since the precursors do not live for very long in this group, so they do not have much chance to travel. For the intermediate DNP group, there is a clear shift in the direction of the fuel flow, which causes more precursors to produce delayed neutrons in the less important piping  and upper plenum regions. The longest lived DNP group seems to be spread almost equally throughout the entire reactor, which means that those delayed neutrons are contributing significantly less to the overall neutron economy. This reduces the overall effective delayed neutron fraction since the delayed neutrons are in less important regions. This directly affects the controllability of the reactor through the reactor period, which is heavily dependent upon the delayed neutrons.

\section{Depletion}

Depletion is the process of burning the fuel and simulating the changes to the materials this causes in the system. Depletion requires specific information to run properly, such as decay constants,  fission yields, and fission and transmutation cross section data. The fission and transmutation cross section data can come from a transport calculation, while the other data can be read from a data library \cite{leppanen_development_2007}. Additionally, running depletion involves solving the Bateman equations, a generic version of which can be seen in Equation \ref{eq:Bateman_default}.

There are several different approaches to solving these equations, such as the Transmutation Trajectory Analysis (TTA) method and the Chebyshev Rational Approximation Method (CRAM). These are the two different methods used in Serpent2 %, and are both built into Serpent2
\cite{leppanen_serpent_2015}. Other codes may use external software to compute depletion, such as REM used by MCNP and ORIGEN-S used by KENO-VI \cite{aufiero_extended_2013}. Another code %, aside from Serpent2,
that contains built-in depletion solvers is ERANOS \cite{aufiero_extended_2013}.

The purpose behind solving these equations and having depletion models is to generate an accurate representation of the composition of the target materials during reactor operation. This is important in determining how long the reactor can run, how the safety of the reactor develops over time, and how operation may have to change to adjust to the new reactor state. For molten salt reactors, depletion also allows for information regarding fresh fuel feed rate and fission product removal rates, as well as how variations of those parameters can affect reactor performance and behaviour.

% discuss how Serpent implements depletion and how Aufiero implemented repr

%\subsection{CRAM}

% https://serpent.vtt.fi/serpent/publications.htm only discusses CRAM

%\subsection{TTA}

%\subsection{Reprocessing}

% Reference Aufiero

\section{The Molten Salt Breeder Reactor}

The molten salt breeder reactor (MSBR) is a useful design to analyze for several reasons. Because it is a molten salt reactor, online reprocessing is used within the design.
In addition, it contains an inner and outer zone, zone I and II, respectively, each of which has different neutronic behaviour \cite{robertson_conceptual_1971}.
A portion of the core geometry is shown in Figure \ref{fig:msbr_ryklev}.
More specifically, the inner zone has a softer spectrum, a higher fission rate, and a 13\% fuel-to-graphite ratio \cite{park_whole_2015}.
The outer zone has a harder spectrum, a higher breeding rate, and a 37\% fuel-to-graphite ratio.
The differences between the inner and outer zones prove to be an issue when accurately modeling using a unit-cell or one-region approach, as those models are unable to capture the different characteristics of each region \cite{rykhlevskii_modeling_2019}.
The geometric differences in the regions can also be seen in Figure \ref{fig:msbr_ryklev}, which allows for the difference in the fuel-to-graphite ratio to be visually noticeable.

\begin{figure}[H]
  \centering
  \includegraphics[scale=0.25]{images/msbr_ryk_1.PNG}
  \caption{MSBR core axial slice showing the different regions from \cite{rykhlevskii_modeling_2019}. Zones II-A and II-B are where the spectrum is harder and there is increased breeding. Zone I is where there is more fission and a softer spectrum. The yellow is fuel salt, the purple is graphite, and the cyan is the reactor vessel.}
   \label{fig:msbr_ryklev}
\end{figure}

Table \ref{tab:msbr_gen_data} has general information on the MSBR, while Table \ref{tab:msbr_salt_data} has specific information on the distribution of the fuel salt during operation. It can be seen in Table \ref{tab:msbr_gen_data} that the salt inventory cycle time is 10 days, which means that the entire salt inventory is cycled in 10 days. However, the loop cycle time, or time it takes for the salt to complete a full loop in the primary system, is 11 seconds \cite{robertson_conceptual_1971}.

\begin{table}[H]
\renewcommand{\arraystretch}{1.25}
\caption{Summary of Principle Data for MSBR \cite{robertson_conceptual_1971}}
\label{tab:msbr_gen_data}
\begin{center}
\begin{tabular}{ c | c }
 \hline
 Component & Data\\
 \hline
 \hline
 Thermal Capacity & 2250 MW$_{th}$\\
 Vessel Inner Diameter & 6.77 m\\
 Core Height & 3.96 m\\
 Vessel Pressure & 0.52 MPa\\
 Central Core Zone Salt Fraction & 0.13\\
 Outer Core Zone Salt Fraction & 0.37\\
 Maximum Flow Velocity (Core) & 2.6 $\frac{m}{s}$\\
 Fuel Salt (Vessel) & 30.4 m$^3$\\
 Fuel Salt (Primary System) & 48.7 m$^3$\\
 Thorium Inventory & 68,100 kg\\
 Breeding Ratio & 1.06\\
 Processing Rate & 1 gpm\\
 Salt Inventory Cycle Time & 10 days\\
 Salt Components & $\ce{^{7}LiF-BeF_{2}-ThF_{4}-UF_{4}} $\\
 Salt Composition & 71.7-16-12-0.3 mole \% \\
 \hline
\end{tabular}
\end{center}
\end{table}

Table \ref{tab:msbr_salt_data} shows that the salt inventory in the core is approximately 19 m$^3$, with 8.2 m$^3$ from zone I and 10.8 m$^3$ from zone II.
%However, the MSBR model provided in SaltProc uses the volume of the primary system for its calculation \cite{rykhlevskii_modeling_2019}. This is an approximation which allows the depletion simulation to use the correct amount of net material, though it also means that material which would not be irradiated in the piping is present in the core. The core geometry uses the correct dimensions, which allows cross sections to remain as if the correct core volume were implemented.
%Overall, the depletion results are closer to the expected result if the fuel salt were fully utilized.


\begin{table}[H]
\renewcommand{\arraystretch}{1.25}
\caption{MSBR Primary System Salt Inventory \cite{robertson_conceptual_1971}}
\label{tab:msbr_salt_data}
\begin{center}
\begin{tabular}{  c | c  }
 \hline
 Region & Volume [m$^3$]\\
 \hline
 \multicolumn{2}{c}{Reactor}\\
  \hline
 Core Zone I & 8.2\\
 Core Zone II & 10.8\\
 Plenums, Inlets, Outlets & 6.2\\
 Annulus & 3.8\\
 Reflectors & 1.4\\
 \hline
\multicolumn{2}{c}{Primary Heat Exchangers}\\
 \hline
 Tubes & 7.6\\
 Inlets, Outlets & 0.8\\
  \hline
 \multicolumn{2}{c}{Miscellaneous}\\
  \hline
 Pump Bowls & 5.2\\
 Piping (including drain line) & 4.1\\
 Off-gas bypass loop & 0.3\\
 Tank heels and miscellaneous & 0.3\\
 \hline
  \multicolumn{2}{c}{Total}\\
  \hline
 Fuel Salt (Primary System) & 48.7\\
 \hline
\end{tabular}
\end{center}
\end{table}


\begin{table}[H]
\renewcommand{\arraystretch}{1.25}
\caption{MSBR Online Reprocessing Cycle Times \cite{robertson_conceptual_1971}}
\label{tab:msbr_cycle_times}
\begin{center}
\begin{tabular}{ p{0.27\linewidth} | p{0.25\linewidth} | p{0.35\linewidth} }
 \hline
 Reprocessing Group & Element(s) & Cycle Time (Full Power)\\
 \hline
 \hline
 Rare Earths & Y, La, Ce, Pr, Nd, Pm, Sm, Gd & 50 days\\
 Rare Earths & Eu & 500 days\\
 Noble Metals  & Se, Nb, Mo, Tc, Ru, Rh, Pd, Ag, Sb, Te & 20 seconds\\
 Seminoble Metals & Zr, Cd, In, Sn &  200 days\\
 Gases & Kr, Xe & 20 seconds\\
 Volatile Fluorides & Br, I & 60 days\\
 Discard & Rb, Sr, Cs, Ba & 3435 days\\
 Salt Discard & Th, Li, Be, F & 3435 days\\
 Protactinium & Pa & 3 days\\
 Higher Nuclides & Np, Pu & 16 years\\
 \hline
\end{tabular}
\end{center}
\end{table}


%Pa233, Np237, Pu242

Shown in Table \ref{tab:msbr_cycle_times} are the cycle times for different elements in the MSBR.
Here I define the cycle time as the time it takes for a given element to be completely removed, which is a common way to define the cycle time in the literature \cite{betzler_molten_2017, rykhlevskii_modeling_2019, robertson_conceptual_1971}.
According to Robertson et al, the protactinium and rare earth processing has the largest impact on neutronics and performance \cite{robertson_conceptual_1971}.

The fuel processing of the MSBR is defined to operate continuously with processing methods \cite{robertson_conceptual_1971}. For protactinium removal, fluorination is first used to remove uranium before protactinium isolation. Next is countercurrent bismuth with lithium and thorium for stripping any remaining uranium and the protactinium from the bismuth. This is followed by hydrofluorination to separate the uranium and protactinium. This process can be seen in Figure \ref{fig:msbr_pa}. For a 10 day protactinium cycle time, a fuel salt flow rate of 0.88 gallons per minute (gpm) is used. The processed salt is processed for rare earth fission products before being fed back into the reactor.
% Page 111 flow sheet
\begin{figure}[H]
  \centering
  \includegraphics[scale=0.55]{images/msbr_pa_robertson.PNG}
  \caption{MSBR protactinium processing scheme from Robertson et al \cite{robertson_conceptual_1971}.}
   \label{fig:msbr_pa}
\end{figure}

Rare earth fission product removal is performed using the metal-transfer process, which uses lithium chloride and bismuth containing a reductant. More specifically, bismuth containing a reductant of thorium and lithium is used to strip the rare earth fission products from the fuel salt. The rare earth fission products are then transported to the lithium chloride acceptor salt, though lithium bromide or a mix of both could be used. Figure \ref{fig:msbr_re} shows this process, and Figure \ref{fig:msbr_re_pa} shows the protactinium and rare earth processing schemes together.

\begin{figure}[H]
  \centering
  \includegraphics[scale=0.65]{images/msbr_re_robertson.PNG}
  \caption{MSBR rare earth processing scheme from Robertson et al \cite{robertson_conceptual_1971}.}
   \label{fig:msbr_re}
\end{figure}

\begin{figure}[H]
  \centering
  \includegraphics[scale=0.75]{images/msbr_flows_robertson.PNG}
  \caption{Combined MSBR rare earth and protactinium processing schemes from Robertson et al \cite{robertson_conceptual_1971}.}
   \label{fig:msbr_re_pa}
\end{figure}

To strip gaseous fission products, a bubble generator introduces helium sparging gas in 15 to 20 mil bubbles. 10 percent of the flow is redirected to the bypass loop which contains a gas separator. This gas separator strips the fission products with approximately 100 percent efficiency. The off-gas system also includes an approximately two hour holdup during which noble metals deposit on the fuel salt drain tank surface. Figure \ref{fig:msbr_nm_pa} shows this process, beginning with the number per time, or flux, of the noble gasses in the salt migrating to the graphite voids and helium sparging bubbles. This is followed by the volume holdup into more processing to handle the noble metal fission product cleanup.

\begin{figure}[H]
  \centering
  \includegraphics[scale=0.65]{images/msbr_nm_robertson.PNG}
  \caption{MSBR noble metal flux and off-gas system from Robertson et al \cite{robertson_conceptual_1971}.}
   \label{fig:msbr_nm_pa}
\end{figure}

%Periodic discard of salt at a rate of 0.1 cubic feet per day occurs due to buildup of non-volatile fluorides during fluorination accumulating in the decay tank, such as from zirconium and corrosion product nickel.

%From SaltProc's MSBR example, the removal of each group can be attributed to a particular process within the physical system \cite{rykhlevskii_modeling_2019}. The entrainment separator is used to remove the gases; a nickel mesh is used to remove noble metals, seminoble metals, and volatile fluorides; and various processes given by Robertson are lumped together into the liquid metal process which removes rare earths, discard, and protactinium \cite{robertson_conceptual_1971}. The higher nuclides are removed at such an infrequent rate that they are not included within the SaltProc processes. The MSBR design removes the protactinium with the intent that the protactinium-233 will not absorb neutrons and will instead be fed out of the core to decay into uranium-233, which can then be fed back into the core.


\section{MSR Modeling Approaches}
\label{s:models-methods}

For modeling of MSRs, it is important to model discrete and temporal behavior in core physics.
%The first takes a given fuel composition and analyzes the performance of the reactor given that fuel composition. This may include
With discrete modeling, one may aim to focus on aspects such as DNP movement \cite{fei_molten_2020, shi_gen-foam_2021} or reactor physics, such as kinetics and dynamics \cite{singh_plant-level_2020, cervi_development_2019, aufiero_development_2014, cui_development_2022, singh_dynamics_2017}.
This type of model typically is focused on the order of seconds.

For temporal modeling, the approach involves depleting the fuel, typically focusing on time scales ranging from days to years. Because the online reprocessing of an MSR has such large effects on its fuel composition during a long depletion simulation, these depletion simulations of MSRs typically incorporate a method to simulate online reprocessing.
These simulations can also account for other factors, such as DNP movement, as shown by Zhou et al \cite{zhou_fuel_2018}.
The movement of DNPs can have a noticeable effect on reactor performance by affecting neutron energy spectra and localized reactivity effects, though they do not have an impact when looking at longer time scales \cite{betzler_implementation_2017}.
One method used to approximate spatial dependence of fuel salt without modeling the piping of the core is the scaled flux method \cite{betzler_liquid-fueled_2021}.
Most depletion simulations, however, focus primarily on depletion and do not consider the DNP movement during depletion.

Table \ref{tab:codes_types} shows this in the "DNP" column, which is dominated by models which do not include DNPs.
The table is sorted by the "Reprocessing" column, which signifies the method(s) used in the different MSR models and is also sorted by the publication year in the "Year" column.
The table shows that there are many different MSRs, shown in the "Reactor" column, which have been modeled using a variety of spatial refinement levels, shown in the "Model" column, ranging from unit cells to full 3D models.

A key takeaway from Table \ref{tab:codes_types} is that batchwise reprocessing methods are commonly used to a similar extent as continuous reprocessing methods to handle MSR online reprocessing.
This is important because, as previously explained, the batchwise reprocessing methods are a less accurate way to simulate online reprocessing for a physically continuous scheme.
Therefore, because it is still common for works to use this method, it is important that the differences between continuous and batchwise methods are better understood.


For all of the continuous reprocessing models given in Table \ref{tab:codes_types}, all of them use the "fictitious decay constant" approach, which is shown in Equation \eqref{eq:Bateman_type_one_decay}. For the batchwise models, most use the Steady batchwise approach shown in Equation \eqref{eq:batchwise_rem_eff} and remove a fraction at each depletion step rather than bulk removal at certain steps.

%The "Reprocessing" column of Table \ref{tab:codes_types} shows the mathematical approach taken to the online reprocessing of the work, and does not necessarily represent the reactor reprocessing approach. However, since online reprocessing is considered in the table, offline batchwise reprocessing is not included, such as in the work by Zou et al \cite{zou_transition_2020}. The works shown in the table all use continuous reprocessing schemes, but some approximate the continuous physical process using batchwise methods.

%For modeling of MSRs, there are typically three main types of models which are used. The first is a model which takes a fuel composition and runs without accounting for reprocessing, but accounts for the fuel movement effects, such as DNP drift, shown by Fei et al \cite{fei_molten_2020}. The second is a model which accounts for both fuel movement and reprocessing, such as Zhou et al \cite{zhou_fuel_2018}. The third is a model which does not account for fuel movement but accounts for reprocessing. Models which use the second or third approach are detailed in Table \ref{tab:codes_types}.

%The purpose of this table is to show that batchwise approximations to continuous reprocessing schemes have been implemented historically and recently. These methods have been implemented for a range of various reactors, various model types and dimensions, use various depletion step sizes, and cover various net depletion times. Additionally, tracking delayed neutron precursors is not common in depletion calculations, primarily due to their negligible impact on depletion results.

In Sections \ref{litrev-msr-continuous}, \ref{litrev-msr-batchwise}, and \ref{litrev-msr-mixed}, I will discuss the results of each of the studies through the lens of the current work. That is, I will summarize the effects of using those depletion and reprocessing methods and note what differences may be observed from the literature. Then, in Section \ref{litrev-msr-gaps}, I will comment on the gaps that this work will address.

% Remove scale and step columns from table, include in description of model
\begin{table}[H]
\renewcommand{\arraystretch}{1.25}
\caption{Molten Salt Reactor Models Analyzed with Methods Approximating Online Reprocessing}
\label{tab:codes_types}
\begin{center}
\begin{tabular}{  c | c | c | c | c | c}
 \hline
 Publication & Reprocessing & Model & Reactor & DNP & Year\\
 \hline
 \hline
 Valdez et al \cite{jr_vicente_valdez_modeling_2020} & Continuous & Unit Cell & N/A & No & 2020\\
 Zhuang et al \cite{zhuang_extended_2020} & Continuous & Full 3D & MSFR & No & 2020\\
 Xia et al \cite{xia_development_2019} & Continuous & Full 3D & N/A & No & 2019\\
 Zhou et al \cite{zhou_fuel_2018} & Continuous & Full 3D & MSBR & Yes & 2018\\
 Aufiero et al \cite{aufiero_extended_2013} & Continuous & Full 3D & MSFR & No & 2013\\
 Rodriguez-Vieitez \cite{rodriguez-vieitez_transmutation_2002} & Continuous & Unit Cell & ADNA & No & 2002\\
 \hline

 Rykhlevskii \cite{rykhlevskii_modeling_2019} & Batchwise & Full 3D & MSBR & No & 2019\\
 Betzler et al \cite{betzler_molten_2017} & Batchwise & Unit Cell & MSBR & No & 2017\\
 Ridley and Chvala \cite{ridley_method_2017} & Batchwise & Full 3D & Toy & No & 2017\\
 Ahmad et al \cite{ahmad_neutronics_2015} & Batchwise & Unit Cell & DMSR & No & 2015\\
 Park et al \cite{park_whole_2015} & Batchwise & Full 3D & MSBR & No & 2015\\
 Sheu et al \cite{sheu_depletion_2013} & Batchwise & Full 3D & MOSART & No & 2013\\
 Nuttin et al \cite{nuttin_potential_2005} & Batchwise & Full 3D & MSBR & No & 2005\\

 \hline
 Fiorina et al \cite{fiorina_preliminary_2012} & Mixed & 2D Slice & MSFR & No & 2012\\
 Nagy et al \cite{nagy_parametric_2008} & Mixed & Unit Cell & Toy & Yes & 2008\\
 Merle-Lucotte et al \cite{merle-lucotte_thorium_2007} & Mixed & Full 3D & TMSR & No & 2007\\
 Bauman et al \cite{h_f_bauman_rod_1971} & Mixed & 2D Slice & N/A & Yes & 1971\\

 \hline
\end{tabular}
\end{center}
\end{table}

\subsection{Continuous Reprocessing}
\label{litrev-msr-continuous}

Table \ref{tab:codes_types} shows that many different codes have been used to simulate continuous reprocessing in MSRs. One common aspect between all of them is that they use long depletion step sizes (on the order of months) and long net depletion times (on the order of years).

Valdez et al \cite{jr_vicente_valdez_modeling_2020} demonstrates the efficacy of the updated version of ChemTriton. The version was added after TRITON implemented new tools for continuous material feeds and removals. Valdez et al discusses in depth how the removal rates for isotopes can be calculated using a well mixed approximation and accounting for non-modeled piping by introducing a correcting decay factor. Although it does not account for DNP movement, it does account for neutron important regions without having to model the piping regions.

Zhuang et al \cite{zhuang_extended_2020} discusses an OpenMC extension called OpenMCB-MSR to implement continuous reprocessing in an analysis of the MSFR. The depletion step sizes used are not specified in the work, though they appear to be on the order of decades, with a final depletion time of 200 years.
%with continuous reprocessing step sizes could have been set to above single digit days and still maintained high precision.

Xia et al \cite{xia_development_2019} demonstrates the MOlten salt reactor specific DEpletion Code (MODEC). This code uses continuous reprocessing, and has been used to analyze the Molten Chloride salt Fast Reactor (MCFR) by Liaoyuan et al \cite{liaoyuan_th-u_2021}. The MCFR analysis uses SCALE6.1 for transport and MODEC for depletion.

Zhou et al \cite{zhou_fuel_2018} covers FAMOS (Fuel cycle Analysis code for MOlten Salt reactors). This code uses OpenMC to generate homogenized cross section data followed by the DIF3D diffusion solver along with extensions to model DNP movement, thermal hydraulics, depletion, and continuous reprocessing. Zhou et al analyzes the molten salt fast reactor over 200 operational years and the molten salt breeder reactor for 20 years. The depletion step size is not specifically given in the work, but from analysis of the plots, small time steps seem to have been used at the beginning of cycle and then larger time steps of ~100 days are then used for the rest of the steps.

Aufiero et al \cite{aufiero_extended_2013} displays the Serpent2 newly developed continuous reprocessing functionality. A full 3D simplified core layout of the molten salt fast reactor (MSFR) is used, and continuous reprocessing is used to operate the removal and feed processes. It evaluates the model at various scales, from steady state uranium concentrations at over 60 years to radiotoxicity over $10^8$ years. The step sizes vary as well, though not specifically mentioned, appearing very small at beginning of cycle and expanding to around 10 years at steady state for one of the figures presented.

Rodriguez-Vieitez \cite{rodriguez-vieitez_transmutation_2002} analyzes the Accelerator-Driven Neutron Applications (ADNA) Tier-1 reactor using MCNP. Although the model Rodriguez-Vieitez used does implement continuous reprocessing in the Bateman equation, it is not used for the purpose of fuel cycle analysis and depletion. Rather, the work iteratively works towards the equilibrium fuel composition in order to analyze the effect of varying different reactor parameters, such as geometry.

\subsection{Batchwise Reprocessing}
\label{litrev-msr-batchwise}

The second category of studies summarized in Table \ref{tab:codes_types} use batchwise reprocessing. This method uses depletion step sizes on the order of several days, and uses net depletion times on the order of years.

Rykhlevskii \cite{rykhlevskii_modeling_2019} uses SaltProc to simulate the MSBR. Rykhlevskii uses Serpent2, a full 3D core of the MSBR, three day time steps, and runs for 60 years.
Rykhlevskii calculated the average thorium-232 refueling rate, compared safety parameters at beginning of cycle and equilibrium, and collected other useful data for understanding and further simulating the MSBR.
The batchwise reprocessing in SaltProc includes both the removal and feed processes.

Betzler et al \cite{betzler_molten_2017} uses ChemTriton, a tool designed for SCALE to perform online reprocessing of MSRs. It contains work on multiple reactors, such as the MSBR and the MSRE. The tool iteratively runs SCALE/TRITON over small time steps to simulate continuous reprocessing, and includes enhanced time-dependent feed and separations.
Betzler et al use this tool to optimize the thorium concentration to maintain a critical configuration while maintaining breeding capability, show the impact of fission product removal on core lifetime and fuel utilization, and more.
For the MSBR case, Betzler et al uses three day time, similar to Rykhlevskii and Park et al.
This time step is selected because previous parametric studies showed that it is sufficiently short enough to model the online removal of protactinium \cite{powers_new_2013}.
%The reason for not using shorter time steps is that using shorter time steps increases computational cost while also providing very little improvements to calculated eigenvalue and fuel cycle metrics, discussed by Betzler et al as well as by Powers et al \cite{powers_new_2013}.

Ridley and Chvala \cite{ridley_method_2017} use batchwise reprocessing on a toy MSR using Serpent2 and a custom in-house python library. The feed rates were set to fluctuate in order to maintain criticality while also accounting for density imbalance in Serpent2 by increasing the volume of the material. The full core 3D model was then analyzed using seven day depletion steps for a reactor operating period of ten years.

Ahmad et al \cite{ahmad_neutronics_2015} analyzes single fluid and two fluid denatured molten salt reactors (DMSRs). Ahmad et al use MCNP5 and ORIGEN2 as the codes to perform transport and depletion over 30 years. Although the steps at which fuel is adjusted is at five year intervals, the burnup intervals are set to six months. The reason such large steps were used is because the work analyzes proliferation and resource requirements for the reactors. This means that the neutronics and safety parameters were not the focus of the study. Additionally, only gaseous fission products and fission products removed via sparging are included in the reprocessing scheme. No other chemical extraction takes place.
%This is because the authors wanted the system to be as simple as possible. ORIGEN provides continuous reprocessing capability, however for this work it would not have been necessary since the authors wanted to use a simpler approach for the model \cite{gauld_isotopic_2011}.

Park et al \cite{park_whole_2015} is very similar to the MSBR example in SaltProc, with the primary difference being MCNP6 is used for transportation calculations instead of Serpent2. The depletion was performed using CINDER90 and custom Python scripts. Another difference is that SaltProc uses adjusted values for the removal efficiency rates of xenon, krypton, and protactinium, while the work by Park et al uses the cycle time values from Robertson et al \cite{robertson_conceptual_1971}. This model uses three day time steps and models the fuel composition over 20 years.

Sheu et al \cite{sheu_depletion_2013} analyzes the Molten Salt Actinide Recycler \& Transmuter (MOSART) reactor and uses custom scripts to implement batchwise reprocessing. The codes used are SCALE6 for transport and TRITON for depletion. The step sizes used include 15, 30, and 60 days, in which the authors found that 30 days gave the best results in terms of accuracy and run time balancing for the 30 year simulation.

Nuttin et al \cite{nuttin_potential_2005} analyzes the MSBR using a slightly simplified geometry with MCNP. The reprocessing approach used here is also the same as the approach used in SaltProc, which is to take the inverse of the cycle time to determine the removal rate. Depletion is performed in this work by coupling MCNP to an evolution program and the NJOY code. However, Nuttin et al use ten day steps instead of three day steps that SaltProc uses. It also runs for 100 years instead of 20.

\subsection{Mixed Batchwise and Continuous Reprocessing}
\label{litrev-msr-mixed}

A mixture of batchwise and continuous methods can be implemented if the reprocessing scheme includes physically batchwise processes or if mass balancing is an issue for the model.

Fiorina et al \cite{fiorina_preliminary_2012} is on the Molten Salt Fast Reactor (MSFR) and uses a modified version of the ERANOS-based EQL3D procedure. Fiorina et al make use of continuous reprocessing for non-soluble fission products, but uses batchwise reprocessing for soluble fission products. The batchwise reprocessing was incorporated for simplicity of implementation in the model. The time steps used are shorter at the beginning of cycle and extend out to roughly 30 year steps for a 200 year depletion. Additionally, the authors also performed a radiotoxicity analysis which extends out to one million years.

Nagy et al \cite{nagy_parametric_2008} uses a toy model to go through different reprocessing schemes based on the MSBR and MOSART reprocessing designs \cite{robertson_conceptual_1971, ignatiev_progress_2007}. SCALE-5 XSDRN\cite{noauthor_scale_2005} is used for neutron transport and ORIGEN-S is used for depletion. To account for DNPs, the authors assume that one-third of the delayed neutron precursors decay outside the core.
Using a delayed neutron fraction of 0.0027 for $^{233}$U, the authors claim that 0.0009, or 90 pcm, should be subtracted from the eigenvalue to approximate the impact from the movement of delayed neutron precursors.
Depletion steps are performed every five days, during which continuous reprocessing was implemented for only gaseous fission product removal, while refueling and other fission products used batchwise reprocessing.
The batchwise reprocessing was executed by altering the binary output of ORIGEN.

Merle-Lucotte et al \cite{merle-lucotte_thorium_2007} covers depletion of the Thorium Molten Salt Reactor (TMSR) using MCNP4 \cite{goluoglu_software_1998} for transport and REM for material evolution. Continuous reprocessing is used for gaseous fission product extraction, while the other fission products are processed with batchwise reprocessing.

Bauman et al \cite{h_f_bauman_rod_1971} presents the Reactor Optimum Design (ROD) code for circulating-fuel reactors.
ROD allows users to use continuous or batchwise reprocessing for MSRs. ROD operates by taking a one or two dimensional input and determining the equilibrium fuel composition. This is a useful code for determining a general prediction for a model, but is limited to one or two dimensional problems, a diffusion solver, 15 energy groups, and 200 fission products.
%less precise than more modern codes which offer more detailed geometry and other features.

\subsection{Discussion}
\label{litrev-msr-gaps}

Overall, from these works it can be seen that using continuous reprocessing allows for larger time steps.
These larger time steps allow for decreased computational cost, which enables the works which use continuous reprocessing to simulate results over a longer time period.
However, continuous reprocessing is more complex to implement than batchwise reprocessing.

For codes which do not have continuous reprocessing built-in, authors will find it more efficient to use a batchwise method which can be handled externally rather than implement continuous reprocessing into a code.
This is particularly the case if the authors have a time constraint and need to generate results quickly.
Implementation of continuous reprocessing into an existing code is a work in itself, as shown by authors such as Aufiero et al and Zhuang et al.

Thus, unless every depletion code used in MSR modeling is updated to handle continuous reprocessing, batchwise approximations will continue to be implemented.
Because this approximation is common and is likely to continue in the future, it is of increasing importance to determine the impacts of this approximation.
It is well understood that smaller step sizes are required for batchwise methods to approximate continuous methods, but there has yet to be a full analysis on how the results for an MSR using the two different methods compare.
This is the gap that this work seeks to fill.
I will show how the eigenvalue and various isotopic compositions vary while altering only the reprocessing methods implemented.


%Because the net depletion time is similar to that of continuous reprocessing while implementing much smaller depletion step sizes, this results in a higher computational cost.
% Gaps, best models, etc.

\section{SaltProc}
\label{s:theSaltProc}

\subsection{MSBR Model}

SaltProc comes with a full 3D model of the core of the MSBR created in Serpent2.
The MSBR model provided in SaltProc uses the volume of the primary system, 48.7 cubic meters, for its calculations rather than the 19 cubic meters of the core \cite{rykhlevskii_modeling_2019}.
This is an approximation which allows the depletion simulation to use the correct amount of net material, though it also means that material which would not be irradiated in the piping is present in the core.
The core geometry uses the correct dimensions, which allows cross sections to remain as if the correct core volume were implemented.

\subsection{Reprocessing}
\label{s:SP-repr}

SaltProc is used to handle batchwise reprocessing in this work because it uses Serpent2, which has built-in continuous reprocessing; comes with the MSBR geometry in Serpent2 as an example; and has multiple versions.
Of these different versions of SaltProc, they have slightly different methods in how the depletion data is stored as well as how reprocessing rates are calculated.
However, the two versions of interest are version 0.1 and 0.3, which use different batchwise reprocessing approaches.
These two different approaches are referred to here as the "Bulk" approach and the "Steady" approach.

SaltProc version 0.1 was the first full version of SaltProc publicly released, and provides all the functionality necessary for modeling the MSBR with batchwise reprocessing.
Using this version, Rykhlevskii uses three day depletion steps and does not perform fractional reprocessing at each step, instead performing reprocessing according to the cycle times \cite{rykhlevskii_advanced_2018}.
This approach is termed as the "Bulk" approach to batchwise reprocessing, and is explained in more detail below. This is the same approach used by Gehin and Powers for an analysis of the MSBR \cite{gehin_liquid_2016}.
This version of SaltProc, version 0.1, makes use of the h5py Python package to handle the hdf5 data files which contain depletion data \cite{collette_python_hdf5_2014}. Within the hdf5 file, the atom density of each isotope before and after batchwise reprocessing is recorded as well as different reactor parameters, such as the neutron multiplication factor.

SaltProc version 0.3 includes an example MSBR which users can run as soon as SaltProc is installed, though some of the run parameters such as neutrons per generation and run time have to be altered to generate a more valid model.
This model also uses three day steps, but instead of performing the batch removal only at the cycle time value, this model removes a fraction every three days, which is called the "Steady" approach for batchwise reprocessing.
Additionally, this version of SaltProc uses the PyTables Python package for handling the hdf5 data files. The data files are structured similarly to v0.1 of SaltProc, but instead of atom density they use mass, which makes analysis of the files slightly more user friendly.


\chapter{Methods}

\renewcommand*\descriptionlabel[1]{\hspace\leftmargin$#1$}
\setcounter{tocdepth}{5}
\setcounter{secnumdepth}{5}
\newcommand{\minus}{\scalebox{0.65}[1.0]{$-$}}

%\section{Notes}

%A 1-gpm
%(0.06 liter/sec) side stream of the primary salt is
%continuously processed to remove
% 233 P a , to recover the
%bred
% 233 U , and to adjust the fissile content. Robertson summary
 
% Robertson does not define thorium feed rate?
 
% Serpent2 continuous reprocessing section cite own presentations at Serpent user group meetings

\section{Batchwise Reprocessing}

SaltProc is used to handle batchwise reprocessing in this work because it provides a batchwise reprocessing scheme which can be customized fairly easily. There are different versions of SaltProc which have slightly different methods in how the depletion data is stored as well as how reprocessing rates are calculated.
%an accurate model of the MSBR, uses Serpent2 for transport and depletion calculations, and because it provides

\subsection{Bulk}

SaltProc version 0.1 is the first full version of SaltProc which was publicly released, and provides all the functionality necessary for modeling the MSBR with batchwise reprocessing. This version uses 3 day steps and does not perform fractional reprocessing at each step, but instead performs reprocessing according to the cycle times, which is termed as the "Bulk" approach to batchwise reprocessing. This is the same approach used by Gehin and Powers for an analysis of the MSBR \cite{gehin_liquid_2016}.

For example, a 30 day cycle time would be implemented as no reprocessing during the first 27 days, then full 100\% removal of the target at the 30 day mark. This works well for reactor processes which are performed batchwise, such as adding uranium to the reactor which can be performed as a batch process. However, for approximating online reprocessing, this approach is not ideal, as it is less able to capture the frequency of the reprocessing.

For the refueling of the reactor, the thorium feed rate was set to maintain the thorium mass in the reactor. The uranium feed rate was set to be equivalent to the protactinium removal rate. This carries an implicit assumption that there is already a backlog of decayed protactinium which has transmuted into uranium which is ready to be pumped back into the core. Both the thorium feed and protactinium to uranium refueling processes occur every 3 days. These refueling rates can be seen in Figures \ref{fig:Th-feed-v1} and \ref{fig:U-feed-v1}, which are recreated from the data from Rykhlevskii \cite{rykhlevskii_advanced_2018}.

% Include figure of U and Th feed rates over time.

\begin{figure}[H]
  \centering
  \includegraphics[scale=0.7]{images/Th232rem_massv01.png}
  \caption{Thorium feed rate in the MSBR as a function of time while using bulk batchwise reprocessing \cite{rykhlevskii_advanced_2018}.}
   \label{fig:Th-feed-v1}
\end{figure}

\begin{figure}[H]
  \centering
  \includegraphics[scale=0.7]{images/Pa233rem_massv01.png}
  \caption{Uranium feed rate in the MSBR as a function of time while using bulk batchwise reprocessing \cite{rykhlevskii_advanced_2018}.}
   \label{fig:U-feed-v1}
\end{figure}

The large jumps in the thorium feed rate are due to the bulk removal of absorbers, leading to spikes in the effective multiplication factor \cite{rykhlevskii_advanced_2018}. The uranium feed rate is smooth because it is based purely on the outflow of protactinium and does not fluctuate as rapidly as the thorium, which is the only external feed in the MSBR system. The average thorium-232 feed rate is 2.38 kg/day, while the average uranium-233 feed rate is 2.31 kg/day. These average values are used as the baseline for feed rates for the other batchwise method and for the continuous methods.

Another difference is that this version makes use of the h5py Python package to handle the hdf5 data files which contain depletion data. Within the hdf5 file, the atom density of each isotope before and after batchwise reprocessing is recorded as well as different reactor parameters such as the neutron multiplication factor.

\subsection{Steady}

SaltProc version 0.3 includes an example MSBR which users can run as soon as SaltProc is installed, though some of the run parameters such as neutrons per generation and run time have to be altered to generate a more valid model. This model also uses 3 day steps, but instead of performing the batch removal only at the cycle time value, this model removes a fraction every 3 days, which is called the "Steady" approach for batchwise reprocessing.

For example, a 30 day cycle time would result in 10\% removal every 3 days. This more accurately models online reprocessing, which is primarily what the MSBR employs in its reprocessing scheme aside from the salt discard.

The reactor refueling is performed slightly differently from v0.1 as well. This version assumes a constant ratio between the uranium refueling and the thorium refueling. The net refueling for each 3 day depletion step is set to maintain the mass of the system. Thus, the average values of 2.38 kg/day and 2.31 kg/day for thorium-232 and uranium-233, respectively, are used to determine the ratio of uranium to thorium which is implemented.

Additionally, this version of SaltProc uses the PyTables Python package for handling the hdf5 data files. The data files are structured similarly to v0.1 of SaltProc, but instead of atom density they use mass, which makes analysis of the files slightly more user friendly.

\subsection{Batch Approaches Summary}

\begin{table}[H]
\renewcommand{\arraystretch}{1.25}
\caption{Batchwise Reprocessing Methods}
\label{tab:batch_methods}
\begin{center}
\begin{tabular}{ | c | c | c | c | c | }
 \hline
        Approach & $\Delta t$ $[s]$ & $T_{cyc}$ $[s]$ & Fractional Removal Rate [s$^{-1}$] & Step Removal\\
 \hline
 \hline
        Bulk & 1 & 20 & - & \, 0/1$^{*}$\\
        Bulk & 10 & 20 & - & \, 0/1$^{*}$ \\
        Bulk & 40 & 20 & - & 1 \\
        Steady & 1 & 20 & 0.05 & 0.05\\
        Steady & 10 & 20 & 0.05 & 0.5\\
        Steady & 40 & 20 & 0.025 & 1\\
 \hline
\end{tabular}
\end{center}
\end{table}
        \begin{center}
\footnotesize{$^{*}$ Bulk removal is 0 until the depletion step is equal to the cycle time, at which point it removes 100\%.}\\
        \end{center}
        
Table \ref{tab:batch_methods} reflects the information shown in Figure \ref{fig:bulk_repr_cnst}. The table shows how the bulk method will only remove 100\% at the cycle time, and otherwise removes nothing. Additionally, the bulk method will remove 100\% as soon as possible if the depletion step size, $\Delta t$, is larger than the cycle time, $T_{cyc}$. 

The table also describes how the steady method fractional removal is constant up until the depletion step size becomes larger than the cycle time, at which point the removal at each step is 100\%, causing the steady method to then be equivalent to the bulk method. However, for shorter times, the steady method allows for a consistent fractional removal rate by adjusting the removal at each step.


\begin{figure}[H]
  \centering
  \includegraphics[scale=0.45]{images/bulk-compare-cycles.png}
  \caption{Plot showing how bulk and steady batchwise reprocessing removal works as a function of time for an example with a cycle time of 20 seconds.}
   \label{fig:bulk_repr_cnst}
\end{figure}

Figure \ref{fig:bulk_repr_cnst} shows an example fractional removal scheme for a fission product with a cycle time of 20 seconds. The Bulk method simply extracts 100\% of the product every 20 seconds, no matter how small the depletion step size is. The steady method removes some fraction every depletion step, which allows a semi-continuous process as the depletion step size becomes smaller and smaller. This can be seen in how a 10 second depletion step results in 50\% removal every 10 seconds, whereas a 1 second step allows for 5\% removal per depletion step.


% Add comparison of batch/steady for matrix of possible combinations. 

\begin{figure}[H]
  \centering
  \includegraphics[scale=0.45]{images/batch-steady-identical-compare.png}
  \caption{Plot showing how bulk and steady batchwise reprocessing are identical while the cycle time is shorter than or equal to the depletion step size.}
   \label{fig:bulk_repr_comp}
\end{figure}

A useful piece of information to note is that if the cycle time is shorter than or equal to the depletion step size, then the steady and bulk batchwise methods will be identical to each other, as demonstrated in Figure \ref{fig:bulk_repr_comp} which shows a simple model case with 1 second depletion steps and 1 second cycle times for an imaginary isotope. For the MSBR simulation, this means that the noble metals, gasses, and protactinium reprocessing will remain the same for either batchwise method employed. However, the longer cycle time groups such as the rare earths and volatile fluorides have cycle times which are 10-20 times longer than the depletion step size. Figure \ref{fig:bulk_repr_diff} shows a simulation of this for a situation where an isotope is generated at a rate of 1 gram per second, and is simulated with some combination of cycle time and depletion step size where the depletion step size is smaller than the cycle time.

\begin{figure}[H]
  \centering
  \includegraphics[scale=0.45]{images/batch-40-20.png}
  \includegraphics[scale=0.45]{images/batch-20-1.png}
  \caption{Plot showing how bulk and steady batchwise reprocessing are different while the cycle time is longer than the depletion step size for a 40 second cycle time and 20 second depletion step (top) and a 20 second cycle time with a 1 second depletion step (bottom).}
   \label{fig:bulk_repr_diff}
\end{figure}

One interesting aspect to note from this figure is that the steady method approaches a steady state value which is at the peak of the bulk method. This result can be confirmed by checking that the solutions for both the steady and bulk methods are valid. To check the solution, the 40 second cycle time and 20 second depletion step will be used. 

For the bulk method, after 100\% removal after the first cycle time, or 40 seconds, the mass then oscillates between 20 and 0 grams. This makes sense since 20 grams are generated over 20 seconds while every 40 seconds there is 100\% removal. 

For the steady method, the initial mass of 1 gram gains 20 grams over 20 seconds, yielding 21 grams. However, the steady removal then requires removal of 50\%, previously discussed and shown in Figure \ref{fig:bulk_repr_cnst}. This 50\% removal then drops the mass to 10.5 grams. 20 more grams are added to this value and 50\% is removed again, continuing iteratively. This can then be re-written as an infinite sum in order to determine the steady state solution. The first three iterations are shown in Equations \eqref{eq:ss-mass-p1} and \eqref{eq:ss-mass-p2}, where the value of 20 is the mass added during each depletion step and the value of 0.5 is the fractional removal performed each depletion step.

\begin{equation} \hfill 
m_{ss} = (((N_0+20)(0.5)+20)(0.5)+20)(0.5)
\hfill\label{eq:ss-mass-p1} \end{equation}

\begin{equation} \hfill 
m_{ss} = 0.5^3 N_0 + 0.5^3 (20) + 0.5^2 (20) + 0.5 (20)
\hfill\label{eq:ss-mass-p2} \end{equation}

Continuing this form infinitely yields Equation \eqref{eq:ss-mass-p3}, where the infinite sum yields 1, resulting in a net value of 20, the result of which is unaffected by the initial mass of the isotope and instead depends on the generation and reprocessing rates.

\begin{equation} \hfill 
m_{ss} = 20 \sum_{n=1}^\infty \frac{1}{2}^n 
\hfill\label{eq:ss-mass-p3} \end{equation}

% Add generic form (x added per second, y removal rate)

Overall, this shows that the steady and bulk batchwise methods have some similarities, but for elements with longer cycle times, the bulk method will experience oscillations whereas the steady method will level off smoothly.

\section{Continuous Reprocessing}

The continuous reprocessing functionality of Serpent2 was undocumented for several years aside from on the Serpent2 forums, but more recently work has gone into providing documentation. However, there has not been comparisons of Serpent2 continuous reprocessing with Serpent2 batchwise reprocessing with no differences between the models aside from the mathematical reprocessing approach used.

There are three different options which can be used for the Serpent2 built-in continuous reprocessing. The different options are defined in Serpent2 as 0, 1, and 2; which are henceforth referenced as Constant, Decay, and Step reprocessing. These names are selected because they closely describe the behaviour of each of the methods.

\subsection{Constant}
The Constant reprocessing method is the only method of the three options which does not conserve mass. This option instead generates mass the same way as the Decay reprocessing method for the initial step, then continues using that mass rate from that point onwards.

This method is useful for handling a constant fresh fuel feed rate, or any other situation where a constant addition of mass is desired. Because this method does not remove mass, it is not useful for any sort of constant removal.
%The main issue with this method is that it, for Serpent 2.32.1, it causes depletion to cease functioning while this method is active. This is an issue, since for most cases where a user wants to incorporate a feed rate, depletion is being considered.

\subsection{Decay}
The Decay reprocessing method is the most useful of the three options, and that is due to its flexibility in its usefulness. This method adds a decay term to whatever material it is attached to, and that decay term becomes a feed term to whatever material is set to receive the flow. For fission product removal, this is likely some material which is not modeled in the geometry of the problem but only exists to receive the fission product waste. However, this term can instead be leveraged to turn it into a feed rate.

This can be performed by attaching the decay term to a material which contains a volume of the desired feed, and then having that material feed into the core. The feed then "decays" from the feed tank into the core, essentially operating as a feed. Using this same method but altering the reprocessing constant or volume of the feed tank allows for an essentially constant feed rate, where the removed mass is negligible compared to the net mass.

Additionally, for this method, there are multiple approaches which can be used to calculate the reprocessing constants that should be implemented. These approaches are named "Cycle Time Decay", "Cycle Rate", and "SaltProc Cycle Rate" accordingly. The Cycle Time Decay approach treats the cycle time as if it is twice a half-life, and the cycle time is an exponential process. The Cycle Rate approach treats the cycle time as the time where 100\% of removal occurs, and then linearly extrapolates by assuming an equal percentage removal occurs up to that point.

The reason these approaches are implemented is because molten salt reactor reprocessing schemes provide reprocessing data in terms of cycle times. The cycle time is the amount of time it takes for a given element to be removed from the system. This cannot be directly solved using the Bateman equation, as shown in Equations \eqref{eq:decay_1} through \eqref{eq:decay_3} and \eqref{eq:decay_4} through \eqref{eq:decay_7}. 

\begin{equation} \hfill
\frac{dN}{dt} = -\lambda_r N
\hfill\label{eq:decay_1} \end{equation}

\begin{equation} \hfill
N(\tau) = 0 = N_0 e^{-\lambda_r \tau}
\hfill\label{eq:decay_2} \end{equation}

\begin{equation} \hfill
-ln(0) = \lambda_r \tau
\hfill\label{eq:decay_3} \end{equation}

Equations \eqref{eq:decay_1} through \eqref{eq:decay_3} show how the solution results in a reprocessing constant of infinity when only continuous removal is considered.

\begin{equation} \hfill
\frac{dN}{dt} = C -\lambda_r N
\hfill\label{eq:decay_4} \end{equation}

\begin{equation} \hfill
N(\tau) = 0 = N_0 e^{-\lambda_r \tau} + \frac{C}{\lambda_r} \left( 1 - e^{-\lambda_r \tau} \right)
\hfill\label{eq:decay_5} \end{equation}

\begin{equation} \hfill
0 = N_0 e^{-\lambda_r \tau} + \frac{C}{\lambda_r} - \frac{C}{\lambda_r} e^{-\lambda_r \tau}
\hfill\label{eq:decay_6} \end{equation}

\begin{equation} \hfill
0 = \lambda_r N_0 e^{-\lambda_r \tau} + C - C e^{-\lambda_r \tau}
\hfill\label{eq:decay_7} \end{equation}


 Equations \eqref{eq:decay_4} through \eqref{eq:decay_7} show that the only real solution to a steady accumulation and continuous removal is with a reprocessing constant of 0. 

\subsubsection{Cycle Time Decay}

The Cycle Time Decay approach is a simple method which makes use of the mathematical form of the Decay reprocessing method. Because it adds a "decay-like" term to the Bateman equation, this approach takes the cycle time for the target, cuts it in half, and then treats that value as the reprocessing half-life for that target. This process is shown in Equations \eqref{eq:ctd_1} and \eqref{eq:ctd_2}.

\begin{equation} \hfill
\tau_{1/2} = \frac{T_{cyc}}{2}
\hfill\label{eq:ctd_1} \end{equation}

\begin{equation} \hfill
\lambda_r = \frac{ln(2)}{\tau_{1/2}}
\hfill\label{eq:ctd_2} \end{equation}

%An example of how this approach is mathematically modeled can be seen in Figure \ref{fig:ctd_examp}.

%\begin{figure}[H]
%  \centering
%  \includegraphics[scale=0.6]{images/Cycle_Time_Decay_example_0.png}
%  \caption{A comparison between a linear 20 second cycle time and the Cycle Time Decay approach.}
%   \label{fig:ctd_examp}
%\end{figure}

%It can be seen in this figure that this approach is accurate in being close to the cycle time, though this accuracy is limited to above approximately 40\% of the atoms. Because fission products are continuously generated though, it is expected to be within the more accurate regions.
%Figure \ref{fig:ctd_examples} shows how the Cycle Time Decay approach fares when accumulation of fission products is incorporated into the problem. 

%\begin{figure}[H]
%  \centering
%  \includegraphics[scale=0.6]{images/Cycle_Time_Decay_example_10.png}
%  \includegraphics[scale=0.6]{images/Cycle_Time_Decay_example_5.png}
%  \includegraphics[scale=0.6]{images/Cycle_Time_Decay_example_2.5.png}
%  \caption{Comparison with linear 20 second cycle time and Cycle Time Decay approach for 10, 5, and 2.5 %atoms per second accumulation, respectively.}
%   \label{fig:ctd_examples}
%\end{figure}

%The top plot of Figure \ref{fig:ctd_examples} shows a situation where the accumulation of fission products occurs at a higher rate than the cycle time, leading to a larger steady-state value than initially present. The Cycle Time Decay does not reach the 200 value and instead is lower at 144. The middle plot shows a situation where the accumulation rate matches the cycle time removal rate, the same rate that the linear cycle time would have them removed, resulting in a steady-state value. However, the Cycle Time Decay approach drops to approximately 72 levels off. The bottom plot shows an accumulation which is half the linear cycle time removal rate. These results are somewhat close, though the Cycle Time Decay does not reach the same steady-state value, instead stabilizing around 40 instead of 50. For various accumulation rates ranging from 0.1 to 1 million atoms per second, the steady-state difference comes out to approximately 27.86\% for each. This percent difference also stays the same for different cycle times as well.

An example of implementing this approach for a 30 second cycle time would then have a 15 second reprocessing half-life. This is then converted to a reprocessing constant in the same way that a decay half-life is converted to a decay constant, which can be seen in Equation \eqref{eq:ctd_eqn}, resulting in a value of 0.0462 $s^{-1}$.

\begin{equation} \hfill
\lambda_{r} = \frac{ln(2)}{15} = 0.0462 s^{-1}
\hfill\label{eq:ctd_eqn} \end{equation}

\subsubsection{SaltProc Cycle Time Decay}

This approach is the same as Cycle Time Decay, but alters for any target which has a cycle time less than the batchwise reprocessing step incorporated by SaltProc. For example, a 3 day cycle time for some target would be treated the same as the standard Cycle Time Decay. However, a cycle time shorter than the 3 day batchwise reprocessing step used by SaltProc for the MSBR has its half-life extended to the SaltProc minimum value of 1.5 days. For example, a 30 second cycle time would instead be treated as a 3 day cycle time, which would result in a half-life of 1.5 days. Plugging in, this would result in a reprocessing constant of 0.462 $days^{-1}$, or 5.348E-6 $s^{-1}$. This is the reprocessing constant for any cycle time which is less than or equal to three days. 

\subsubsection{Cycle Rate}

The Cycle Rate approach uses a linear approximation such that the inverse of the cycle time is the rate at which material is removed. This is represented by, for example, 10\% removal per second would neglect the efficiency decrease over time and give 100\% removal after 10 seconds. This is calculated by investigating a unit time progression, i.e. 1 second or 1 day. Over this time period, the removal of atoms should be the fractional rate value, which means the final atom count at a time of 1 should be 1 - $X$, where $X$ is the fractional removal rate. This can be seen in Equations \eqref{eq:nat_log_0} and \eqref{eq:nat_log_3}.

\begin{equation} \hfill
X = \frac{1}{T_{cyc}}
\hfill\label{eq:nat_log_0} \end{equation}

\begin{equation} \hfill
\frac{dN}{dt} = -\lambda_r N
\hfill\label{eq:nat_log_1} \end{equation}

\begin{equation} \hfill
N(t) = N_0 e^{-\lambda_r t}
\hfill\label{eq:nat_log_1} \end{equation}

\begin{equation} \hfill
N(t = 1) = (1-X) N_{0}
\hfill\label{eq:nat_log_2} \end{equation}

\begin{equation} \hfill
(1-X) N_{0} = N_{0} e^{-\lambda_r (1)}
\hfill\label{eq:nat_log_2_1} \end{equation}

\begin{equation} \hfill
-ln(1-X)  = \lambda_r
\hfill\label{eq:nat_log_2_5} \end{equation}

\begin{equation} \hfill
\lambda_r = ln\left( \frac{1}{1-X} \right)
\hfill\label{eq:nat_log_3} \end{equation}

However, the mathematical model actually results in exponential decay in the removal effectiveness due to a decreasing amount of the target to remove. %Figure \ref{fig:cr_examp} shows how this approach compares to the actual exponential decay.

%\begin{figure}[H]
%  \centering
%  \includegraphics[scale=0.6]{images/Cycle_Rate_example_0.png}
%  \caption{A comparison between a linear 20 second cycle time and the Cycle Rate approach.}
%   \label{fig:cr_examp}
%\end{figure}

%It can be seen from this figure that the Cycle Rate approach removes at a rate sufficient to match the cycle time of 20 seconds while there are ~60\% of the atoms present. Once the atom count drops, the removal becomes less effective however, which is physical but makes the model less effective at matching the cycle time. Because the Cycle Rate approach is used on fission products which are being constantly generated, the difference is expected to be fairly small.

%\begin{figure}[H]
%  \centering
%  \includegraphics[scale=0.6]{images/Cycle_Rate_example_10.png}
%  \includegraphics[scale=0.6]{images/Cycle_Rate_example_5.png}
%  \includegraphics[scale=0.6]{images/Cycle_Rate_example_2.5.png}
%  \caption{Comparison with linear 20 second cycle time and Cycle Rate approach for 10, 5, and 2.5 atoms %per second accumulation, respectively.}
%   \label{fig:cr_examples}
%\end{figure}



An example cycle time of 30 seconds would be modeled by taking the inverse, which gives 0.033 $s^{-1}$. This value is then converted to a reprocessing constant by plugging it into the solved differential equation form shown in Equations \eqref{eq:cr_exmp1} and \eqref{eq:cr_eqn}, giving a value of 0.0339 $s^{-1}$.

\begin{equation} \hfill
X = \frac{1}{30} = 3.33E\minus2 s^{-1}
\hfill\label{eq:cr_exmp1} \end{equation}

\begin{equation} \hfill
\lambda_{r} = ln\left(\frac{1}{1 - 3.33E\minus2}\right) = 0.0339 s^{-1}
\hfill\label{eq:cr_eqn} \end{equation}

%The reason this approach is used is two-fold. Firstly, SaltProc v0.3 uses the same linear fractional removal approximation for its batchwise removal, so this same method is employed to better compare the two models. Secondly, due to the mathematical nature of the Decay reprocessing model, 100\% removal only occurs for a reprocessing constant of infinity. This means an approximation must be made. This is also a reasonable approximation since there is constant addition of material, meaning that the exponential line and the linear line are not going to be significantly different.


\subsubsection{SaltProc Cycle Rate}

The SaltProc Cycle Rate approach is the same as the Cycle Rate approach, but takes into account the limiting nature of the 3 day batchwise reprocessing step used by SaltProc. For example, a 6 day cycle time target would be modeled the same using the SaltProc Cycle Rate approach as the standard Cycle Rate approach. However, anything shorter than 3 days would be modeled differently, since that is the batchwise reprocessing step inocorporated by SaltProc for modeling the MSBR. For example, a 30 second cycle time would be analyzed instead as a 3 day cycle time, since SaltProc can only remove 100\% of material after a minimum of 3 days. This means the inverse would be 0.333 $days^{-1}$, or 3.858E-4 $s^{-1}$. Converting to a reprocessing constant gives a value of 3.858E-6 $s^{-1}$. This is the reprocessing constant for any cycle time which is less than or equal to three days.

\subsubsection{Direct Linear Approach}

Another approach is to directly apply the Cycle Rate removal rate, which is the inverse of the cycle time, as the reprocessing constant \cite{hombourger_eql0d_2020}. This is referred to as the Direct Linear approach. This approach is similar to the Cycle Rate approach, though the derivation for it is slightly different. This can be seen in Equations \eqref{eq:dir_lin_0} through \eqref{eq:dir_lin_10}, where Equations \eqref{eq:dir_lin_8} and \eqref{eq:dir_lin_9} uses L'Hôpital's rule.

\begin{equation} \hfill
X = \frac{1}{T_{cyc}}
\hfill\label{eq:dir_lin_0} \end{equation}

\begin{equation} \hfill
\frac{dN}{dt} = -\lambda_r N
\hfill\label{eq:dir_lin_1} \end{equation}

\begin{equation} \hfill
N(t) = N_0 e^{-\lambda_r t}
\hfill\label{eq:dir_lin_1_2} \end{equation}

\begin{equation} \hfill
N_{cur} = N_{prev} e^{-\lambda_r \Delta t}
\hfill\label{eq:dir_lin_2} \end{equation}

\begin{equation} \hfill
N_{cur} = (1 - \Delta t X) N_{prev}
\hfill\label{eq:dir_lin_3} \end{equation}

\begin{equation} \hfill
(1 - \Delta t X) N_{prev} = N_{prev} e^{-\lambda_r \Delta t}
\hfill\label{eq:dir_lin_4} \end{equation}

\begin{equation} \hfill
-ln\left((1 - \Delta t X)\right) = \lambda_r \Delta t
\hfill\label{eq:dir_lin_5} \end{equation}

\begin{equation} \hfill
\lambda_r = \frac{-ln\left((1 - \Delta t X)\right)}{\Delta t}
\hfill\label{eq:dir_lin_6} \end{equation}

\begin{equation} \hfill
\lambda_r = lim_{\Delta t \to 0} \frac{-ln\left((1 - \Delta t X)\right)}{\Delta t}
\hfill\label{eq:dir_lin_7} \end{equation}

\begin{equation} \hfill
\lambda_r = \frac{0}{0}
\hfill\label{eq:dir_lin_8} \end{equation}

\begin{equation} \hfill
\lambda_r = lim_{\Delta t \to 0} \frac{X}{1-X\Delta t}
\hfill\label{eq:dir_lin_9} \end{equation}

\begin{equation} \hfill
\lambda_r = X = \frac{1}{T_{cyc}}
\hfill\label{eq:dir_lin_10} \end{equation}

These equations follow a similar path to the Cycle Rate approach, but instead of using the linear approximation value after a unit time step, this approach generates the reprocessing constant while implementing the linear approximation as the time step goes to zero.

The differences between Cycle Rate and Direct Linear approaches can be seen in Table \ref{tab:repr_decay_view}, where for longer cycle times, the difference is negligible, but for shorter cycle times, the difference becomes larger. However, since extremely short cycle times are not realistically practical, the two approaches are approximately equivalent. Overall, the Direct Linear approach is more numerically stable, however, since it does not have asymptotic until reaching a cycle time of 0 seconds, whereas the Cycle Rate method has asymptotic behaviour at 1 second cycle time. This can be seen in Figure \ref{fig:dl_cr_asymptotic}. The asymptotic behaviour for a 0 second cycle time is not an issue because there are no negative cycle times, meaning the value is only approached from the positive side and behaves as physically expected.

\begin{table}[H]
\renewcommand{\arraystretch}{1.25}
\caption{Decay Reprocessing Approaches}
\label{tab:repr_decay_view}
\begin{center}
\begin{tabular}{ | c | c | c | c | c | }
 \hline
 Cycle Time & Removal Rate [$s^{-1}$] & CR $\lambda_{r}$ [$s^{-1}$] & DL $\lambda_{r}$ [$s^{-1}$] & $\Delta \lambda_{r}$\\
 \hline
 \hline
 3 d & 3.86E-6 & 3.86E-6 & 3.86E-6 & 7.44E-12\\
 20 s & 0.05 & 5.13E-2 & 5.00E-2 & 1.29E-3\\
 5 s & 0.2 & 2.23E-1 & 2.00E-1 & 2.31E-2\\
 2 s & 0.5 & 6.93E-1 & 5.00E-1 & 1.93E-1\\
 1 s & 1 & - & 1 & -\\
 \hline
\end{tabular}
\end{center}
\end{table}

\begin{figure}[H]
  \centering
  \includegraphics[scale=0.55]{images/dl_cr_asymptote.png}
  \caption{A comparison of the Direct Linear and Cycle Rate reprocessing constants for different cycle times.}
   \label{fig:dl_cr_asymptotic}
\end{figure}

Because the shortest cycle time for the MSBR is 20 seconds, this shows that the Cycle Rate and Directly Linear approaches are roughly equivalent for this reprocessing scheme.

%Figure \ref{fig:dl_cr_compare_linear} shows the implementation of both methods for an example nuclide with a cycle time of CYCLE TIME and an accumulation rate of ACCUMULATION RATE. It can be seen from this figure how the Direct Linear and Cycle Rate approaches differ in modeling of the same isotope.

%\begin{figure}[H]
%  \centering
%  \includegraphics[scale=0.25]{images/direct_linear_compare.png}
%  \caption{A comparison of the Direct Linear and Cycle Rate approaches for a particular element with a %given accumulation rate and cycle time(s).}
%   \label{fig:dl_cr_compare_linear}
%\end{figure}


\subsubsection{Decay Approaches Summary}

Figure \ref{fig:repr_cnst} shows how the reprocessing constants for the different approaches vary as a function of cycle time.

\begin{figure}[H]
  \centering
  \includegraphics[scale=0.45]{images/cont-compare-cycles.png}
  \caption{Plot of how reprocessing constants for different approaches vary with cycle time.}
   \label{fig:repr_cnst}
\end{figure}

Figure \ref{fig:repr_cnst} shows the three different continuous decay based methods for reprocessing as a function of various cycle times. From the figure, it can be seen that the Cycle Rate method does not have results for cycle times less than or equal to 1 second, which is a significant weakness of the method since physical processes could exist with a cycle time in that range. The other two methods provide results down to a 0 second cycle time, which allows full coverage of possible cycle time values.

The Direct Linear method behaves similarly to the Cycle Time Decay method during small cycle times and the Cycle Rate method for longer cycle times. This is due to the general form of the equations for each, where at small cycle times the inverse cycle time relationship with the Cycle Time Decay method dominates, while at larger cycle times the Cycle Rate log of the inverse term matches more closely.

%WHICH IS THE MORE ACCURATE APPROACH AMONG THE THREE (There isn't a way to validate this without more information on how cycle time is generated. Perhaps if instead we find data for molten salt reprocessing and compare against that it might be useful?) (Maybe check average net atoms for steady and bulk batch and compare, then maybe check a comparison of net atoms at 1/2 cycle time and cycle time, etc.)

%CITE "Liquid Fuel Molten Salt Reactors for Thorium Utilization" This paper uses bulk batchwise reprocessing

%INCLUDE PLOT OF N(t) FOR DIFFERENT CYCLE TIMES WHERE ONLY INITIAL MASS AND REPROCESSING ARE CONSIDERED. (DIRECT LINEAR AND CYCLE RATE ARE THE SAME FOR LARGE CYCLE TIMES)

%REPLICATE ATOMS OVER TIME PLOTS FOR DIFFERENT MODELS AND DIFFERENT ACCUMULATION RATES TO SEE HOW STEADY-STATE VALUES COMPARE.

\subsection{Step}

The Step reprocessing method implemented in Serpent2 is mathematically very similar to the Decay model, but instead of updating continuously in time, it instead is updated during new depletion steps. In this manner, it is a sort of mix between the Constant method and the Decay method. This is because over a single depletion step, it is a constant added or subtracted from the Bateman equation, while over many depletion steps, it follows the same exponential decay form of the Decay method.

This particular method is not useful for extracting fission products, and is not needed for constant feed rates since that can be modeled using the Decay method. This method could be useful for inducing a step drop in feed rate, though this would require running only a single depletion step until the drop occurred. Additionally, this drop could be simulated by running the Decay method and reducing the reprocessing constant. This would allow for flexibility in the distribution of depletion steps as well without having to worry about changing the behaviour of the reprocessing functionality. 

Another potential issue with the Step method is that the depletion step can last long enough that the constant mass removal causes the mass to go negative. However, Serpent2 will cease running if this occurs. The Decay method does not mathematically allow for negative mass, the Constant method does not conserve mass, and the Step method stops running once there is negative mass.

The main potential use of the Step reprocessing may be for movement of material at a set rate. This could be implemented by writing a script to check the current number of atoms of the target and adjusting the reprocessing rate accordingly. However, for the MSBR, this form of reprocessing is not necessary, and is thus not implemented.

\section{Mass Balancing}

One of the useful features of batchwise reprocessing is that the net mass of the core can be balanced by adjusting the feed rates to provide the same amount into the core that is removed through reprocessing. Alternative methods also exist, such as removing excess mass or increasing the volume \cite{ridley_method_2017}. SaltProc version 0.1 does not account for mass balancing but maintains a constant thorium mass, whereas SaltProc version 0.3 has the feed rates set to maintain mass. Mass balancing is particularly important in Serpent2 due to the way masses in Serpent2 are handled.

In Serpent2, an increase in mass does not affect volume, but instead increases the density of that isotope in the material accordingly. This affects macroscopic cross section calculations, and can lead to variation in results if not accounted for, since in reality volumetric expansion could be assumed. Though there are several methods to handle mass balancing with a batch method, it not currently continuously possible in Serpent2.

In order to balance the mass in Serpent2 continuously, one approach could be to iteratively perform depletion calculations while updating feed rates until a balanced mass is found while also minimizing some other parameter, such as net mass of material added. However, the current reprocessing options available in Serpent2 only allows for pseudo-constant and decaying feed rates over the depletion step. With those two feed types available, it is not possible to have a constant mass balance.

It is possible to have the masses at the end of each depletion step remain constant, but this does handle mass fluctuations during depletion. It does, however, solve the issue of the density variation causing a difference in cross sections.

Overall, if the net mass difference is sufficiently small, it does not have to be considered since the results would not be significantly impacted. To check this for the MSBR, a simple back-of-the-envelope calculation can be used. To determine the maximum possible increase in density, the mass loss due to fission and reprocessing is neglected, and only mass addition from the thorium feed rate is considered. The uranium feed is not added because it is equivalent to the protactinium removal, so it would have a negligible impact on net mass. 

\begin{equation} \hfill
\Delta m = \Dot{m}_{feed} \Delta t \approx (2.5)(6000) = 15,000 kg
\hfill\label{eq:m_gain} \end{equation}

It can be seen in Equation \eqref{eq:m_gain} that the net mass gain for an average feed rate of 2.5 kilograms per day over 6000 days is 15,000 kg \cite{rykhlevskii_fuel_2020, betzler_molten_2017}. This does seem to be a very large value, but the importance of the mass in the depletion calculation is primarily in how it affects cross sections, which means that the impact on the overall thorium density is important. The thorium density is 1.46 grams per cubic centimeter, and the net volume is approximately 48.71 cubic meters. The density with the added mass is calculated using Equation \eqref{eq:new_rho}.

\begin{equation} \hfill
\rho_f = \frac{m_0 + \Delta m}{V} = \frac{(4.871E7)(1.45919E\minus3) + (15,000)}{4.871E7} 
\hfill\label{eq:new_rho} \end{equation}

The final density comes out to be 1.767 grams per cubic centimeter, which is a percent difference of ~21\%. Although this is an upper bound on the mass difference, a 21\% difference in the expected cross section would result in significant error in the results. Therefore, since it is possible for the mass balancing to have an impact, it should be investigated to ensure the mass balancing is not impacting the results in any unexpected manner.

%DIFFERENT FEED RATE VARIANTS IMPLEMENTED

%MSBR MASS BALANCE BATCH AND CONTINUOUS


%\section{Uncertainty}

%SHANNON ENTROPY

%STOCHASTIC ERROR VS ACTUAL DIFFERENCE IN RESULTS

%DEPLETION ERROR BUILDUP

%\section{Continuous Reprocessing Implementation}

\section{Effects of Delayed Neutrons on Depletion}

Delayed neutrons have a softer energy spectrum and drift along with the movement of the fuel salt in a fluid fueled molten salt reactor. This means that the delayed neutron precursors, of which some fraction leave the core and some move to less neutron important regions, have the potential to alter the depletion results of the MSBR model.

However, it has been shown by Zhou et al that for the MSBR, the delayed neutron precursor drift has a negligible impact on depletion results \cite{zhou_fuel_2018}. Although Zhou et al has shown this, it is still worth considering the maximum possible effect delayed neutron precursors could have on depletion results. In order to determine this, the Serpent2 functionality of disabling delayed neutrons is employed \cite{leppanen_serpent_2015}. Using this method, two different models are generated of the MSBR. The first is one in which the delayed neutron precursors are evenly distributed within the core, which is the model implemented by SaltProc and this work \cite{rykhlevskii_modeling_2019}. The second model is one in which delayed neutrons no longer exist. This is beyond the greatest effects the delayed neutron precursor drift could have on depletion, so the absolute maximum effect of delayed neutron precursors on depletion results can be determined.

Because the delayed neutron precursors are fission products or come from the decay of fission products, depletion must be performed. For reprocessing, the average feed rate from SaltProc is used with continuous Direct Linear reprocessing, while Direct Linear reprocessing is used for fission product removal. 

The impact on $k_{eff}$ was investigated, and it was determined that the delayed neutrons from fission products add roughly 20 pcm to $k_{eff}$ after 6,000 days of operation, which is roughly steady state operation. The masses of various isotopes, such as uranium-235 and thorium-232, were also compared, though no statistically significant difference was found. Therefore, the net impact of delayed neutrons from fission product precursors is negligible on depletion results, which agrees with the results from Zhou et al.

%Figure \ref{fig:keff_compare} shows how the multiplication factors evolve over time. The final difference between the two values is FINALDIFF.

%\begin{figure}[H]
%  \centering
%  \includegraphics[scale=0.45]{images/keff_dnp_compare.png}
%  \caption{Comparison of $k_{eff}$ over time with and without delayed neutrons.}
%   \label{fig:keff_compare}
%\end{figure}

%PLOT OF SPECTRUMS WITH AND WITHOUT DELAYED NEUTRONS

%TABLE OF IMPORTANT ISOTOPE CONCENTRATIONS FOR WITH AND WITHOUT





% Toggle remove this for debugging
%\nocite{*}


\section{Serpent MSBR Model}

The results generated in this work come from the MSBR, particularly the model developed by Rykhlevskii \cite{rykhlevskii_advanced_2018}. The geometry of the model has remained unchanged, while the reprocessing has been overhauled using the continuous reprocessing functionality in Serpent2 which was previously undocumented. Previously, SaltProc's batchwise reprocessing was used, but the continuous reprocessing in Serpent2 is incorporated in this work.

\subsection{Reprocessing Structure}

There are two different reprocessing schemes developed in this work. Both schemes follow the MSBR reprocessing scheme, and vary how refueling functions. The first method matches the method of Rykhlevskii, where the uranium-233 feed is treated as equivalent to the protactinium-233 removal rate. In this case, the average uranium-233 feed rate from Rykhlevskii is used over the entire simulation time \cite{rykhlevskii_advanced_2018}. A simplified overview of this scheme can be seen in Figure \ref{fig:spmatchrepr}.


\begin{figure}[H]
  \centering
  \includegraphics[scale=0.45]{images/sp-match-repr-scheme.png}
  \caption{Simplified reprocessing scheme based on mimicking SaltProc.}
   \label{fig:spmatchrepr}
\end{figure}

The second reprocessing scheme is physically based, and instead creates a protactinium decay tank. From this tank, any generated uranium is continuously removed and sent back to the core, which is the intended design scheme for the MSBR \cite{robertson_conceptual_1971}. A simplified version of the phyiscally realistic uranium feed can be seen in Figure \ref{fig:spmatchrepr}.

\begin{figure}[H]
  \centering
  \includegraphics[scale=0.45]{images/phys-repr-scheme.png}
  \caption{Simplified reprocessing scheme based on the phyiscal MSBR processes.}
   \label{fig:spmatchrepr}
\end{figure}

Overall, it can be anticipated that the two methods will be approximately equal at steady-state, since at that point the assumption by Rykhlevskii that the uranium input is the same as the protactinium output should be valid. However, there is expected to be a fairly large difference for the first month or two of operation, as the half life of protactinium-233 is on the order of one month. This means that the uranium feed will take some time before it is providing a steady source of fissile material to the core.



\chapter{Results}
\renewcommand*\descriptionlabel[1]{\hspace\leftmargin$#1$}
\setcounter{tocdepth}{5}
\setcounter{secnumdepth}{5}



\section{Depletion Time Step Refinement}
\label{s:dep-mesh}

As described in Section \ref{s:models-methods}, there are many different methods available for modeling the online reprocessing of molten salt reactors.
Previously, Rykhlevskii has generated results for the molten salt breeder reactor using the bulk batchwise method \cite{rykhlevskii_advanced_2018}.
This method is primarily useful due to the straightforward nature of its implementation.
However, it is less accurate and more costly than other methods, as discussed in the Methodology section.
This is the reason that Rykhlevskii continued to develop SaltProc from bulk batchwise reprocessing in version 0.1 into steady batchwise reprocessing in version 0.3.
The steady batchwise method has approximately the same computational cost, but provides more accuracy by applying reprocessing during each depletion step available.
This provides a far better approximation of continuous reprocessing.

The continuous methods are all similar in terms of computational cost.
The main difference is how cycle time data is interpreted.
Because the cycle times given are the amount of time to remove 100\% of a target, and the continuous methods all use exponential terms, different strategies are used.
The most straightforward method is direct linear, seen in Section \ref{s:DL}, while the methods of cycle rate and cycle time decay, Sections \ref{s:CR} and \ref{s:CTD}, respectively, are based on assumptions.

In the Methodology section, I discussed the theoretical basis of these methods.
In this chapter, I will demonstrate the practical application of each in order to compare them.
I will first perform a depletion time step refinement study in order to determine the effects of depletion step on the effective multiplication factor and nuclide compositions.
Intuitively, it is expected that the batchwise methods will perform worse than the continuous methods as the step size grows larger.
This is due to the less frequent application of reprocessing on a process which is intended to be occurring continuously.
The less frequent depletion steps also means the cross section and flux data is not updated as the fuel is depleted, leading to less accurate results.
%However, because this impacts both continuous and batchwise reprocessing, it will not be
%so is of less concern in this analysis.

To illustrate the impact of less frequent application of reprocessing, consider a simple thermal-spectrum system where uranium is fissioning and reprocessing is taking place.
If xenon is physically continuously reprocessed, then the total neutron absorption in the core by xenon is reduced.
However, using a longer batchwise step size results in xenon forming and capturing neutrons after the depletion simulation forms the xenon, the transport simulation updates the cross sections for the xenon, and another depletion step is run where the xenon has a capture cross section.
Only after the depletion step size time elapses is batchwise reprocessing performed and the xenon is removed.
By this point, many neutrons have already been captured, affecting reactor performance.

I will show two types of results to demonstrate the effect of altering the depletion step.
The first result I will show is the effect on the composition of some select isotopes, as altering the depletion step directly affects the concentration of isotopes affected by reprocessing.
The other result I will show is the effective multiplication factor, which is directly affected by the change in the composition.
Through these results I show the direct and indirect impacts of changing the depletion step size.

\subsection{Batchwise Reprocessing}

\subsubsection{Bulk Batchwise}

%A depletion step mesh refinement study using the bulk batchwise reprocessing version of SaltProc was performed by Rykhlevskii comparing 3, 6, 12, and 24 day depletion steps over 25 years for the Transatomic Power Molten Salt Reactor \cite{rykhlevskii_fuel_2020}.
%Because the bulk batchwise method has already been analyzed and discussed, results for that method will not be separately generated here.
Inclusion of a depletion step refinement study of bulk batchwise is not performed, as it can only strictly perform worse than steady batchwise methods.
The reason for this was discussed in Section \ref{s:batch-generic}.
However, the refueling rates from the MSBR work performed by Rykhlevskii using the bulk batchwise method are used here, and the bulk batchwise method itself is implemented in Section \ref{s:bulk} to compare with continuous methods \cite{rykhlevskii_advanced_2018}.
%, as well as the bulk batchwise model of the MSBR by Rykhlevskii will be used as an item of comparison to the continuous and steady batchwise methods implemented in this work \cite{rykhlevskii_advanced_2018}.

\subsubsection{Steady Batchwise}
\label{s:steady-batch-mesh}

A depletion step refinement study using the steady batchwise reprocessing version of SaltProc was performed by Rykhlevskii comparing 3, 6, 12, and 24 day depletion steps over 25 years for the Transatomic Power Molten Salt Reactor \cite{rykhlevskii_fuel_2020}. Rykhlevskii used a three day depletion step in the bulk batchwise analysis of the MSBR over 20 years \cite{rykhlevskii_advanced_2018}. Rykhlevskii and Powers et al. both suggest the three day depletion time step size to correlate with the removal interval of protactinium-233 \cite{rykhlevskii_fuel_2020, powers_new_2013}. In this work, the depletion time step of three days will be used for the steady batchwise method.

%Similar to the work by Rykhlevskii, this work instead checks 1, 1.5, 3, and 6 day depletion steps.
%These shorter time steps are used to check the differences between compositions and in the effective multiplication factor.
%As the time step becomes shorter, the method will become both more costly and more accurate.
%Therefore, finding a balance of cost and accuracy for the computational resources available is of particular interest.


%This is to check if a smaller step size than 3 days should be implemented, or if the accuracy gained is not worth the increased computational cost.

%As smaller depletion step sizes are imposed, more resources are required for processing and memory.
%Considering the computational cost increases associated with using a smaller depletion step size, this depletion step size selection was well made.
%For the batchwise results in this work, the three day depletion step will also be used, as it strikes a balance between cost and accuracy that works well for the resources available.
%However, I plan to compare to longer and shorter depletion times to this three day depletion

%In this work the krypton and xenon removal is altered from 100\% per three days to around 91\% per three days.
%the cycle time was previously implemented as 100\% removal, the krypton and xenon removal is instead around 91\%.
%This is because of the different liquid phase mass transfer coefficient, and was discussed by Rykhlevskii previously \cite{rykhlevskii_fuel_2020}.
%For the other elements, the cycle times used are the same as from Robertson \cite{robertson_conceptual_1971}.
%One difference this has is that instead of having a mass of zero in the xenon-135 and xenon-136 figures, there will be plot showing how the xenon concentrations vary over time.
%Additionally, the protactinium removal is 9.5\% instead of 95\% every three days, which results in a shift in the average feed rates which were determined in the bulk batchwise SaltProc solution.

\mycomment{

\paragraph*{Effects of Reprocessing}

Figures \ref{fig:steady-batch-k} through \ref{fig:steady-batch-xe136} show how the different depletion step sizes impact the effective multiplication factor and concentration of different isotopes in the system after reprocessing has occurred.
It can be seen by the evolution of the effective multiplication factor that different steady state values are being approached. The difference after 24 days between the 1 day depletion step and 6 day depletion step is roughly 656 pcm, 490 pcm for the 1 and 3 day steps, and 260 pcm for the 1 and 1.5 day steps. Based on the trend shown, it is expected for these values to continue diverging over time. The errors in this Figure \ref{fig:steady-batch-k} are at most 16 pcm.


\begin{figure}[H]
  \centering
  \includegraphics[scale=0.7]{images/keff-batch-dep-refine.png}
  \caption{k$_{eff}$ over time using various depletion step sizes with steady batchwise reprocessing.}
   \label{fig:steady-batch-k}
\end{figure}

A similar trend of approaching a different steady state value can be seen from the thorium-232 and uranium-233 masses over time. After 30 days, the thorium-232 maximum difference is roughly 22 kilograms between the 1 day and 6 day depletion steps. The uranium-233 difference is roughly 13 kg, which is a slightly smaller difference than the thorium-232. The reason the longer time step has a reduced thorium-232 and uranium-233 mass is because they are both treated as feed rates which maintain a constant net mass for the system. Because the reprocessing is less frequent, there are more parasitic absorbers built up in the core, reducing the total amount of material which is burned. Therefore, there is a smaller feed rate provided to the system.

\begin{figure}[H]
  \centering
  \includegraphics[scale=0.7]{images/Th232_sp_comp.png}
  \caption{Thorium-232 mass over time using various depletion step sizes with steady batchwise reprocessing.}
   \label{fig:steady-batch-th}
\end{figure}

\begin{figure}[H]
  \centering
  \includegraphics[scale=0.7]{images/U233_sp_comp.png}
  \caption{Uranium-233 mass over time using various depletion step sizes with steady batchwise reprocessing.}
   \label{fig:steady-batch-u}
\end{figure}

Because xenon-135 has a very high thermal neutron absorption cross section, it is useful to see how its mass varies throughout depletion in these cases.
In Figure \ref{fig:steady-batch-u}, it can be seen that it is at a steady state value which is the same for each of the different depletion step sizes.
Previously, it was shown that the relative error should scale based on the depletion step size, yet these values are the same. The reason for this is because the cycle time of xenon in the system is 20 seconds, while the smallest depletion step size is significantly larger than that. This results in a situation where the mass of xenon-135 builds up, stabilizes, and is reprocessed. In order to view the actual steady state of xenon-135 for the cycle time of 20 seconds, a depletion step size would have to be on the order of seconds instead of days.

\begin{figure}[H]
  \centering
  \includegraphics[scale=0.7]{images/Xe135_sp_comp.png}
  \caption{Xenon-135 mass over time using various depletion step sizes with steady batchwise reprocessing.}
   \label{fig:steady-batch-xe135}
\end{figure}

After xenon-135 absorbs a neutron, it can form xenon-136.
By monitoring the concentration of xenon-136 in the system, the amount of xenon-135 which has undergone parasitic absorption in the core can be measured indirectly.
Figure \ref{fig:steady-batch-xe136} shows how this concentration varies over time.
Note here that the xenon-136 concentrations are strongly affected by depletion step size.
This means that there are more parasitic absorptions by xenon-135, directly impacting the reactor physics behaviour.

The equilibrium concentration of xenon-136, which directly correlates to the amount of parasitic absorption by xenon-135, directly correlates to the depletion step size.
The ratio of depletion step sizes is somewhat close to the ratio of the equilibrium concentrations of xenon-136.
The ratio of the xenon-136 concentrations at 1.5, 3, and 6 day depletion steps to the concentration at the 1 day depletion step are 1.53, 3.1, and 6.3, respectively.
This linear growth of the expected equilibrium concentration of xenon-136 does not hold as the step sizes become very small.
For very small steps, the xenon-136 will match the continuous concentration, becoming non-linear.

\begin{figure}[H]
  \centering
  \includegraphics[scale=0.7]{images/Xe136_sp_comp.png}
  \caption{Xenon-136 mass over time using various depletion step sizes with steady batchwise reprocessing.}
   \label{fig:steady-batch-xe136}
\end{figure}

\paragraph*{Feed and Removal Rates}

Figures \ref{fig:steady-batch-th-repr} and \ref{fig:steady-batch-u-repr} show the feed rates of thorium-232 and uranium-233, respectively. As previously discussed, the overall feed rate has the objective of maintaining constant mass. This explains why the feed rate for longer depletion steps is smaller than the feed rates associated with shorter depletion steps, as the reprocessing is less frequent, leaving more parasitic absorbers in the core.

\begin{figure}[H]
  \centering
  \includegraphics[scale=0.7]{images/feed_Th232_6d_sp_comp.png}
  \caption{Thorium-232 feed rate over time using various depletion step sizes with steady batchwise reprocessing.}
   \label{fig:steady-batch-th-repr}
\end{figure}

\begin{figure}[H]
  \centering
  \includegraphics[scale=0.7]{images/feed_U233_6d_sp_comp.png}
  \caption{Uranium-233 feed rate over time using various depletion step sizes with steady batchwise reprocessing.}
   \label{fig:steady-batch-u-repr}
\end{figure}

Figures \ref{fig:steady-batch-xe135-repr} and \ref{fig:steady-batch-xe136-repr} show the removal rates of xenon-135 and xenon-136, respectively. The xenon-136 removal rate is fairly straightforward to understand, as it was previously shown that the steady state concentration of xenon-136 is higher for the longer depletion step sizes. This means that the removal rate would also need to be higher, since the amount removed is proportional to the amount present in the system. For the xenon-135, the removal rate relationship may not immediately be clear since the steady state mass of xenon-135 after reprocessing is the same for the different depletion step sizes. However, the removal rate is higher for the shorter depletion step size because the xenon-135 builds up to a steady state value before reprocessing occurs. This means that every time reprocessing occurs, roughly the same amount of xenon-135 is removed. Thus, more frequent reprocessing yields an increased removal rate.

\begin{figure}[H]
  \centering
  \includegraphics[scale=0.7]{images/waste_Xe135_6d_sp_comp.png}
  \caption{Xenon-135 removal rate over time using various depletion step sizes with steady batchwise reprocessing.}
   \label{fig:steady-batch-xe135-repr}
\end{figure}

\begin{figure}[H]
  \centering
  \includegraphics[scale=0.7]{images/waste_Xe136_6d_sp_comp.png}
  \caption{Xenon-136 removal rate over time using various depletion step sizes with steady batchwise reprocessing.}
   \label{fig:steady-batch-xe136-repr}
\end{figure}

}

\subsection{Continuous Reprocessing}
\label{s:cont-meshing}

%For the batchwise methods, depletion time steps beyond 6 days are not considered. This is because a larger time step causes more error, and it can be seen that the difference between the 6 day depletion step and the smaller depletion steps is non-negligible.

For the continuous methods, the error buildup from using larger depletion steps is due to not updating various simulation data rather than from lack of updating the material compositions based on reprocessing. Due to this, larger time steps can be considered. In order to determine the impact of the updated simulation data, initially chosen depletion time steps of 1, 3, 6, 15, and 30 days are implemented. Additionally, although different continuous methods have been introduced, the depletion time step refinement only needs to consider one of the methods. This is because the continuous methods used all have the same general form and only alter the specific reprocessing constants implemented. Thus, the direct linear method is selected to determine the optimal depletion step size for the continuous methods. Additionally, it has been used in several other works and is very similar to the cycle rate method \cite{xia_development_2019, nuttin_potential_2005, zhou_fuel_2018}.

\subsubsection{Submonthly Depletion Steps}

The results from using depletion steps of 1, 3, 6, 15, and 30 days can be seen in Figures \ref{fig:DL-cont-k} through \ref{fig:DL-cont-xe136}. The results regarding the effective multiplication factor show that, after 30 days, all of the effective multiplication factors are within stochastic error of each other.

\begin{figure}[H]
  \centering
  \includegraphics[scale=0.7]{images/DL_NSTEP_keff.png}
  \caption{k$_{eff}$ over time using various depletion step sizes with direct linear continuous reprocessing.}
   \label{fig:DL-cont-k}
\end{figure}

For the thorium-232 and uranium-233 masses, the values align within fractions of a percent of each other after 30 days of depletion. Additionally, it can be seen that there are sudden jumps in the thorium-232 mass. This is because of rounding which occurs due to a limit to the precision of the data which is available from Serpent2.

\begin{figure}[H]
  \centering
  \includegraphics[scale=0.7]{images/DL_NSTEP_Th-232_mass.png}
  \caption{Thorium-232 mass over time using various depletion step sizes with direct linear continuous reprocessing.}
   \label{fig:DL-cont-th}
\end{figure}

\begin{figure}[H]
  \centering
  \includegraphics[scale=0.7]{images/DL_NSTEP_U-233_mass.png}
  \caption{Uranium-233 mass over time using various depletion step sizes with direct linear continuous reprocessing.}
   \label{fig:DL-cont-u}
\end{figure}

The xenon-135 steady state mass is the same for each of the different step sizes, which is expected. Additionally, the xenon-136 steady state mass is also the same for the different depletion step sizes, showing that the capture rate of the xenon-135 does not vary significantly even when adjusting the depletion step size from 1 day to 30 days.

\begin{figure}[H]
  \centering
  \includegraphics[scale=0.7]{images/DL_NSTEP_Xe-135_mass.png}
  \caption{Xenon-135 mass over time using various depletion step sizes with direct linear continuous reprocessing.}
   \label{fig:DL-cont-xe135}
\end{figure}

\begin{figure}[H]
  \centering
  \includegraphics[scale=0.7]{images/DL_NSTEP_Xe-136_mass.png}
  \caption{Xenon-136 mass over time using various depletion step sizes with direct linear continuous reprocessing.}
   \label{fig:DL-cont-xe136}
\end{figure}

Because these results are all very close to each other, it is prudent to test for larger depletion step sizes in order to determine how much computational cost reduction can be afforded while maintaining high accuracy in the results. Although using a 30 day depletion step is already an order of magnitude larger step size than that used in the batchwise method in this work, further coarsening of the time step would allow for further improvements to the computational cost reduction.

\subsubsection{Larger Depletion Steps}

The next set of depletion step sizes selected are 30, 60, 120, and 360 days. The smallest step size, 30 days, is selected in order to properly compare the larger depletion step sizes with a step size which has been shown to give results which are very close to those of a 1 day depletion step size. The results can be seen in Figures \ref{fig:DL-cont-k-2} through \ref{fig:DL-cont-xe136-2}.

\begin{figure}[H]
  \centering
  \includegraphics[scale=0.7]{images/DL_NSTEP_keff-large.png}
  \caption{k$_{eff}$ over time using various depletion step sizes with direct linear continuous reprocessing.}
   \label{fig:DL-cont-k-2}
\end{figure}

The 60 day depletion time effective multiplication factor at 360 days is not within stochastic error of the value for the 30 day depletion time. The error bounds are only 5 pcm apart, but since the computational cost decrease is only a factor of two, the 30 day depletion step size is sufficiently large for the cases implemented in this work while maintaining a high level of precision and are used as the default when not specified.
The largest uncertainty in the effective multiplication factor here is 17 pcm.

The thorium-232 and uranium-233 masses follow a similar trend where the larger depletion step size has a higher mass of both isotopes. The difference in thorium can be seen by visualizing how the larger step sizes operate similarly to a tangent line from the shorter depletion step sizes. This is because the cross section and spectral data is not updated, and so the same rate of decrease is used over the entire depletion step. Because the system starts at beginning of cycle, the thorium is not yet in steady state and more is burned than added, so the decrease is expected.

\begin{figure}[H]
  \centering
  \includegraphics[scale=0.7]{images/DL_NSTEP_Th-232_mass-large.png}
  \caption{Thorium-232 mass over time using various depletion step sizes with direct linear continuous reprocessing.}
   \label{fig:DL-cont-th-2}
\end{figure}

For the uranium-233, a similar trend can be seen, though it is less pronounced. The main difference with thorium-232 is the fact that the thorium mass changes due to feed and neutron absorption, whereas the uranium-233 has feed, neutron absorption, and accumulation from the decay chain of the thorium-232 absorption. With this in mind, the difference of the uranium-233 and thorium-232 in Figures \ref{fig:DL-cont-th-2} and \ref{fig:DL-cont-u-2} becomes more clear. For the shorter depletion steps, a similar trend as in the thorium-232 occurs where the rate is matches for the depletion step. However, the thorium-232 is also a factor, and since the thorium-232 has a greater mass for larger depletion step sizes, this results in more uranium-233 being bred as well.

\begin{figure}[H]
  \centering
  \includegraphics[scale=0.7]{images/DL_NSTEP_U-233_mass-large.png}
  \caption{Uranium-233 mass over time using various depletion step sizes with direct linear continuous reprocessing.}
   \label{fig:DL-cont-u-2}
\end{figure}

The xenon-135 and xenon-136 masses match closely, with differences on the order of milligrams. This agrees well with the previous results generated as well. The reason the results agree well is because the xenon-135 concentration is expected to approach a constant result with increasing flux. This is because the xenon-135 comes from the decay of iodine-135, which scales proportionally to the flux. However, the xenon-135 has a large capture cross section, and is altered into xenon-136 proportionally to the flux as well. This shows why there is not a large difference even though there is a difference of 324 pcm between the effective multiplication factors of the 360 and 30 day depletion step sizes.

\begin{figure}[H]
  \centering
  \includegraphics[scale=0.7]{images/DL_NSTEP_Xe-135_mass-large.png}
  \caption{Xenon-135 mass over time using various depletion step sizes with direct linear continuous reprocessing.}
   \label{fig:DL-cont-xe135-2}
\end{figure}

The xenon-136 is also captured with a rate proportional to the flux, though the capture cross section is significantly lower than that of xenon-135 for the thermal spectrum used in the MSBR. This can be seen in the mass of xenon-136, which has a higher steady state concentration that that of xenon-135. The main limiting factor on both of these xenon isotopes is the reprocessing term, which keeps the steady state mass of both much smaller than they would be without the reprocessing.

\begin{figure}[H]
  \centering
  \includegraphics[scale=0.7]{images/DL_NSTEP_Xe-136_mass-large.png}
  \caption{Xenon-136 mass over time using various depletion step sizes with direct linear continuous reprocessing.}
   \label{fig:DL-cont-xe136-2}
\end{figure}


\subsection{Variable Depletion Step Size}

Previously, the depletion step size needed for a reasonably converged result was investigated using a constant depletion step size. However, it is understood that the reactor will experience some rapid initial changes, after which the changes slow and eventually the reactor reaches steady state. Therefore, it is reasonable to expect that in the early times, small depletion step sizes should be used, while at later times, larger depletion step sizes should be used.

Because it was previously shown for continuous reprocessing that the depletion step sizes smaller than 30 days give results within stochastic error for the effective multiplication factor, the 30 day depletion step size will be the smallest considered step size. Several cases are considered over a net depletion time of 360 days in order to determine the effects of varying the depletion step size at various points in the simulation.

The depletion step sizes used for each case are given in Table \ref{tab:step-sizes}, and the effective multiplication factors at 360 days are given in Table \ref{tab:var-dep-step-size} for each case.

\begin{table}[H]
    \centering
    \caption{Test Cases' Variable Step Sizes}
    \begin{tabular}{c|c|c|c|c|c}
    \hline
    Case & 30 Days & 90 Days & 180 Days & 330 Days & 360 Days \\
    \hline
    A & 12 & 0 & 0 & 0 & 0 \\
    \hline
    B & 6 & 0 & 1 & 0 & 0 \\
    \hline
    C & 3 & 1 & 1 & 0 & 0 \\
    \hline
    D & 1 & 0 & 0 & 1 & 0 \\
    \hline
    E & 6 & 0 & 1 & 0 & 0 \\
    \hline
    F & 0 & 0 & 0 & 0 & 1 \\
    \hline
    \end{tabular}
    \label{tab:step-sizes}
\end{table}

By default, the cases use the shortest time step before the longer steps. However, Case E varies from this, allowing for it to be directly comparable to case B. Case E uses the 180 day step first, followed by the six 30 day steps.

%Case A uses 12 30 day steps; Case B uses 6 30 day steps and 1 180 day step; Case C uses 3 30 day steps, 1 90 day step, and 1 180 day step; Case D uses 1 30 day step and 1 330 day step; Case E uses 1 180 day step and 6 30 day steps; and Case F uses 1 360 day step. The results for each case at the time value of 360 days are given in Table \ref{tab:var-dep-step-size}.


\begin{table}[H]
\renewcommand{\arraystretch}{1.25}
\caption{Variable Depletion Step Size Results}
\label{tab:var-dep-step-size}
\begin{center}
\begin{tabular}{ c | c | c | c }
 \hline
 Case & $k_{eff}$ & $\Delta k_{eff}$ [pcm]  & Simulations\\
 \hline
 \hline
 A & 1.06801 $\pm$ 0.00017  & - & 12\\
 B & 1.06844 $\pm$ 0.00015 & 43 $\pm$ 32  & 7\\
 C & 1.06857 $\pm$ 0.00016  & 56 $\pm$ 33  & 5\\
 D & 1.07034 $\pm$ 0.00016 & 233 $\pm$ 33  & 2\\
 E & 1.06846 $\pm$ 0.00016 & 45 $\pm$ 33  & 7\\
 F & 1.07071 $\pm$ 0.00017 & 270 $\pm$ 34 & 1\\
 \hline
\end{tabular}
\end{center}
\end{table}

%\begin{table}[H]
%\renewcommand{\arraystretch}{1.25}
%\caption{Varyiable Depletion Step Size Results}
%\label{tab:var-dep-step-size}
%\begin{center}
%\begin{tabular}{ | c | c | c | c | c | c | }
% \hline
% Case & $k_{eff}$ & $\Delta k_{eff}$ [pcm] & $Th_{232}$ [Mg] & $U_{233}$ [Mg] & Simulations\\
% \hline
% \hline
% A & 1.06801 $\pm$ 0.00017  & - & 71.05 & 1.44 & 12\\
% B & 1.06844 $\pm$ 0.00015 & 43 $\pm$ 32 & 71.05 & 1.44 & 7\\
% C & 1.06857 $\pm$ 0.00016  & 56 $\pm$ 33 & 71.05 & 1.44 & 5\\
% D & 1.07034 $\pm$ 0.00016 & 233 $\pm$ 33 & 71.06 & 1.45 & 2\\
% E & 1.06846 $\pm$ 0.00016 & 45 $\pm$ 33 & 71.05 & 1.44 & 7\\
% F & 1.07071 $\pm$ 0.00017 & 270 $\pm$ 34 & 71.06 & 1.45 & 1\\
% \hline
%\end{tabular}
%\end{center}
%\end{table}

%From the thorium-232 and uranium-233 mass results, it can be seen Cases D and F are outliers. The trait in common with these two runs is the largest depletion step size implemented, as both of these runs have larger depletion step sizes than any of the other runs, resulting in far fewer simulations performed.

For the result of the effective multiplication factor, the difference from Case A was taken for the rest of the cases as case A is assumed to be the most realistic.
These results show that Cases B, C, and E are all very close to Case A. Cases D and E are the outliers, which is also reflected in the number of simulations run. Case D uses a short depletion time step initially followed by a single long step, which should have allowed for the simulation data to update based on the presence of fission products which were not initially present.
However, this impact is likely offset by the continuous reprocessing, resulting in more impact purely due to the general changes in the system. It seems that multiple steps are needed during the first 360 days of operation in order to properly account for the changes within the reactor, such as Case E, which leads with a depletion step size of 180 days but then proceeds to follow it with six 30 day depletion steps.

The variable depletion step size could be considered to have an impact if the reactor were at steady-state, but after only 360 days of operation, there are still changes occurring \cite{rykhlevskii_advanced_2018}. This is the reason why having more simulations yielded better results than the cases which had fewer simulations.

\subsection{Computational Cost Analysis}

%In order to provide a useful measure of computational cost, Table \ref{tab:comp-cost-methods-1} is generated.
Computational cost of each method is presented in Table \ref{tab:comp-cost-methods-1}.
This table represents the compuational cost of forming the depletion step mesh refinement study in the units of "number of simulations required," as this is the largest impact on run time. This unit must include the depletion matrix solve, but it may also include the transport solve time, data loads, and other overhead costs. To convert to units of time, the average time per simulation can be calculated and then multiplied by these values. In this case, the time cost per simulation will be represented using $\tau$, which is larger for batchwise, $\tau_{b}$, than continuous, $\tau_{c}$.
The depletion step size used by the batchwise methods is 3 days, while the continuous method uses 30 day depletion steps.

%The cost for SaltProc based batchwise methods can be calculated by first accounting for all depletion solves.
For unoptimized batchwise, the number of simulations is roughly double what is actually required, as the pre-reprocessed simulation is also run. An optimized batchwise method could remove the double running, which would significantly reduce computational cost. Additionally, as can be seen in the formula for calculating the cost, reduction can be achieved by maximizing the depletion step size or reducing the net depletion time, or $T_{net}$.

\begin{table}[H]
\renewcommand{\arraystretch}{1.25}
\caption{Computational Cost Using Constant Depletion Steps}
\label{tab:comp-cost-methods-1}
\begin{center}
\begin{tabular}{ c | c | c | c | c }
 \hline
 Method & Cost Formula & 30 days & 1 year & 10 years\\
 \hline
 \hline
 Unoptimized Batchwise & $2 T_{net} \Delta t^{-1} \tau_{b} + \tau_{b}$ & 21 $\tau_{b}$ & 245 $\tau_{b}$ & 2436 $\tau_{b}$ \\
 Optimized Batchwise & $T_{net} \Delta t^{-1} \tau_{b} + \tau_{b}$ & 11 $\tau_{b}$ & 123 $\tau_{b}$ & 1219 $\tau_{b}$\\
 Continuous (All) & $T_{net} \Delta t^{-1} \tau_{c} + \tau_{c}$ & 2 $\tau_{c}$ & 14 $\tau_{c}$ & 123 $\tau_{c}$\\
 \hline
\end{tabular}
\end{center}
\end{table}

Because continuous reprocessing enables the use of depletion time steps on the order of 10x larger while retaining a similar level of precision, the computational cost is significantly lower than the batchwise computational cost.

\section{Advanced Protactinium Decay Model}
\label{s:advance-decay}

The original method used by Rykhlevskii to model the decay of protactinium-233 into uranium-233 was to feed a mass of uranium-233 into the core equivalent to the mass of protactinium-233 pumped out every 3 days \cite{rykhlevskii_advanced_2018}. With the implementation of continuous reprocessing methods, the breeding can be more accurately modeled. In the new method, protactinium-233 is continuously removed and allowed to decay. Once it decays, the uranium-233 is then continuously added back into the core. These two methods can be directly compared. I will refer to them here as "SaltProc Average" for the average feed into the MSBR calculated using SaltProc and "Decay Tank" for the model that continuously removes protactinium and adds the uranium generated via decay back into the core.

It is expected for the largest impact to be at beginning of cycle, while at steady state this improved method will likely have only a minor effect on the results. The results of this analysis can be seen in Figures \ref{fig:DL-cont-k-adv} through \ref{fig:DL-cont-xe136-adv}, where the continuous method implemented for both was the direct linear method, and 305 day depletion steps were used over a period of 3660 days, or roughly ten years. Although this depletion step size is larger than those previously used, it is expected to still give reasonable accurate results. Additionally, the discard and salt discard processes are not considered here, as only the differences between the different feed rate methods are being investigated.

The effective multiplication factor for the realistic uranium-233 feed approach can be seen to drop rapidly, and then begin leveling off after around 3 years.
The reason for this difference in behaviour is because the feed which uses the SaltProc averaged value makes a steady state assumption, which allows for a full supply of uranium-233 available for feeding the reactor at all times.
In can be seen that the slopes of the two different methods are beginning to converge slowly after a few year, and it is expected that they would eventually match.
However, if the batch discard processes are included, a true steady state equilibrium would not be reached, resulting in at least some minor differences between the methods.
The largest stochastic error in the effective neutron multiplication factor is 19 pcm, and so is too small to appear in Figure \ref{fig:DL-cont-k-adv}.

\begin{figure}[H]
  \centering
  \includegraphics[scale=0.7]{images/adv-keff.png}
  \caption{k$_{eff}$ over time using various depletion step sizes with direct linear continuous reprocessing comparing the decay tank and SaltProc averaged feed rates.}
   \label{fig:DL-cont-k-adv}
\end{figure}

The thorium-232 masses have a fairly constant difference from each other over the entire run. The reason for this is likely due to the fact that there is less uranium-233 being added to the system with the realistic approach, meaning more interactions with thorium-232 occur instead. This results in a lower effective multiplication factor, at least initially, which provides increased uranium-233 production, and lower thorium-232 mass.

\begin{figure}[H]
  \centering
  \includegraphics[scale=0.7]{images/adv-Th232.png}
  \caption{Thorium-232 mass over time using various depletion step sizes with direct linear continuous reprocessing comparing the decay tank and SaltProc averaged feed rates.}
   \label{fig:DL-cont-th-adv}
\end{figure}

The uranium-233 for the realistic method follows a trend which is expected, which is a sharp initial drop followed by a decrease in the rate of change. After the sharp drop, the protactinium decay tank fills, and fresh uranium-233 is supplied to the system in an increasing quantity. For the averaged method, because the uranium-233 was burned at a higher rate, this leads to a decrease in the total amount. It is expected that there will be a few oscillations in the uranium-233 mass for the averaged system until it finally achieves a steady state value, though this is only true if the system only has continuous reprocessing processes. This oscillation is expected because initially, there is more uranium-233 which fissions, rapidly decreasing the quantity while maintaining a higher effective multiplication factor. This causes the uranium-233 to burn quickly, dropping the net mass until it reaches a point that the effective multiplication factor decreases while the uranium-233 mass is also lowered, reducing probability of fission as well as the number of neutrons. This allows the feed rate to build up more uranium-233 in the system, repeating the process.

\begin{figure}[H]
  \centering
  \includegraphics[scale=0.7]{images/adv-U233.png}
  \caption{Uranium-233 mass over time using various depletion step sizes with direct linear continuous reprocessing comparing the decay tank and SaltProc averaged feed rates.}
   \label{fig:DL-cont-u-adv}
\end{figure}

For the xenon-135 and xenon-136, the masses are slightly smaller than the averaged method initially, which is due to the decreased uranium-233, resulting in fewer fission products being generated. However, after more time passes, the values match. This is due to the steady state value of xenon-135 not significantly changing when the flux is already large.

\begin{figure}[H]
  \centering
  \includegraphics[scale=0.7]{images/adv-Xe135.png}
  \caption{Xenon-135 mass over time using various depletion step sizes with direct linear continuous reprocessing comparing the decay tank and SaltProc averaged feed rates.}
   \label{fig:DL-cont-xe135-adv}
\end{figure}

\begin{figure}[H]
  \centering
  \includegraphics[scale=0.7]{images/adv-Xe136.png}
  \caption{Xenon-136 mass over time using various depletion step sizes with direct linear continuous reprocessing comparing the decay tank and SaltProc averaged feed rates.}
   \label{fig:DL-cont-xe136-adv}
\end{figure}

\section{Comparisons of Methods}

There are many different works which have implemented batchwise reprocessing computational methods to simulate a continuous, online reprocessing scheme, which are presented in Section \ref{litrev-msr-batchwise}.
It follows that the accuracy of each method should be resolved.
From previous work by Rykhlevskii, as well as through the depletion step mesh refinement provided in this work, it has been shown that increasing the depletion step size causes batchwise methods to become less effective at simulating the physical, continuous process of online reprocessing \cite{rykhlevskii_fuel_2020}.
This can be seen in Section \ref{s:batch-cont-compare} where the theoretical impact is discussed and Section \ref{s:steady-batch-mesh} where the impacts are shown in the MSBR.

\subsection{Continuous Methods}

Sections \ref{s:dep-mesh} and \ref{s:advance-decay} have exclusively used the direct linear continuous reprocessing method due to similarities between the continuous methods.
However, it is useful to determine how the different continuous methods compare when applied to an MSBR model.
For this comparison, the steady SaltProc-based averaged feed rates and modified cycle times are used, shown in Section \ref{s:steady}.
A 30 day depletion step size is also used, the reason for which is provided in Section \ref{s:cont-meshing}.


The effective multiplication factor, over the course of 360 days, has a difference on the order of 500pcm between the cycle time decay, CTD, method and the direct linear and cycle rate, DL and CR, methods respectively.
These methods are described in detail in Sections \ref{s:CTD}, \ref{s:DL}, and \ref{s:CR}, respectively.
The direct linear and cycle rate methods are very close to each other, which is expected since the reprocessing constants are also very similar. The control method, CTRL, shows the effects of no reprocessing and no feed.
The maximum stochastic uncertainty in these values is 17 pcm.

\begin{figure}[H]
  \centering
  \includegraphics[scale=0.7]{images/cont-compare-keff-1.png}
  \caption{k$_{eff}$ over time comparing different continuous reprocessing methods.}
   \label{fig:continuous-compare}
\end{figure}

The reason the continuous methods shown here continue decreasing and are eventually overtaken by the control reprocessing is due to the protactinium reprocessing. Because the protactinium in these methods is continuously removed and the uranium-233 feed is averaged only over the first 90 days of the steady batchwise method, it is expected that the continuous model will become inaccurate some time after 90 days. This accuracy can be improved by implementing the advanced protactinium decay model which removes the dependency on an average uranium-233 feed rate.

However, the main purpose of this figure is to demonstrate how the different continuous methods compare to each other over a long period of time. Additionally, this figure shows how the effective multiplication factor without reprocessing drops below critical after roughly 300 days, and that having either a good estimate for the uranium-233 feed rate or a continuous method to generate the uranium-233 feed rate is very important for depletion modeling of the MSBR.


\subsection{Continuous and Batchwise Methods}

The continuous method selected for the comparison is the direct linear reprocessing method. For the batchwise methods, the bulk and steady batchwise reprocessing schemes from SaltProc early and current versions are used. For the bulk batchwise reprocessing, the cycle times from the Robertson report are followed \cite{robertson_conceptual_1971}. Additionally, the uranium-233 and thorium-232 feed rates in the continuous model use the values from the bulk batchwise results \cite{rykhlevskii_advanced_2018}.

For the steady batchwise comparison, the modified cycle times are used which include an efficiency term on the xenon, krypton, and protactinium reprocessing constants. For the continuous comparison, the average thorium-232 and uranium-233 feed rates are generated from the 90 day runs and are implemented.

For all of these comparisons, the direct linear reprocessing method uses the average feed rate of the given batchwise method and the cycle times from the Robertson report \cite{robertson_conceptual_1971}. The SaltProc direct linear reprocessing method, which is intended to be a continuous approximation of the batchwise reprocessing method, will match the reprocessing constants used in the given batchwise method.

If the results are sufficiently close, then it may be reasonable for a continous method to be implemented in place of a batchwise method when the reprocessing scheme calls for a physically batchwise processes. This would allow for reduced computational cost and potentially large depletion step sizes to be implemented. If the results are not close, then these results will provide information on how the difference in reprocessing methods can affect results.


\subsubsection{Continuous and Bulk Batchwise Methods}
\label{s:bulk}

In this comparison, there are three different methods considered. These methods are the continuous direct linear, DL, method, the continuous SaltProc direct linear, SPDL, method; and the bulk batchwise SaltProc method. Figure \ref{fig:bulk-comapre-keff} shows how the effective multiplication factor varies over time for these methods, with stochastic error less than 20 pcm.
It can be seen that there is a jump at around 50 days for the bulk SaltProc data, which is caused by removal of the rare earths, as shown in the MSBR reprocessing scheme \cite{robertson_conceptual_1971}. However, it is clear to see here that the continuous methods are not able to match the bulk batchwise reprocessing results very closely at all, even using the SaltProc direct linear method which attempts to replicate the same reprocessing constants.

\begin{figure}[H]
  \centering
  \includegraphics[scale=0.7]{images/soln-3-keff.png}
  \caption{k$_{eff}$ over time comparing bulk batchwise results to continuous methods.}
   \label{fig:bulk-comapre-keff}
\end{figure}

Figures \ref{fig:bulk-comapre-th232} and \ref{fig:bulk-comapre-u233} show the mass of thorium-232 and uranium-233 over time, respectively. Overall, the masses of both isotopes are within a few kilograms, and do not have a large impact on the differences between the methods.
%For example, the spike in thorium-232 mass at 50 days when the rare earths are removed does not have nearly as much impact as the removal of the rare earths does on reactor behaviour.

\begin{figure}[H]
  \centering
  \includegraphics[scale=0.7]{images/soln-3-Th232.png}
  \caption{Thorium-232 mass over time comparing bulk batchwise results to continuous methods.}
   \label{fig:bulk-comapre-th232}
\end{figure}

\begin{figure}[H]
  \centering
  \includegraphics[scale=0.7]{images/soln-3-U233.png}
  \caption{Uranium-233 mass over time comparing bulk batchwise results to continuous methods.}
   \label{fig:bulk-comapre-u233}
\end{figure}

Figures \ref{fig:bulk-comapre-xe135} and \ref{fig:bulk-comapre-xe136} show how the xenon-135 and xenon-136 masses, respectively, change over time for the different methods. However, since this version of SaltProc removes 100\% each depletion time step, resulting in a flat line of zero grams, the mass before reprocessing is shown instead. This gives a better idea of the amount of xenon which is in the system most of the time. From these figures, it can be seen that the SPDL continuous method does a fairly good job of matching the bulk batchwise reprocessing method's masses. However, the SPDL method allows for slightly more fission product poisons to exist, such as xenon-135, which allows for more parasitic absorption, into isotopes such as xenon-136. This is the reason that the SPDL effective multiplication factor is lower than that of bulk SaltProc.

\begin{figure}[H]
  \centering
  \includegraphics[scale=0.7]{images/soln-3-Xe135.png}
  \caption{Xenon-135 mass over time comparing bulk batchwise results to continuous methods.}
   \label{fig:bulk-comapre-xe135}
\end{figure}

\begin{figure}[H]
  \centering
  \includegraphics[scale=0.7]{images/soln-3-Xe136.png}
  \caption{Xenon-136 mass over time comparing bulk batchwise results to continuous methods.}
   \label{fig:bulk-comapre-xe136}
\end{figure}

From these results, it can be concluded that using continuous reprocessing to perform a bulk reprocessing scheme will not yield same results. This also means that attempting to simulate a continuous reprocessing scheme using a bulk batchwise method will not yield results that are the same as a continuous reprocessing method.

\subsubsection{Continuous and Steady Batchwise Methods}

The next set of comparisons are between the steady batchwise results from SaltProc and two continuous methods, direct linear and SaltProc direct linear. The effective multiplication factor over time for each of the three methods can be seen in Figure \ref{fig:steady-compare-keff}, and it has stochastic uncertainty less than 20 pcm.
It can be seen that the direct linear continuous $k_{eff}$ is closer to the steady batchwise $k_{eff}$ than the SaltProc direct linear result. Among figures \ref{fig:steady-compare-th232} through \ref{fig:steady-compare-xe136}, the plot which demonstrates the reason behind this is the xenon-135 mass over time.

\begin{figure}[H]
  \centering
  \includegraphics[scale=0.7]{images/soln-1-2-keff.png}
  \caption{k$_{eff}$ over time comparing steady batchwise results to continuous methods.}
   \label{fig:steady-compare-keff}
\end{figure}

\begin{figure}[H]
  \centering
  \includegraphics[scale=0.7]{images/soln-1-2-th232.png}
  \caption{Thorium-232 mass over time comparing steady batchwise results to continuous methods.}
   \label{fig:steady-compare-th232}
\end{figure}

The uranium-233 mass over time for the different methods reflects very closely with the effective multiplication factors. The uranium-233 comes from beta decay of protactinium-233, where protactinium is one of the elements designated to be removed in the MSBR reprocessing scheme. For direct linear, the removal of protactinium is approximated using 100\% removal over three days. For SaltProc direct linear, the removal is instead approximated to 9.5\% over 3 days, which is the value used in steady SaltProc. However, because the SPDL method is continuously removing the protactinium, this means that there is less protactinium-233 which decays into uranium-233 when compared to the steady batchwise method in SaltProc. Due to the short net depletion time, the uranium-233 feed rate plays a much larger role in dominating the uranium-233 concentration.

\begin{figure}[H]
  \centering
  \includegraphics[scale=0.7]{images/soln-1-2-u233.png}
  \caption{Uranium-233 mass over time comparing steady batchwise results to continuous methods.}
   \label{fig:steady-compare-u233}
\end{figure}

A similar effect with the protactinium can be seen in Figures \ref{fig:steady-compare-xe135} and \ref{fig:steady-compare-xe136}. Although the SPDL method is using a reprocessing constant which should allow for matching SaltProc, the continuous removal results in fairly large discrepancies. The xenon-135 difference is only on the order of tenths of a gram, but because the quantity is constantly suppressed in the continuous SPDL method, the overall capture is reduced drastically, resulting in a difference of roughly 40 grams in the xenon-136 mass.

\begin{figure}[H]
  \centering
  \includegraphics[scale=0.7]{images/soln-1-2-xe135.png}
  \caption{Xenon-135 mass over time comparing steady batchwise results to continuous methods.}
   \label{fig:steady-compare-xe135}
\end{figure}

\begin{figure}[H]
  \centering
  \includegraphics[scale=0.7]{images/soln-1-2-xe136.png}
  \caption{Xenon-136 mass over time comparing steady batchwise results to continuous methods.}
   \label{fig:steady-compare-xe136}
\end{figure}

Overall, using a continuous method to approximate a steady batchwise method does not seem generally viable. This is because isotopes which are continuously removed have a very different behaviour from isotopes which are removed after interactions occur over an entire depletion step. If the reprocessing scheme calls for a batchwise process, it may be important to reactor behaviour to allow for those capture or decay processes to occur without influence from reprocessing.

\subsection{Overall Differences}

Figures \ref{fig:prev-cur-keff-plot} through  \ref{fig:prev-cur-xe136-plot} show how the steady batchwise reprocessing method with the average uranium-233 feed and the direct linear continuous method with the realistic protactinium decay compare with each other.
The batchwise method uses the long term bulk average uranium-233 feed rate, and implements altered cycle rates for protactinium, xenon, and krypton. The continuous method uses only the cycle time data from the Robertson report to generate the reprocessing constants \cite{robertson_conceptual_1971}.

The effective multiplication factor for the two different approaches, shown in Figure \ref{fig:prev-cur-keff-plot}, initially has the steady batchwise effective neutron multiplication factor falling at a greater rate.
This is due to the initial buildup of fission product poisons, while the continuous reprocessing allows for these poisons to be removed more efficiently.
However, as time goes on, the lacking uranium-233 feed due to the empty protactinium decay tank causes the current work to dip below the previous work.
Over the course of three months, the longer lived behaviour cannot be seen.
However, Section \ref{s:advance-decay} covers the advanced protactinium decay model over a longer time period.
That section shows the results over the course of several years and reveals that the reactivity difference due to the different refueling methods eventually decreases.

\begin{figure}[H]
  \centering
  \includegraphics[scale=0.7]{images/prev-cur-keff.png}
  \caption{k$_{eff}$ over time with three day depletion steps for steady batchwise and direct linear continuous reprocessing using different uranium-233 feed schemes.}
   \label{fig:prev-cur-keff-plot}
\end{figure}

The thorium-232 feed rate for both approaches is the same, yet the steady batchwise method burns it slightly faster than the continuous method. This effect is likely caused by the presence of more uranium-233 in the previous work, which allows for a faster burn rate of the thorium-232 due to a higher net flux. The total amount of thorium-232 over time can be seen in Figure \ref{fig:prev-cur-th232-plot}.

\begin{figure}[H]
  \centering
  \includegraphics[scale=0.7]{images/prev-cur-th232.png}
  \caption{\ce{^{232}Th} over time with three day depletion steps for steady batchwise and direct linear continuous reprocessing using different uranium-233 feed schemes.}
   \label{fig:prev-cur-th232-plot}
\end{figure}

This difference in the uranium-233 masses, shown in Figure \ref{fig:prev-cur-u233-plot}, is primarily due to the effects of the different uranium-233 feed rate approaches implemented in each of the different works.

\begin{figure}[H]
  \centering
  \includegraphics[scale=0.7]{images/prev-cur-u233.png}
  \caption{\ce{^{233}U} over time with three day depletion steps for steady batchwise and direct linear continuous reprocessing using different uranium-233 feed schemes.}
   \label{fig:prev-cur-u233-plot}
\end{figure}

For the xenon-135 and xenon-136 masses, shown in Figures \ref{fig:prev-cur-xe135-plot} and \ref{fig:prev-cur-xe136-plot}, respectively, these results are primarily impacted due to the continuous and reprocessing methods implemented.

\begin{figure}[H]
  \centering
  \includegraphics[scale=0.7]{images/prev-cur-xe135.png}
  \caption{\ce{^{135}Xe} over time three day depletion steps for steady batchwise and direct linear continuous reprocessing using different uranium-233 feed schemes.}
   \label{fig:prev-cur-xe135-plot}
\end{figure}

\begin{figure}[H]
  \centering
  \includegraphics[scale=0.7]{images/prev-cur-xe136.png}
  \caption{\ce{^{136}Xe} over time three day depletion steps for steady batchwise and direct linear continuous reprocessing using different uranium-233 feed schemes.}
   \label{fig:prev-cur-xe136-plot}
\end{figure}

Overall, these results show the combined effects of implementing continuous reprocessing methods with a continuously updated uranium-233 refeed tank which is initially empty. This is compared with batchwise reprocessing methods and an averaged uranium-233 feed rate.

%\subsection{One Year Depletion}

%The results of a one year net depletion time can be seen in Figures \ref{fig:optimal-compare-sspdldl-2}  through \ref{fig:SPCR-cont-136Xe-2}, which reflect the results shown in Figures \ref{fig:optimal-compare-sspdldl}  through \ref{fig:SPCR-cont-136Xe}.

%\begin{figure}[H]
%  \centering
%  \includegraphics[scale=0.1]{images/placeholder.png}
%  \caption{k$_{eff}$ over time with small depletion steps for steady batchwise, SaltProc direct linear continuous, and direct linear continuous reprocessing.}
%   \label{fig:optimal-compare-sspdldl-2}
%\end{figure}

%\begin{figure}[H]
%  \centering
%  \includegraphics[scale=0.1]{images/placeholder.png}
%  \caption{\ce{^{235}U} over time with small depletion steps for steady batchwise, SaltProc direct linear continuous, and direct linear continuous reprocessing.}
%   \label{fig:SPCR-cont-235u-2}
%\end{figure}

%\begin{figure}[H]
%  \centering
%  \includegraphics[scale=0.1]{images/placeholder.png}
%  \caption{\ce{^{135}Xe} over time with small depletion steps for steady batchwise, SaltProc direct linear continuous, and direct linear continuous reprocessing.}
%   \label{fig:SPCR-cont-135Xe-2}
%\end{figure}

%\begin{figure}[H]
%  \centering
  %\includegraphics[scale=0.1]{images/placeholder.png}
%  \caption{\ce{^{136}Xe} over time with small depletion steps for steady batchwise, SaltProc direct linear continuous, and direct linear continuous reprocessing.}
%   \label{fig:SPCR-cont-136Xe-2}
%\end{figure}

%\subsection{Ten Year Depletion}

%CONTINUOUS AND BULK SALTPROC PREVIOUS RESULTS \hline

%The results of a one year net depletion time can be seen in Figures \ref{fig:optimal-compare-sspdldl-2}  through \ref{fig:SPCR-cont-136Xe-2}, which reflect the results shown in Figures \ref{fig:optimal-compare-sspdldl}  through \ref{fig:SPCR-cont-136Xe}.

%\begin{figure}[H]
%  \centering
%  \includegraphics[scale=0.1]{images/placeholder.png}
%  \caption{k$_{eff}$ over time with small depletion steps for steady batchwise, SaltProc direct linear continuous, and direct linear continuous reprocessing.}
%   \label{fig:optimal-compare-sspdldl-2}
%\end{figure}

%\begin{figure}[H]
%  \centering
%  \includegraphics[scale=0.1]{images/placeholder.png}
%  \caption{\ce{^{235}U} over time with small depletion steps for steady batchwise, SaltProc direct linear continuous, and direct linear continuous reprocessing.}
%   \label{fig:SPCR-cont-235u-2}
%\end{figure}

%\begin{figure}[H]
%  \centering
%  \includegraphics[scale=0.1]{images/placeholder.png}
%  \caption{\ce{^{135}Xe} over time with small depletion steps for steady batchwise, SaltProc direct linear continuous, and direct linear continuous reprocessing.}
%   \label{fig:SPCR-cont-135Xe-2}
%\end{figure}

%\begin{figure}[H]
%  \centering
%  \includegraphics[scale=0.1]{images/placeholder.png}
%  \caption{\ce{^{136}Xe} over time with small depletion steps for steady batchwise, SaltProc direct linear continuous, and direct linear continuous reprocessing.}
%   \label{fig:SPCR-cont-136Xe-2}
%\end{figure}


\section{Mass Balancing}

The continuous reprocessing method allows for larger depletion steps to be used without losing out on the impact of reprocessing while also allowing for fewer simulations to be run when compared to batchwise methods. However, one potential issue with this method is mass balancing.

The effects of mass balancing are of particular note in the Serpent2 code because an increase in mass causes a corresponding increase in density due to constant volumes. A simple workaround to this is to either remove some fraction of mass to return the net mass to equilibrium or adjust the volumes so the density remains constant, thus maintaining physicality.

However, if the issue of mass balancing is not considered, then the results could be fairly inaccurate.
This is because changes in isotopic composition and densities will occur, where material is added to a fixed volume space. This has the potential to raise the density more than is physically possible, thus generating non-physical results.

\subsection{Impact Minimization Using Modified Feeds}

One aspect investigated is minimizing the change in the net mass by only altering the average feed rates implemented. It is determined through repeated simulations that increasing the thorium-232 feed rate drives the second derivative of the net mass negative. The uranium-233 feed increasing drives the second derivative to be positive. The reason for these effects are due to the change in the effective multiplication factor, mass of the added feed, and removal of fission products by reprocessing.
Using these relationships, a process to determine the feed rates to minimize the change in the net mass is determined. The process is as follows:  hold one feed rate constant while adjusting the other;
if the final net mass is higher than initial net mass, then reduce the feed rates and vice versa;
once the final net mass is same as initial net mass, make adjustment based on second derivative;
if the second derivative is positive, increase thorium-232 or decrease uranium-233 feed;
if the second derivative is negative, increase uranium-233 or decrease thorium-232 feed;
iterate until second derivative is zero and net mass remains roughly constant.

After iterating through this processes several times using a depletion time step of 300 days over 10 steps, the modified average feed rates of 1.9 kg/day of thorium-232 and 2.16 kg/day of uranium-233 were determined to result in a very small change to the net mass on the order of 0.02\%.

\subsection{Impact of Mass Balancing in the MSBR}

Previously in Section \ref{s:mass-bal-meth}, I calculated the maximum impact of improper mass balancing and and showed there could be on the order of a 21\% difference impact on the density of thorium-232 when only considering the feed rate and neglecting loss terms. Realistically, that percent difference in the density should be far above the maximum in this case, since loss terms are very important in determining the amount of thorium-232 present in the system. However, it is expected for the thorium-232 mass to change over time, so only viewing the mass of thorium is not very useful. Instead, viewing the net mass of the system will allow for insight into how the overall system behaviour could be expected to change due to density variations.

\begin{figure}[H]
  \centering
  \includegraphics[scale=0.7]{images/net-mass-pcnt-change.png}
  \caption{Change in net mass of the fuel salt for various methods.}
   \label{fig:net-mass-bal}
\end{figure}

Figure \ref{fig:net-mass-bal} shows the evolution of the net mass of the fuel salt over time while using a 30 day depletion time step over 360 days. The "Control" data points show how the net mass changes without any reprocessing feeds or removals.
The method which has the largest change in the net mass is the direct linear method with the averaged uranium-233 feed rate. The net mass difference is roughly 198 kilograms, which appears to be fairly substantial. However, the approximate impact upon the simulation accuracy is dependent on the density and the total volume of the system. Equation \eqref{eq:new_rho_2} shows how the final density is impacted by the added mass, where the initial fuel salt density, $\rho_0$, is 3.35 grams per cubic centimeter.

\begin{equation} \hfill
\rho_f = \frac{m_0 + \Delta m}{V} = \rho_0 + \Delta \rho
\hfill\label{eq:new_rho_2} \end{equation}

Continuing with the direct linear case using the averaged uranium-233 feed, the change in the density is on the order of four milligrams per cubic centimeter, yielding a final density percent difference of 0.12\%. This density change due to adding material to a constant volume should be kept in mind if no mass balancing measures are taken, as it will have a non-physical impact on the results.
%In terms of the impact on overall results, the change in density due to implementing the continuous method is fairly negligible unless the required simulation is of very high precision.



\chapter{Conclusions}
\renewcommand*\descriptionlabel[1]{\hspace\leftmargin$#1$}
\setcounter{tocdepth}{5}
\setcounter{secnumdepth}{5}

\section{Comparison of Continuous and Batchwise Reprocessing Methods}

In this work, the differences and similarities between modeling MSRs using continuous and batchwise reprocessing have been discussed. Overall, it was shown that in terms of step size, computational cost, and steady state relative error, the continuous reprocessing method yields better results when handling continuous online reprocessing of an MSR.

For step sizes, it was shown mathematically how implementing larger depletion step sizes with a batchwise reprocessing method causes the relative steady state error to become larger.
It was also shown that this error corresponds to the cycle time of the elemental target, meaning that the maximum depletion time step which will yield valid results is dependent upon the reprocessing scheme of the given reactor.
For continuous reprocessing, the depletion time step is instead limited by updating the simulation data and does not depend on the reprocessing scheme, allowing for greater flexibility.
Variable step sizes were also considered for continuous reprocessing.
For the time frame of one year in the MSBR, using depletion time steps larger than 30 days led to increased error.
Changing the ordering of the steps did not have significant impact, but changing the largest depletion step size did directly impact the accuracy of the results.

For computational cost, generic equations were generated which showed the approximate costs of continuous and batchwise methods. The main differences in the costs came from the smaller depletion step size required of batchwise reprocessing to generate accurate results and from the doubling up of runs at the same time step.
Although the doubling up of runs could be removed, the continuous reprocessing method still retains a lower computational cost due to the larger depletion step sizes it allows.

For the steady state relative error, an equation was generated under steady state assumptions.
This allows for understanding how the difference in nuclide concentrations when using the different reprocessing methods comes about.
Specifically, it perfectly isolates the differences, allowing for understanding the level of error without accounting for more complex, but realistic scenarios.

The realistic scenarios for this work were demonstrated using the MSBR.
This allowed for non-steady state analysis, where the difference in reprocessing methods have many impacts.
For a given nuclide, its rate of generation from fission products may be impacted by the refueling reprocessing, the reprocessing removal of its parent, and the change in the fission rate due to parasitic absorption which depends on the reprocessing rates.
These complexities are shown within the results, which demonstrate how the batchwise and continuous reprocessing methods differ.

\section{Continuous Reprocessing Investigations}

Because the continuous reprocessing method provides many benefits when compared to the batchwise reprocessing method, investigating weaknesses in the method is important.
One of these weaknesses which was investigated in this work was the mass balancing of the fuel salt in the simulation.
The analysis of the net mass showed that most of the methods caused an increase in mass of the MSBR after one year.
Using an iterative process, the smallest net mass change after one year using the MSBR reprocessing scheme was found by adjusting the feed rates.
This smallest net mass change was 0.02\%, while the largest net mass change was 0.12\%.
This means that over the course of one year, the net mass in the MSBR has an insignificant impact on the results.

This work also discussed how continuous methods compared to each other and how continuous methods could approximate a batchwise method.
Overall, it was shown that, although there are several different methods to Decay reprocessing, they all yield similar results.

Finally, the inclusion of a decay tank to more model the MSBR with more physical accuracy was shown.
This method behaved differently from the method used by Rykhlevskii, with noticeable differences even after ten years of operation \cite{rykhlevskii_advanced_2018}.
This method was then combined with the continuous reprocessing to compare with a batchwise reprocessing method that used an average feed rate instead.
This led to combined differences due to the reprocessing methods and due to the refueling, which caused large differences after only 90 days of operation.
This shows how the combination of assumptions can lead to differing results quickly for an MSR model.


\section{Future Work}

There are areas that merit further investigation that will supplement this work.
The first area of interest would be replication of works which have implemented batchwise reprocessing to simulate a continuous reprocessing scheme, such as those shown in Section \ref{litrev-msr-batchwise}.
This would primarily consist of replicating the results using continuous reprocessing.
Additional work would be calculating the error for different isotopes, specifically isotopes which are expected to have a large impact on reactor behavior, affected by the reprocessing scheme.
This would lead nicely into an analysis of the difference in the reactor physics after depletion caused by those key isotopes.

Another useful analysis is looking into those same works, which have implemented batchwise reprocessing to simulate a continuous reprocessing scheme, and perform a depletion time step refinement study.
This would be useful to demonstrate the difference in reactor physics if the batchwise reprocessing scheme implemented used depletion time steps twice as long or twice as short.
Either would alter the concentrations of elements with short cycle times significantly.
The concentrations could potentially vary by a factor of two, which would then lead to differences in the reactor physics.
This would have an impact on the final results, and could be compared directly with the previous results of the work.

Another area of study is to replicate works which implement continuous reprocessing, such as those shown in Section \ref{litrev-msr-continuous}.
Particularly, investigating the development of the net mass would be of interest, as it could indicate non-physical results.
Specifically, of interest is the magnitude of the change in the net mass and how it affects the densities, and thus the macroscopic cross sections.
A change of a few percent in the macroscopic cross section due to a non-physical change in the mass would mean that the results presented in the work would have an accounted source of error affecting the results.
If the net mass change is large enough, determining feed rates which balance the mass or implementing batchwise reprocessing to balance the mass would be implemented.
This would allow for a comparison against the same results presented in the work with balanced mass to determine how the changes in the reactor physics altered the overall results in the work.





\chapter*{Appendix A: Batchwise Steady State Mass Derivation}
This appendix serves as a supplement to the discussion in Section \ref{s:batch-sum}.
Here, I show the steady state mass of some arbitrary nuclide which is processed using Steady batchwise reprocessing.

The first step of this derivation is to separate the $\lambda$ term into the continuous $\lambda_c$, which is the continuous losses in the Bateman equation, and the batchwise reprocessing $\lambda_r$.
This is given by:

\begin{equation} \hfill
\lambda = \lambda_c + \lambda_r.
\hfill \end{equation}

Next, set up and solve the differential equation for the continuous portions of the process.
This is set as the rate of change of the mass over time is equivalent to the constant rate of gain, $C$, minus the continuous losses.
This is given by:

\begin{equation} \hfill
\frac{dm}{dt} = C - \lambda_c m,
\hfill \end{equation}

where

\begin{equation} \hfill
m(t) = \left(m_0 - \frac{C}{\lambda_c} \right) e^{-\lambda_c t}  + \frac{C}{\lambda_c}.
\hfill \end{equation}

Then, I modify the continuous equation to a process which updates at discrete time steps:

\begin{equation} \hfill
m_{n+1} = \left(m_n - \frac{C}{\lambda_c} \right) e^{-\lambda_c \Delta t}  + \frac{C}{\lambda_c}.
\hfill \end{equation}

After each new discrete time step, I apply the batchwise reprocessing. 
Each time step is evaluated as a depletion time step.
This is given by:

\begin{equation} \hfill
m_{n+1} = \left(\left(m_n - \frac{C}{\lambda_c} \right) e^{-\lambda_c \Delta t}  + \frac{C}{\lambda_c}\right) \left(1 - {\lambda_r \Delta t} \right).
\hfill \label{eq:nplusone}\end{equation}

From here, I define some arbitrary variables for ease of use:

\begin{equation} \hfill
\alpha = \frac{C}{\lambda_c},
\hfill \end{equation}

\begin{equation} \hfill
\epsilon = e^{-\lambda_c \Delta t},
\hfill \end{equation}

\begin{equation} \hfill
\gamma = \left(1 - {\lambda_r \Delta t} \right).
\hfill \end{equation}

Using these in Equation \eqref{eq:nplusone}, the mass can be calculated for several subsequent steps.
The more compact form of Equation \eqref{eq:nplusone} is given by:

\begin{equation} \hfill
m_{n+1} = \left(\left(m_n - \alpha \right) \epsilon  + \alpha\right) \gamma.
\hfill \label{eq:one} \end{equation}

The initial condition is defined as:

\begin{equation} \hfill
m_0 = m_0.
\hfill \label{eq:two} \end{equation}

Combining Equations \eqref{eq:one} and \eqref{eq:two}, the first step is given by:

\begin{equation} \hfill
m_1 = m_0 \epsilon \gamma - \alpha \epsilon \gamma + \alpha \gamma,
\hfill \end{equation}

and the second step is given by:

\begin{equation} \hfill
m_2 = m_0 \epsilon^2 \gamma^2 - \alpha \epsilon^2 \gamma^2 + \alpha \epsilon \gamma^2 - \alpha \epsilon \gamma + \alpha \gamma.
\hfill \end{equation}

Finally, the n$^{th}$ step mass can be derived by following the pattern shown in the previous equations.
This is given by:

\begin{equation} \hfill
m_n = m_0 \epsilon^n \gamma^n + \alpha \sum_{j=1}^n \gamma^j (\epsilon^{j-1} - \epsilon^j).
\hfill \end{equation}

As the time approaches infinity, the steady state mass will be reached.
Because this is an iterative process, this means $n$ will be set to approach infinity as well to reach steady state.
%Moving towards the goal of the steady state mass, the value of $n$ is taken to be approaching infinity.
Because the values of $\gamma$ and $\alpha$ are both between zero and one, simplifications can be made.
This is given by:

\begin{equation} \hfill
m_{ss} = m_0 \epsilon^{\infty} \gamma^{\infty} + \alpha \sum_{j=1}^{\infty} \gamma^j (\epsilon^{j-1} - \epsilon^j),
\hfill \end{equation}

which simplifies to:
 
\begin{equation} \hfill
m_{ss} = \alpha \sum_{j=1}^{\infty} \gamma^j (\epsilon^{j-1} - \epsilon^j).
\hfill \end{equation}

The next step is to simplify the summation term by changing $\epsilon$ back to its original form.
This is given by:

\begin{equation} \hfill
m_{ss} = \alpha \sum_{j=1}^{\infty} \gamma^j (e^{-\lambda_c \Delta t (j-1)} - e^{-\lambda_c \Delta t j}).
\hfill \end{equation}

Next, I group the exponential terms together that have $j$ in the exponent, given by:

\begin{equation} \hfill
m_{ss} = \alpha \sum_{j=1}^{\infty} \gamma^j e^{-\lambda_c \Delta t j} (e^{\lambda_c \Delta t} - 1).
\hfill \end{equation}

The terms which multiply but do not contain $j$ from the summation can be pulled outside of the summation, given by:

\begin{equation} \hfill
m_{ss} = \alpha (e^{\lambda_c \Delta t} - 1) \sum_{j=1}^{\infty} \gamma^j e^{-\lambda_c \Delta t j}.
\hfill \end{equation}

Next, the geometric series can be solved by using the following equation from \cite{Mathematica}:

\begin{equation} \hfill
\sum_{j=1}^{\infty} x^j = \frac{x}{1-x} \ni |x| < 1
\hfill \end{equation}

This yields the equation for the steady state mass:

\begin{equation} \hfill
m_{ss} = \alpha (e^{\lambda_c \Delta t} - 1) \frac{\gamma^j e^{-\lambda_c \Delta t j}}{1 - \gamma^j e^{-\lambda_c \Delta t j}}.
\hfill \end{equation}

Plugging in the variables $\alpha$ and $\gamma$ yields the full equation for the steady state mass with batchwise reprocessing:

\begin{equation} \hfill 
m_{ss} =  \frac{C}{\lambda - \lambda_r} \left(e^{(\lambda - \lambda_r) \Delta t} - 1 \right) \sum_{n=1}^{\infty} \left( \left(1 - \lambda_r \Delta t \right) e^{-(\lambda - \lambda_r) \Delta t} \right)^n.
\hfill \end{equation}


\backmatter

\bibliographystyle{apalike}
\bibliography{bibliography}

\end{document}
\endinput
%%
%% End of file `thesis-ex.tex'.
