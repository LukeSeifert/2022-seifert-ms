\renewcommand*\descriptionlabel[1]{\hspace\leftmargin$#1$}
\setcounter{tocdepth}{5}
\setcounter{secnumdepth}{5}


\section{Molten Salt Reactors}

One of the generation IV reactor designs is the molten salt reactor, or MSR \cite{kelly_generation_2014}.
Specifically, in this work, I will be discussing the liquid fueled MSR.
%In this work, the main MSR which will be discussed is the liquid fueled MSR.
Moving forward, I will use the term MSR when referring to the liquid fueled variant unless specified otherwise.
The MSR is an advanced reactor design which is intended to improve safety and efficiency over previous reactor designs.
%For example, the operating pressure in an MSR kept close to atmospheric while the temperature is higher than that of traditional light water reactors.
%The lower pressure is useful from a safety perspective, while high temperatures with the high thermal capacity and conductivity of molten salts allows for high efficiency.



%The MSR potentially offers a lower "n'th-of-a-kind" cost of electricity when compared to coal and pressurized water reactors, which are a large portion of the nuclear fleet \cite{moir_cost_2002}.
%Of particular interest in this thesis is the liquid fueled MSR design, which is significantly different from the current light water reactor, or LWR, designs which dominate the nuclear fleet currently in use.
%Light water reactors include boiling water reactors (BWRs) and pressurized water reactors (PWRs), both of which are well understood reactor designs currently implemented.
%Light water reactors use a solid fuel containing uranium-235 with a water coolant, which also serves as a moderator to slow down the neutrons to thermal energies. This fuel generates heat as the fissile uranium-235 fissions. After some lengthy amount of depletion, it can be reshuffled or removed from the reactor to be placed into a cooling pool.

MSRs, instead of a solid fuel, use a liquid fuel composed of a molten salt mixed with some fissile isotope, such as uranium-235.
There are a wide variety of benefits to MSRs: high operating temperature, low operating pressure, and high thermal conductivity and capacity of the salts.
%Some of the benefits of MSRs include
Other benefits include a high coefficient of thermal expansion for negative temperature coefficient of reactivity, high resource utilization through higher burn-up, reduced cost of fabricating and transporting fuel elements, and the removal of fission products to improve the neutron economy \cite{serp_molten_2014}.
% WHy MSRs? How will it help nuclear? Reactivity/power density?

MSRs have several barriers to deployment which must be addressed before they are commercialized. Corrosion and material degradation is an issue for MSRs, as the high temperature salts are highly corrosive. Another barrier to deployment is in regulation, as the MSR operates very differently from light water reactors on the grid.
In order to properly regulate MSRs, we must have models which are able to simulate them correctly. One type of simulation which is important for determining fuel composition and reactor properties after long periods of operation are depletion simulations, which are of key interest for this work.

\section{Physical Depletion and Reprocessing}

%In a depletion simulation, general data is generated about the model at each depletion time step. After each of these steps, updated data is then used for the next depletion step. This updated data includes cross section data, energy spectra, neutron flux, and can include changes to temperature, volume, densities, or compositions. For MSRs, these updated changes after each depletion step need to account for the effects of online reprocessing.

Depletion is the process of fission over time, leading to fission products and decay chains.
Fission product yields are dependent on both the fuel composition and the interacting neutron energy.
Because the core neutron energy spectrum may shift spatially in the core, this means fuel and fission products are constantly changing in both composition and location.
For an MSR with online reprocessing, the additions and losses from reprocessing have to be considered within depletion.

Reprocessing can be considered through two methods: a continuous process or a batchwise process.
Continuous reprocessing is a process which occurs continuously, and it can include adding fresh fuel to the system, as well as chemically removing fission products.
%MSRs are also capable of batchwise reprocessing, which is a process which occurs in discrete intervals.
%An example of this is salt disposal after some set amount of time, which requires shutdown of the reactor.
MSRs are also capable of batchwise reprocessing, a process which occurs in discrete intervals, and it can include salt disposal after some set amount of time, which requires shutdown of the reactor.
%An example of this is
In general, online reprocessing schemes use continuous reprocessing.

The usefulness of online reprocessing is that the reactor does not need excess reactivity in order to continue operating for a long period of time.
In a typical light water reactor, burnable poisons and control rods are used to lower reactivity at the beginning of the core lifetime so the reactor can remain at a safe critical level over the core lifetime.
In an MSR, the parasitic absorbers, such as xenon-135, can be removed from the reactor during operation to reduce absorption and losses in the core to further improve the neutron economy.
Additionally, fresh fuel salt can be added to the reactor in order to maintain stable operation.

The fresh fuel feed can include fissile isotopes, such as uranium-233 or uranium-235; fertile isotopes, such as uranium-238 or thorium-232; or some combination of both.
Adding a fissile nuclide to the reactor will allow the reactor to continue operating through fission directly.
In fertile nuclides, the nuclide first absorbs a neutron, then the product of that reaction decays and produces a fissile nuclide. This process is illustrated for thorium-232 breeding in Equation \eqref{eq:th-pa-chain}, which is the breeding process used in the Molten Salt Breeder Reactor (MSBR).
This process creates fissile fuel on a time delay related to the fluence that the fuel is exposed to.
By using a balance of fissile and fertile feeds, the reactor can breed new fuel while it continues to operate with fissile fuel.

\begin{equation} \hfill
\ce{^{232}_{90}Th} + \ce{^{1}_{0}n} \longrightarrow \ce{^{233}_{90}Th} \xrightarrow{\beta^-} \ce{^{233}_{91}Pa} \xrightarrow{\beta^-} \ce{^{233}_{92}U}
\hfill\label{eq:th-pa-chain} \end{equation}

The half life of thorium-233 via beta decay is on the order of tens of minutes, while the half life of protactinium-233 via beta decay is roughly 27 days.
%Without reprocessing, the protactinium-233 remaining in the core for such a period of time would be more likely to interact with neutrons, impacting the breeding efficiency.
If the protactinium-233 remains in the core, some of it will be removed by competing neutron absorption.
This will impact the breeding efficiency and also remove neutrons from fissioning in the core.
To minimize this, the MSBR continuously chemically extracts the protactinium from the fuel salt, allowing it to decay in a separate tank until it decays into a fissile fuel.
Once it has decayed, another continuous chemical process removes the fissile fuel from the tank and reloads it into the reactor.
%another continuous chemical process strips uranium which is formed and brings the uranium back into the reactor, continuing the process.

Overall, the two different physical approaches in MSR reprocessing are continuous and batchwise. The reprocessing schemes can be online, while the reactor is operating, or offline, when the reactor is shutdown. Of particular importance for MSR behaviour is online reprocessing, which changes the fuel composition while the reactor is operating.
Because existing software may do depletion or reprocessing but not both, this is an important type of problem to be able to model in MSR research.
%This means that an important type of model for MSR analysis is a depletion simulation which handles the effects of online reprocessing.

\section{Computational Depletion and Reprocessing}

Depletion calculations require knowledge of the reactor composition and the reactor power. Through this, we can calculate the number of fissions and thus the fission products created by depleting the fuel. This is commonly done with neutron transport simulations.
The transport simulation itself can be performed using a stochastic method, namely Monte Carlo, or a deterministic method, such as diffusion.
However, traditional depletion software does not account for material changes in the core as a result of online or offline reprocessing of the fuel.
%This depletion simulation alone will only provide inform on the MSR without consideration for the reprocessing.

%An important type of model to simulate for reactor analysis is a depletion model, where the reactor operates for some length of time using discrete depletion steps. This model is important to determine fuel utilization, neutronic parameters at different stages in the reactor life, and to make further decisions about the reactor.

%In a depletion simulation, general data is generated about the model at each depletion time step. After each of these steps, updated data is then used for the next depletion step. This updated data includes cross section data, energy spectra, neutron flux, and can include changes to temperature, volume, densities, or compositions. For MSRs, these updated changes after each depletion step need to account for the effects of online reprocessing.

In order to model online reprocessing in a depletion simulation, there are two different computational methods which can be implemented. These methods reflect the physically continuous and batchwise reprocessing methods and have the same naming convention.
However, it is fairly common for researchers modeling MSR simulations with a physically continuous reprocessing scheme to use approximate batchwise mathematical models in their simulations.

In this work, I will compare and investigate the differences between continuous and batchwise computational reprocessing methods.
Because the effects of depletion and reprocessing are temporal, I will vary the depletion step size and compare the effects of this on reactor physics.
I will compare the validity of batchwise methods approximating continuous methods, and I will quantify the effects of different variations of the methods within both batchwise and continuous.
%This investigation involves effects of depletion step size, validity of batchwise methods approximating continuous methods, and different sub-methods within both batchwise and continuous methods which can be implemented.

