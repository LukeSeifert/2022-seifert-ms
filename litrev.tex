\renewcommand*\descriptionlabel[1]{\hspace\leftmargin$#1$}
\setcounter{tocdepth}{5}
\setcounter{secnumdepth}{5}

In this chapter, I will provide an overview of MSR modeling in the literature. This includes a discussion on different reprocessing methods and treatment of delayed neutron precursors. I will then provide more general information on depletion codes and the MSBR. Finally, I will provide a set of works and a brief discussion of each in order to show the state of the literature in terms of handling reprocessing in depletion simulations. This set of works will also show the number of works which have employed batchwise methods to approximate continuous reprocessing schemes. This is important, because a key takeaway of this work is that this approximation causes a buildup of error in the material composition.

\section{Molten Salt Reactors}

Modeling and simulation of liquid fueled molten salt reactors (MSRs) differs from solid fueled reactors in two main areas. The first is in the fresh fuel feed and fission product removal streams used during operation in MSRs; which is referred to as online reprocessing. Modeling an MSR without online reprocessing will cause discrepancies that will increase as the net depletion time increases.
This is a result of neutronic differences that arise from a lack of fresh fuel and buildup of fission products. In order to simulate this reprocessing functionality, software can use batchwise and continuous reprocessing methods.

The second difference is in the movement of the fuel salt, which causes the fuel and, consequently, the delayed neutron precursors (DNPs) to move in the core. Because the DNPs have different half lives before the delayed neutrons are born, the effect of the fuel movement on each group varies. However, the overall result is that the delayed neutrons are birthed in less important regions, such as external piping, which means there is a reduced effective delayed neutron fraction in the core.
This fraction is important for safe operation and reactivity control, so modeling it accurately is important to advancing this technology.
%Modeling and simulation of liquid fueled molten salt reactors (MSRs) is different from modeling and simulation of solid fueled reactors in a few different areas. Two main areas of particular importance are the online reprocessing functionality and the movement of the fuel through the reactor. The online reprocessing functionality involves removing fission products from the fuel and adding fresh fuel during operation. The movement of the fuel causes a shift in the location of delayed neutron precursors (DNPs). In solid fueled reactors, the distribution of DNPs follows the flux profile, while in MSRs the DNPs move along with the fuel. This results in a reduction in the effective delayed neutron fraction due to decay of DNPs in less important regions.

\subsection{Online Reprocessing}

The online reprocessing functionality of MSRs is simulated in different ways depending on the particular software used, and depending on the methods employed by that software. The two main ways to approach online reprocessing is by using batchwise reprocessing and continuous reprocessing methods. %For these approaches, the physical process may represent either approach, such as continuously removing protactinium from the core, but adding in the uranium it decays into only after some time has elapsed.
A purely batchwise approach without a sufficiently short time step will not fit the reprocessing design used by an MSR, as online reprocessing is carried out continuously.
For the batchwise method to fit a continuous reprocessing scheme, the time step would need to be on the same time scale as the shortest cycle time, or time it takes for a given element to be completely removed.
For the MSBR, the shortest cycle time is 20 seconds, which would make a depletion simulation on the order of years very computationally costly.
The batchwise method can accurately model a reprocessing scheme which uses batchwise reprocessing, such as a scheme which has salt entirely replaced after some given amount of time. An example of this is a salt discard, which occurs after a set amount of time. An entirely batchwise reprocessing scheme for an MSR would be neglecting the benefit of a liquid fuel, so such a scheme is not common in the literature.
%An MSR using offline reprocessing, however, can be accurately represented using purely batchwise reprocessing.

%Codes implement batchwise reprocessing by running the simulation, stopping, and then adding and removing materials. The process is then iterated. This method is useful in that it is fairly straightforward to implement, can be calibrated to different removal efficiencies, and makes mass balancing of the core straightforward. Some issues with the batchwise approach are that using large time steps will make the results more inaccurate, it has increased computational cost with smaller time steps, and it is an approximation of the actual physical process of reprocessing. It is an approximation because, in online reprocessing of a reactor, the fission products are continuously removed while any feed flows are continuously fed into the reactor.
Codes such as SaltProc and older versions of ChemTriton implement batchwise reprocessing by running the simulation for a fixed amount of time, then stopping, and then adding and removing materials \cite{rykhlevskii_modeling_2019, betzler_molten_2017}. The code iterates through this process for each depletion step until it has reached the final time. This method is useful because it is fairly straightforward to implement, easy to customize, and makes mass balancing of the core straightforward. Some issues with the batchwise approach are that using large time steps will make the results less accurate; it has to rerun each time step for pre-reprocessing and post-reprocessing; and it is an approximation of the actual physical process of reprocessing.
In an operating MSR, the fission products are continuously removed while any feed flows are continuously fed into the reactor, not removed in batches.
The reason that large time steps make the results less accurate can be made clear by imagining the limiting case, a single very large depletion step.
Taking a very large time step would mean that all of the reprocessing that should have taken place during that time step does not.
For example, instead of removing the xenon every 20 seconds, there is only a single removal at the very end after potentially years.
This means that the parasitic absorption of neutrons by fission products is not at all reduced over the entire depletion step, but instead the composition is only adjusted at the very end.
This example makes clear why small depletion steps are required in order to generate useful results using the batchwise method.

%It is an approximation because, in online reprocessing of a reactor, the fission products are continuously removed while any feed flows are continuously fed into the reactor.

%Continuous reprocessing is implemented by adding terms to the Bateman equation similar to an extra decay term. The benefits of using continuous reprocessing are that the model is physically accurate, it doesn't impact computational cost as much as a batchwise process, and it allows for large time steps to be used. A note should be made, however, that the time step size is still limited since the simulation will need to update based on the new material composition and evolving neutron spectrum. However, there is less error from inaccurate material removal and addition. The main downside of the continuous removal approach is that it is more cumbersome to implement and configure due to the nature of how it operates.
Codes such as Serpent2 and newer versions of ChemTriton implement continuous reprocessing by adding terms to the Bateman equation similar to an extra decay term, as can be seen in Equation (\ref{eq:Bateman_type_one_decay_dupe}), recreated here from Equation (\ref{eq:Bateman_type_one_decay}) for convenience of reading \cite{aufiero_extended_2013, jr_vicente_valdez_modeling_2020}. The benefits of using continuous reprocessing are that the model is more physically accurate and it allows for larger time steps to be used without increasing error. However, the largest time step size cannot become too large, or the simulation will become less accurate.
The exact value of the largest time step while remaining sufficiently accurate is dependent upon the reactor, the rate at which the material composition changes, and the level of accuracy which is desired.
This is because the neutron spectrum and cross section data does not update to the new material compositions until a new depletion step begins.


\begin{equation} \hfill
%\frac{dN}{dt} = \sum_j \lambda_j N_j + \gamma \Sigma_f \phi + \Sigma_k \phi - \lambda N - \Sigma \phi - C
%\frac{dN_j}{dt} = \sum_{i \neq j} \left [ \left( \gamma_{i \rightarrow j} \sigma_{f, i} \Phi + \lambda _{i \rightarrow j} + \lambda _{reproc, i \rightarrow j} + \sigma_{i \rightarrow j} \Phi \right) N_i \right ] - \left ( \lambda_j + \lambda_{reproc, j} + \sigma_j \Phi \right ) N_j
%\frac{dN_j}{dt} = \sum_{i \neq j} \left [ \left( \gamma_{i \rightarrow j} \sigma_{f, i} \Phi + \lambda _{i \rightarrow j} + \lambda _{r, i \rightarrow j} + \sigma_{i \rightarrow j} \Phi \right) N_i \right ] - \left ( \lambda_j + \lambda_{r, j} + \sigma_j \Phi \right ) N_j + \sum_{mat} \lambda _{r, i \rightarrow j} N_i
%\frac{dN_j}{dt}_{net} = \frac{dN_j}{dt}_{base} -  \lambda_{r, j} N_j + \sum_{mat} \lambda _{r, i \rightarrow j} N_i
\frac{dN_j}{dt}_{net} =\left( \frac{dN_j}{dt} \right)_{no\text{ }reprocessing} -  \lambda_{r, j} N_j + \sum_{feed} \lambda _{f, j} N_j
\hfill\label{eq:Bateman_type_one_decay_dupe} \end{equation}

%The main downsides of the continuous reprocessing approach are that it is more cumbersome to implement and difficult to configure due to the nature of how it operates.

Another aspect of online reprocessing that must be considered is mass balancing.
%For batchwise reprocessing, mass balancing can be solved in several different ways.
As mentioned previously, mass balancing is straightforward for batchwise reprocessing and can be approached in several different ways.
One method is to have the net mass of the feed rate over some depletion step be equivalent to the mass removed over that same depletion step, which is the current approach used by SaltProc \cite{rykhlevskii_modeling_2019}. Alternatively, the volume of the system can be adjusted so that constant density is maintained in order to keep cross section data consistent \cite{ridley_method_2017}. Another approach is to move excess fuel salt into a bleed-off tank \cite{ridley_method_2017}.

For mass balancing using continuous reprocessing, the depletion time steps can be much larger than those used in batchwise reprocessing.
This can cause a mass balancing issue if the system has a large feed rate but a small removal rate.
This could cause the simulation to continuously increase in mass, thus containing more fuel salt than intended and generating invalid results.
One way this can be treated, assuming the net change in mass in non-negligible, is to use batchwise reprocessing to balance the mass while continuous reprocessing handles the actual reprocessing.
%Solutions to this issue are to use balanced feed and removal rates, check the mass of the simulated core to ensure it is balanced over time, or use batchwise reprocessing at certain times to rebalance the core.
Implementing batchwise reprocessing increases computational cost when performed in Serpent2. This is because using a single input script with depletion steps will cause each depletion step to be run once, while altering the input script by changing volumes or removing mass to a bleed-off tank will cause the time step to be run twice.

\subsubsection{Batchwise Reprocessing}

%Imagine a process exists in which $X$\% of some element is removed from the fuel salt in the system over some time period T. For a batchwise approach, a straightforward approach would be to run the system, deplete for some time step $\Delta t$, remove $\gamma$\%, and repeat. The scaling term $\gamma$ for the removal efficiency is a linear approximation of the removal efficiency based on the time step adjustment and can be seen in Equation \ref{eq:batchwise_rem_eff}. An alternative approach is to establish a depletion time step $\Delta t$ such that the efficiency based time value $T$ is a multiple of $\Delta t$ and only have the batchwise removal occur at the steps when $\Delta t$ is a multiple of $T$.

%\begin{equation} \hfill
%\gamma = X \frac{\Delta t}{T}
%\hfill\label{eq:batchwise_rem_eff} \end{equation}
Batchwise reprocessing starts with some removal rate where some fraction, $f$, of a given element is removed from the fuel salt in the system over some time period, $T_{cyc}$, referred to as the cycle time. Over a given depletion simulation time step, $\Delta t$, a different fraction, $\gamma$, of the element's mass is removed, where $\gamma$ corresponds to $f$ according to Equation (\ref{eq:batchwise_rem_eff}). The scaling term on $f$ for the removal efficiency is a linear approximation based on the time step adjustment in Equation (\ref{eq:batchwise_rem_eff}). If the time step used causes the fraction removed, $\gamma$, to be greater than 1, the amount removed remains at 100\% so as to not have negative mass. This method is called Steady batchwise reprocessing since it uses a steady removal amount during each depletion step.

An alternative approach is to establish a depletion time step, $\Delta t$, such that the cycle time, $T_{cyc}$, is a multiple of $\Delta t$. Then, the fractional batchwise removal, $f$, occurs at the steps when $\Delta t$ is a multiple of $T_{cyc}$, removing the need for scaling. This method is called Bulk batchwise reprocessing because it performs reprocessing all at once in a bulk manner.

\begin{equation} \hfill
\gamma = f \frac{\Delta t}{T_{cyc}}
\hfill\label{eq:batchwise_rem_eff} \end{equation}

One example of implementing batchwise reprocessing can be seen in the depletion simulation of the Molten Salt Breeder Reactor using SaltProc \cite{rykhlevskii_modeling_2019}. This simulation uses batchwise reprocessing every three days with a linear approximation of the cycle times given by Robertson et al in the ORNL report \cite{robertson_conceptual_1971, rykhlevskii_modeling_2019}. The xenon and krypton removals in this work use a gas separation system equation, while the protactinium uses a modeled liquid-liquid reductive extraction process. These efficiencies differ from the cycle time values, causing slight differences in those terms. The MSBR also has two varying feed inputs of \ce{^{232}Th} and \ce{^{233}U}.

Examples of the Steady batchwise method used in SaltProc can be seen in Equations \eqref{eq:SP_20s} and \eqref{eq:SP_60d}. Both examples show what the resulting mass removal per step, or $\gamma$, term should be for different cycle time values. The first shows a cycle time of 20s, which is shorter than the step time of three days. This results in a removal rate over 100\%, which is set to 100\%. The second shows a cycle time of 60 days, longer than the step size. This results in a 5\% removal per step. An alternative approach to the second example, which is implemented in the earlier versions of SaltProc, is to perform no removal until 60 days have elapsed and then remove 100\% of material.


\begin{equation} \hfill
f = 100\%, T = 20 s, \Delta t = 3 d \rightarrow \gamma > 100\% \rightarrow 100\%
\hfill\label{eq:SP_20s} \end{equation}

\begin{equation} \hfill
f = 100\%, T = 60 d, \Delta t = 3 d \rightarrow \gamma = 5\%
\hfill\label{eq:SP_60d} \end{equation}

\subsubsection{Continuous Reprocessing}

%For a continuous approach, removing some fraction is dependent upon the particular isotope since the approach involves adding a term to the Bateman equation. This is one of the methods used in Serpent2 continuous reprocessing, where separations are only performed based on the element, not each particular isotope, an extension developed by Aufiero et al \cite{aufiero_extended_2013}. Equation \ref{eq:Bateman_default} shows the generic form of the Bateman equation, while Equation \ref{eq:Bateman_type_one_decay} shows the terms added to Equation \ref{eq:Bateman_default}, which are the reprocessing constants, or $\lambda_{r}$ added.
The continuous reprocessing method involves directly modifying the Bateman equation, shown in Equation \eqref{eq:Bateman_default}. One of the continuous reprocessing methods used in Serpent2 continuous reprocessing, an extension developed by Aufiero et al, is henceforth refered to as "decay" continuous reprocessing \cite{aufiero_extended_2013}. Equation \eqref{eq:Bateman_default} shows the Bateman equation with production and loss exclusively from fission, absorption, and decay; and Equation \eqref{eq:Bateman_type_one_decay} shows the Bateman equation with the removal from reprocessing subtracted and addition from feed sources added.
%Equation (\ref{eq:Bateman_type_one_decay}) shows the terms added to Equation (\ref{eq:Bateman_default}), which are the reprocessing constants, $\lambda_{r}$, which is of the same form of radioactive decay, hence the name "decay" continuous reprocessing. These terms represent removal of the current isotope as well as feed of the current isotope from other sources.

\begin{equation} \hfill
\begin{split}
%\frac{dN}{dt} = \sum_j \lambda_j N_j + \gamma \Sigma_f \phi + \Sigma_k \phi - \lambda N - \Sigma \phi - C
\frac{dN_j}{dt}_{base} = \sum_{i \neq j}  & \left [ \left( \gamma_{i \rightarrow j} \sigma_{f, i} \Phi + \lambda _{i \rightarrow j} + \sigma_{i \rightarrow j} \Phi \right) N_i \right ]\\
 & - \left ( \lambda_j + \sigma_j \Phi \right ) N_j
\end{split}
\hfill\label{eq:Bateman_default} \end{equation}

\begin{equation} \hfill
%\frac{dN}{dt} = \sum_j \lambda_j N_j + \gamma \Sigma_f \phi + \Sigma_k \phi - \lambda N - \Sigma \phi - C
%\frac{dN_j}{dt} = \sum_{i \neq j} \left [ \left( \gamma_{i \rightarrow j} \sigma_{f, i} \Phi + \lambda _{i \rightarrow j} + \lambda _{reproc, i \rightarrow j} + \sigma_{i \rightarrow j} \Phi \right) N_i \right ] - \left ( \lambda_j + \lambda_{reproc, j} + \sigma_j \Phi \right ) N_j
%\frac{dN_j}{dt} = \sum_{i \neq j} \left [ \left( \gamma_{i \rightarrow j} \sigma_{f, i} \Phi + \lambda _{i \rightarrow j} + \lambda _{r, i \rightarrow j} + \sigma_{i \rightarrow j} \Phi \right) N_i \right ] - \left ( \lambda_j + \lambda_{r, j} + \sigma_j \Phi \right ) N_j + \sum_{mat} \lambda _{r, i \rightarrow j} N_i
%\frac{dN_j}{dt}_{net} = \frac{dN_j}{dt}_{base} -  \lambda_{r, j} N_j + \sum_{mat} \lambda _{r, i \rightarrow j} N_i
\frac{dN_j}{dt}_{net} = \frac{dN_j}{dt}_{base} -  \lambda_{r, j} N_j + \sum_{feed} \lambda _{f, j} N_j
\hfill\label{eq:Bateman_type_one_decay} \end{equation}

The symbols given in the equations are defined as follows \cite{leppanen_development_2007}:
\begin{description}
\item[N_j] is the atomic density of isotope j $\left[\frac{atoms}{cm^3}\right]$.
\item[\gamma_{i \rightarrow j}] is the fractional fission product yield of $j$ in the fission of isotope $i$.
\item[\sigma_{f, i}] is the microscopic fission cross section of isotope $i$ $\left[cm^2\right]$.
\item[\Phi] is the spectrum-averaged scalar flux in the fuel region $\left[\frac{neutrons}{cm^2 s}\right]$.
\item[\lambda _{i \rightarrow j}] is the decay constant of decay $i \rightarrow j$ $\left[s^{-1}\right]$.
%\item[\lambda _{r, i \rightarrow j}] is the reprocessing constant for feed of $i \rightarrow j$
\item[\sigma_{i \rightarrow j}] is the microscopic transmution cross section of reaction $i \rightarrow j$ $\left[cm^2\right]$.
\item[N_i] is the atomic density of isotope $i$ $\left[\frac{atoms}{cm^3}\right]$.
\item[\lambda_j] is the decay constant of isotope $j$ $\left[s^{-1}\right]$.
\item[\lambda_{r, j}] is the reprocessing constant for removal of isotope $j$ $\left[s^{-1}\right]$.
\item[\sigma_j] is the microscopic total transmutation cross section of isotope $j$ $\left[cm^2\right]$.
\item[\lambda _{f, j}] is the feed constant for feed of isotope $i$ $\left[s^{-1}\right]$.
\end{description}

%It can be seen visually that the reprocessing removal essentially operates as increasing the decay rate when applied in this manner. For example, increasing the decay rate of a particular element in a given material in Serpent2 also requires another material to gain a feed rate equivalent to that decay rate. The feed rate of a particular material is given by summing the removal rates of other materials for each isotope.
Equation \eqref{eq:Bateman_type_one_decay} shows that the reprocessing removal has the same mathematical operation as the decay rate. Unlike decay, the isotopes removed by reprocessing instead are transfered to a different material, which operates as a feed for that material. This can be seen in the summation term, which sums over the different feeds. Figure \ref{fig:msr-outline-fig-show} shows how these reprocessing and feed constants fit into the reactor operation. Additionally, the figure shows that these terms relate to reprocessing and feed mass flow rates of $\dot{m}_{r, j}$ and $\dot{m}_{f, j}$, respectively.


\begin{figure}[H]
  \centering
  \includegraphics[scale=0.5]{images/msr-feed-repr-fig.png}
  \caption{Simplified model of how Bateman equation relates to mass flow rates.}
   \label{fig:msr-outline-fig-show}
\end{figure}


Serpent2 has three different settings for continuous reprocessing which alter the effect on the Bateman equations. In the Serpent2 documentation, the different settings are referred to as the "continuous reprocessing modes" which can be selected from the "reprocessor" keyword. The zeroth setting does not conserve mass and will not be discussed in this work.

The first setting, which adds a decay-like term to the Bateman equation, is shown in Equation (\ref{eq:Bateman_type_one_decay}). This approach of adding a decay-like term to the Bateman equation has been done with other codes \cite{jr_vicente_valdez_modeling_2020, rodriguez-vieitez_transmutation_2002}. For this setting, three different approaches are implemented in order to compare their results, and are discussed later in this work. This setting is the most important for this work, and is the only setting implemented.


%The first approach is labeled the Cycle Time Decay (CTD) approach. It takes half the value of the cycle times and treats them as half life values. A similar method of treating removal periods for reprocessing can be seen in Brovchenko et al \cite{brovchenko_neutronic_2019}. Equation (\ref{eq:CTDex}) shows how this approach works with a cycle time of 20 seconds. The operation cuts the cycle time in half and then calculates the decay constant, or reprocessing constant, using the half life from the cycle time.

%\begin{equation} \hfill
%20s \rightarrow \Delta t = 10s \rightarrow \lambda_{r} = \frac{ln(2)}{\Delta t} \approx 6.93E\minus 2
%\hfill\label{eq:CTDex} \end{equation}

%The next approach is labeled the SaltProc Cycle Rate (SPCR) approach. It takes the same efficiency rates used by SaltProc, which uses 3 day depletion steps for the MSBR, and directly converts those rates to continuous reprocessing rates. Taking the inverse of the cycle time is a common way to generate the removal rate \cite{rykhlevskii_modeling_2019, nuttin_potential_2005}. Equation (\ref{eq:SPCRex}) shows how this approach works with a cycle time of 20s. This operation uses the SaltProc value, which in this case becomes 100\% removal over 3 days, converts it to be per second, and then converts that to the rate form usable in Serpent2.

%\begin{equation} \hfill
%20s \rightarrow \frac{1}{3 d} \rightarrow X = \frac{4E \minus 6}{1 s} \rightarrow \lambda_r = ln \left( \frac{1}{1 - X} \right) \approx 4E \minus 6
%\hfill\label{eq:SPCRex} \end{equation}

%This natural logarithm in this approach comes from the differential equation solution based on the removal rate provided. The method used to derive this equation can be seen in Equations \eqref{eq:nat_log_1}, \eqref{eq:nat_log_2}, and \eqref{eq:nat_log_3}. These equations assume that the $X$ term, or the fractional removal rate, is in units of per second.

%\begin{equation} \hfill
%\frac{dN}{dt} = -\lambda_r N
%\hfill\label{eq:nat_log_1} \end{equation}

%\begin{equation} \hfill
%N(t) = (1-X) N_{cur} = N_{cur} e^{-\lambda_r}
%\hfill\label{eq:nat_log_2} \end{equation}

%\begin{equation} \hfill
%-ln(1-X)  = \lambda_r = ln\left( \frac{1}{1-X} \right)
%\hfill\label{eq:nat_log_3} \end{equation}

%The final approach is labeled the Cycle Rate (CR) approach. It takes the cycle times and uses a linear approximation to generate the efficiency rate directly. Equation (\ref{eq:CRex}) shows how this approach works using a cycle time of 20 seconds. This operation takes the cycle time, determines the efficiency rate for 100\% removal over 20 seconds, and then uses that value to calculate the Serpent2 reprocessing constant.

%\begin{equation} \hfill
%20s \rightarrow X = \frac{1}{20 s} \rightarrow \lambda_r = ln \left( \frac{1}{1 - X} \right) \approx 5.12E\minus 2
%\hfill\label{eq:CRex} \end{equation}

%Equations \ref{eq:Bateman_diff_removed} and \ref{eq:Bateman_X_removed} show the relationship between the net removal amount and the efficiency. Equations \ref{eq:Bateman_diff_control} through \ref{eq:Bateman_soln_control} show what the $\lambda$, or reprocessing constant, value would need to be set to such that, at the end of the simulation time $t_f$, the overall removal fraction is equal to $X$. These equations are a demonstration for a case in which there is no decay or production, only some initial amount that is removed continuously.

%\begin{equation} \hfill
%\frac{dN_{removed}}{dt} = \lambda N
%\hfill\label{eq:Bateman_diff_removed} \end{equation}

%\begin{equation} \hfill
%X = \frac{N_{removed}}{N_0 + N_{produced}}
%\hfill\label{eq:Bateman_X_precalc} \end{equation}


%\begin{equation} \hfill
%\frac{dN}{dt} = -\lambda N
%\hfill\label{eq:Bateman_diff_control} \end{equation}

%\begin{equation} \hfill
%N(t) = N_0 e^{-\lambda t}
%\hfill\label{eq:Bateman_eqn_control} \end{equation}

%\begin{equation} \hfill
%X = \frac{N_{removed}}{N_0 + N_{produced}} = 1 - \frac{N_f}{N_0}
%\hfill\label{eq:Bateman_X_removed} \end{equation}

%\begin{equation} \hfill
%\lambda = \frac{ln(\frac{1}{1-X})}{t_f}
%\hfill\label{eq:Bateman_soln_control} \end{equation}

%\begin{equation} \hfill
%X = 1 - e^{-\lambda t_f}
%\hfill\label{eq:Bateman_soln_control} \end{equation}


%A different isotope which naturally decays with a decay constant of $\lambda_d$ will have a different set of equations, which can be seen in Equations \ref{eq:Bateman_diff_decay} through \ref{eq:Bateman_eqn_decay_X}. It can be seen that if a value of 0 is used in Equation \ref{eq:Bateman_eqn_decay_X} for the decay constant, then the value matches the result in Equation \ref{eq:Bateman_soln_control}.

%\begin{equation} \hfill
%\frac{dN}{dt} = -\lambda N - \lambda_d N
%\hfill\label{eq:Bateman_diff_decay} \end{equation}

%\begin{equation} \hfill
%N(t) = N_0 e^{-(\lambda + \lambda_d) t}
%\hfill\label{eq:Bateman_eqn_decay} \end{equation}


%\begin{equation} \hfill
%\frac{dN_{removed}}{dt} = \lambda N = \lambda N_0 e^{-(\lambda + \lambda_d) t}
%\hfill\label{eq:Bateman_diff_decay_removed} \end{equation}

%\begin{equation} \hfill
%N_{removed}(t) = \frac{\lambda N_0}{\lambda + \lambda_d} \left( 1 - e^{-(\lambda + \lambda_d) t} \right)
%\hfill\label{eq:Bateman_eqn_decay_removed} \end{equation}

%\begin{equation} \hfill
%X = \frac{N_{removed}}{N_0 + N_{produced}} = \frac{\lambda}{\lambda + \lambda_d} \left( 1 - e^{-(\lambda + \lambda_d) t} \right)
%\hfill\label{eq:Bateman_eqn_decay_X} \end{equation}

%Although the results match when the decay is removed, while the decay is present, the results cannot be equivalent. This demonstrates one of the flaws of using continuous reprocessing, which is that the effects of the same reprocessing efficiency, $X$, on different isotopes of the same element requires unique reprocessing constants, $\lambda$, for each isotope in order for the removal efficiencies to be equivalent. Therefore, operating a continuous reprocessing scheme using a single removal constant for an element will fundamentally be non-physical due to the variances in removal efficiencies. The current system in Serpent2 allows for a removal constant to be applied to all isotopes of that element, which can circumvent this issue.

The second continuous reprocessing setting used in Serpent2 removes a constant value from each isotopic Bateman equation. Since the chemistry provides a removal rate for elements, the straightforward approach would be to generate a removal rate for each element. This approach can be seen in Equation \ref{eq:Bateman_diff_second_example}, where the $C$ term represents the constant value being removed by continuous reprocessing.

\begin{equation} \hfill
%\frac{dN}{dt} = \sum_j \lambda_j N_j + \gamma \Sigma_f \phi + \Sigma_k \phi - \lambda N - \Sigma \phi - C
\frac{dN_j}{dt} = \sum_{i \neq j} \left [ \left( \gamma_{i \rightarrow j} \sigma_{f, i} \Phi + \lambda _{i \rightarrow j} + \sigma_{i \rightarrow j} \Phi \right) N_i \right ] - \left ( \lambda_j + \sigma_j \Phi \right ) N_j - C
\hfill\label{eq:Bateman_diff_second_example} \end{equation}

This approach has a flaw which can be directly demonstrated through an example if the reprocessing constants are defined for an element and not for each isotope. Imagine an element exists with two isotopes $\alpha$ and $\beta$. There is a removal rate of 10\% per second defined for the element. A batchwise process could remove 10\% of each isotopes relative abundance after a second by using a one second time step, which is physical in terms of fractional removal. This can be seen directly in Table \ref{tab:cont_repr_appr}, where 10\% of both isotopes and the net count are removed.

\begin{table}[ht]
\renewcommand{\arraystretch}{1.25}
\caption{Serpent2 Second Setting Approaches}
\label{tab:cont_repr_appr}
\begin{center}
\begin{tabular}{ c | c | c | c | c | c | c }
 \hline
 Labels & Initial & Batch & Both Fit & $\alpha$ Fit & $\beta$ Fit & Zeroing Fit\\
 \hline
 \hline
$\alpha$ [g] & 1000 & 900 & 949.5 & 900 & 999 & 990\\
$\beta$ [g] & 10 & 9 & -40.5 & -90 & 9 & 0\\
 \hline
 Net [g] & 1010 & 909 & 909 & 810 & 1008 & 990 \\
 \hline
\end{tabular}
\end{center}
\end{table}

For the continuous reprocessing, both isotopes use the same constant removal rate $C$.%however, the same constant removal rate, $C$, is defined for both of the isotopes and is the same value.
To perform reprocessing, the first attempt might be to try and remove 10\% of the net by splitting it amongst the isotopes evenly. The issue with this approach is immediately noticeable, which is that the concentration of isotope $\beta$ becomes negative, as seen in the "Both Fit" column. A logical next approach would be to try and make one of the isotopes exactly correct while bringing the other along with it, which can be seen in the $\alpha$ and $\beta$ Fit columns. The results of this approach are seen in Table \ref{tab:cont_repr_appr}, and it can be seen that only by looking at the isotope with the smallest concentration and using that to determine the amount to remove can the result be ensured to be non-negative. One final approach to consider is to set the smallest concentration to 0, and bring the other isotopes along. This approach can be seen in the "Zeroing Fit" column, and it also does not work very well.

This analysis shows that, for this reprocessing setting to function properly, the reprocessing constants would need to be generated for every isotope individually in order to establish an online reprocessing scheme. However, this would require collecting data on the quantity of each isotope and calibrating reprocessing constants at every depletion time step. Stopping the Serpent2 depletion run to collect this data and then restarting would cause this approach to have similar computational cost to a batchwise reprocessing method, as Serpent would need to run each time point twice due to the restarts.

%Previously, this continuous reprocessing functionality in Serpent2 was undocumented, which led to difficulties in its usage by users \cite{rykhlevskii_modeling_2019}. Recently, however, the Serpent2 documentation was updated by this work as well as work performed at The University of Tennessee with Dr. Chvala and Alex Wheeler.

%Overall, it can be seen that in the two different approaches used in Serpent2, neither is able to maintain physicality like batchwise reprocessing does. However, different approximations can be used in order to apply the Serpent2 continuous reprocessing. Additionally, the way continuous reprocessing is applied can vary. Different uses could be using batchwise reprocessing for a fine mesh and continuous reprocessing for a coarse mesh, where the fine mesh is applied at beginning of cycle (BOC) and end of cycle (EOC) while the coarse mesh is used during steady-state (SS) operations. Another alternative is that the continuous reprocessing could be focused on certain isotopes, such as xenon-135, in order to allow for larger batchwise steps to be used.

\subsection{DNP Movement}

The movement of DNPs in liquid fueled molten salt reactors impacts the operation of the reactor \cite{wooten_review_2018, aufiero_calculating_2014}. %directly since the decay of the precursors occurs at a different location compared to a static reactor \cite{wooten_review_2018, aufiero_calculating_2014}.
There are several effects which this movement can have on the reactor. %Since the delayed neutron precursors move,
There is a probability of decay in non-core regions of the reactor, meaning there are fewer delayed neutrons in the core. This results in a reduced effective delayed neutron fraction, which reduces controllability of the reactor. Additionally, since delayed neutrons have a softer spectrum than prompt neutrons, there is a difference in spectrum in a reactor which has no flow against a reactor which has a circulating fuel.
In a solid fueled reactor, the flux and the concentration of delayed neutron precursors have the same profile. However, a moving fluid fuel causes the precursors to have a shifted profile based on the fluid flow rate, mixing, and decay rate of the precursors. This was shown by Jun Shi and Massimiliano Fratoni in their work, which can be seen in Figure \ref{fig:genfoam_dnp_locations} \cite{shi_gen-foam_2021}.

%\begin{figure}[H]
%  \centering
%  \includegraphics[scale=0.25]{images/dnp1.PNG}
%  \includegraphics[scale=0.25]{images/dnp3.PNG}
%  \includegraphics[scale=0.25]{images/dnp6.PNG}
%  \caption{Plots of DNP concentrations in the molten salt reactor experiment \cite{shi_gen-%foam_2021}. The left side of each image shows concentration with no flow, while the right shows the %concentration with a 1200 gallon per minute flow rate. Top Left) Longest lived DNP group. Top Right) %Third longest lived DNP group. Bottom) Shortest lived DNP group.}
%   \label{fig:genfoam_dnp_locations}
%\end{figure}

\begin{figure}[H]
\centering
\begin{subfigure}{0.4\textwidth}
    \includegraphics[scale=0.25]{images/dnp1.PNG}
    \caption{Longest lived delayed neutron precursor group.}
    \label{fig:first}
\end{subfigure}
\hfill
\centering
\begin{subfigure}{0.4\textwidth}
    \includegraphics[scale=0.25]{images/dnp3.PNG}
    \caption{Third longest lived delayed neutron precursor group.}
    \label{fig:second}
\end{subfigure}
\hfill
\centering
\begin{subfigure}{0.4\textwidth}
    \includegraphics[scale=0.25]{images/dnp6.PNG}
    \caption{Shortest lived delayed neutron precursor group.}
    \label{fig:third}
\end{subfigure}

\caption{Plots of DNP concentrations in the molten salt reactor experiment \cite{shi_gen-foam_2021}. The left side of each image shows concentration with no flow, while the right shows the concentration with a 1200 gallon per minute flow rate.}
\label{fig:genfoam_dnp_locations}
\end{figure}



Figure \ref{fig:genfoam_dnp_locations} shows that for the shortest lived DNPs, the movement of the fuel does not have any significant impact upon the DNP distribution when compared to static fuel. This is reasonable since the precursors do not live for very long in this group, so they do not have much chance to travel. For the intermediate DNP group, there is a clear shift in the direction of the fuel flow, which causes more precursors to produce delayed neutrons in the less important piping  and upper plenum regions. The longest lived DNP group seems to be spread almost equally throughout the entire reactor, which means that those delayed neutrons are contributing significantly less to the overall neutron economy. This reduces the overall effective delayed neutron fraction since the delayed neutrons are in less important regions. This directly affects the controllability of the reactor through the reactor period, which is heavily dependent upon the delayed neutrons.

\section{Depletion}

Depletion is the process of burning the fuel and simulating the changes to the materials this causes in the system. Depletion requires specific information to run properly, such as decay constants,  fission yields, and fission and transmutation cross section data. The fission and transmutation cross section data can come from a transport calculation, while the other data can be read from a data library \cite{leppanen_development_2007}. Additionally, running depletion involves solving the Bateman equations, a generic version of which can be seen in Equation \ref{eq:Bateman_default}.

There are several different approaches to solving these equations, such as the Transmutation Trajectory Analysis (TTA) method and the Chebyshev Rational Approximation Method (CRAM). These are the two different methods used in Serpent2 %, and are both built into Serpent2
\cite{leppanen_serpent_2015}. Other codes may use external software to compute depletion, such as REM used by MCNP and ORIGEN-S used by KENO-VI \cite{aufiero_extended_2013}. Another code %, aside from Serpent2,
that contains built-in depletion solvers is ERANOS \cite{aufiero_extended_2013}.

The purpose behind solving these equations and having depletion models is to generate an accurate representation of the composition of the target materials during reactor operation. This is important in determining how long the reactor can run, how the safety of the reactor develops over time, and how operation may have to change to adjust to the new reactor state. For molten salt reactors, depletion also allows for information regarding fresh fuel feed rate and fission product removal rates, as well as how variations of those parameters can affect reactor performance and behaviour.

% discuss how Serpent implements depletion and how Aufiero implemented repr

%\subsection{CRAM}

% https://serpent.vtt.fi/serpent/publications.htm only discusses CRAM

%\subsection{TTA}

%\subsection{Reprocessing}

% Reference Aufiero

\section{The Molten Salt Breeder Reactor}

The molten salt breeder reactor (MSBR) is a useful design to analyze for several reasons. Because it is a molten salt reactor, online reprocessing is used within the design.
In addition, it contains an inner and outer zone, zone I and II, respectively, each of which has different neutronic behaviour \cite{robertson_conceptual_1971}.
A portion of the core geometry is shown in Figure \ref{fig:msbr_ryklev}.
More specifically, the inner zone has a softer spectrum, a higher fission rate, and a 13\% fuel-to-graphite ratio \cite{park_whole_2015}.
The outer zone has a harder spectrum, a higher breeding rate, and a 37\% fuel-to-graphite ratio.
The differences between the inner and outer zones prove to be an issue when accurately modeling using a unit-cell or one-region approach, as those models are unable to capture the different characteristics of each region \cite{rykhlevskii_modeling_2019}.
The geometric differences in the regions can also be seen in Figure \ref{fig:msbr_ryklev}, which allows for the difference in the fuel-to-graphite ratio to be visually noticeable.

\begin{figure}[H]
  \centering
  \includegraphics[scale=0.25]{images/msbr_ryk_1.PNG}
  \caption{MSBR core axial slice showing the different regions from \cite{rykhlevskii_modeling_2019}. Zones II-A and II-B are where the spectrum is harder and there is increased breeding. Zone I is where there is more fission and a softer spectrum. The yellow is fuel salt, the purple is graphite, and the cyan is the reactor vessel.}
   \label{fig:msbr_ryklev}
\end{figure}

Table \ref{tab:msbr_gen_data} has general information on the MSBR, while Table \ref{tab:msbr_salt_data} has specific information on the distribution of the fuel salt during operation. It can be seen in Table \ref{tab:msbr_gen_data} that the salt inventory cycle time is 10 days, which means that the entire salt inventory is cycled in 10 days. However, the loop cycle time, or time it takes for the salt to complete a full loop in the primary system, is 11 seconds \cite{robertson_conceptual_1971}.

\begin{table}[H]
\renewcommand{\arraystretch}{1.25}
\caption{Summary of Principle Data for MSBR \cite{robertson_conceptual_1971}}
\label{tab:msbr_gen_data}
\begin{center}
\begin{tabular}{ c | c }
 \hline
 Component & Data\\
 \hline
 \hline
 Thermal Capacity & 2250 MW$_{th}$\\
 Vessel Inner Diameter & 6.77 m\\
 Core Height & 3.96 m\\
 Vessel Pressure & 0.52 MPa\\
 Central Core Zone Salt Fraction & 0.13\\
 Outer Core Zone Salt Fraction & 0.37\\
 Maximum Flow Velocity (Core) & 2.6 $\frac{m}{s}$\\
 Fuel Salt (Vessel) & 30.4 m$^3$\\
 Fuel Salt (Primary System) & 48.7 m$^3$\\
 Thorium Inventory & 68,100 kg\\
 Breeding Ratio & 1.06\\
 Processing Rate & 1 gpm\\
 Salt Inventory Cycle Time & 10 days\\
 Salt Components & $\ce{^{7}LiF-BeF_{2}-ThF_{4}-UF_{4}} $\\
 Salt Composition & 71.7-16-12-0.3 mole \% \\
 \hline
\end{tabular}
\end{center}
\end{table}

Table \ref{tab:msbr_salt_data} shows that the salt inventory in the core is approximately 19 m$^3$, with 8.2 m$^3$ from zone I and 10.8 m$^3$ from zone II.
%However, the MSBR model provided in SaltProc uses the volume of the primary system for its calculation \cite{rykhlevskii_modeling_2019}. This is an approximation which allows the depletion simulation to use the correct amount of net material, though it also means that material which would not be irradiated in the piping is present in the core. The core geometry uses the correct dimensions, which allows cross sections to remain as if the correct core volume were implemented.
%Overall, the depletion results are closer to the expected result if the fuel salt were fully utilized.


\begin{table}[H]
\renewcommand{\arraystretch}{1.25}
\caption{MSBR Primary System Salt Inventory \cite{robertson_conceptual_1971}}
\label{tab:msbr_salt_data}
\begin{center}
\begin{tabular}{  c | c  }
 \hline
 Region & Volume [m$^3$]\\
 \hline
 \multicolumn{2}{c}{Reactor}\\
  \hline
 Core Zone I & 8.2\\
 Core Zone II & 10.8\\
 Plenums, Inlets, Outlets & 6.2\\
 Annulus & 3.8\\
 Reflectors & 1.4\\
 \hline
\multicolumn{2}{c}{Primary Heat Exchangers}\\
 \hline
 Tubes & 7.6\\
 Inlets, Outlets & 0.8\\
  \hline
 \multicolumn{2}{c}{Miscellaneous}\\
  \hline
 Pump Bowls & 5.2\\
 Piping (including drain line) & 4.1\\
 Off-gas bypass loop & 0.3\\
 Tank heels and miscellaneous & 0.3\\
 \hline
  \multicolumn{2}{c}{Total}\\
  \hline
 Fuel Salt (Primary System) & 48.7\\
 \hline
\end{tabular}
\end{center}
\end{table}


\begin{table}[H]
\renewcommand{\arraystretch}{1.25}
\caption{MSBR Online Reprocessing Cycle Times \cite{robertson_conceptual_1971}}
\label{tab:msbr_cycle_times}
\begin{center}
\begin{tabular}{ p{0.27\linewidth} | p{0.25\linewidth} | p{0.35\linewidth} }
 \hline
 Reprocessing Group & Element(s) & Cycle Time (Full Power)\\
 \hline
 \hline
 Rare Earths & Y, La, Ce, Pr, Nd, Pm, Sm, Gd & 50 days\\
 Rare Earths & Eu & 500 days\\
 Noble Metals  & Se, Nb, Mo, Tc, Ru, Rh, Pd, Ag, Sb, Te & 20 seconds\\
 Seminoble Metals & Zr, Cd, In, Sn &  200 days\\
 Gases & Kr, Xe & 20 seconds\\
 Volatile Fluorides & Br, I & 60 days\\
 Discard & Rb, Sr, Cs, Ba & 3435 days\\
 Salt Discard & Th, Li, Be, F & 3435 days\\
 Protactinium & Pa & 3 days\\
 Higher Nuclides & Np, Pu & 16 years\\
 \hline
\end{tabular}
\end{center}
\end{table}


%Pa233, Np237, Pu242

Shown in Table \ref{tab:msbr_cycle_times} are the cycle times for different elements in the MSBR.
Here I define the cycle time as the time it takes for a given element to be completely removed, which is a common way to define the cycle time in the literature \cite{betzler_molten_2017, rykhlevskii_modeling_2019, robertson_conceptual_1971}.
According to Robertson et al, the protactinium and rare earth processing has the largest impact on neutronics and performance \cite{robertson_conceptual_1971}.

The fuel processing of the MSBR is defined to operate continuously with processing methods \cite{robertson_conceptual_1971}. For protactinium removal, fluorination is first used to remove uranium before protactinium isolation. Next is countercurrent bismuth with lithium and thorium for stripping any remaining uranium and the protactinium from the bismuth. This is followed by hydrofluorination to separate the uranium and protactinium. This process can be seen in Figure \ref{fig:msbr_pa}. For a 10 day protactinium cycle time, a fuel salt flow rate of 0.88 gallons per minute (gpm) is used. The processed salt is processed for rare earth fission products before being fed back into the reactor.
% Page 111 flow sheet
\begin{figure}[H]
  \centering
  \includegraphics[scale=0.55]{images/msbr_pa_robertson.PNG}
  \caption{MSBR protactinium processing scheme from Robertson et al \cite{robertson_conceptual_1971}.}
   \label{fig:msbr_pa}
\end{figure}

Rare earth fission product removal is performed using the metal-transfer process, which uses lithium chloride and bismuth containing a reductant. More specifically, bismuth containing a reductant of thorium and lithium is used to strip the rare earth fission products from the fuel salt. The rare earth fission products are then transported to the lithium chloride acceptor salt, though lithium bromide or a mix of both could be used. Figure \ref{fig:msbr_re} shows this process, and Figure \ref{fig:msbr_re_pa} shows the protactinium and rare earth processing schemes together.

\begin{figure}[H]
  \centering
  \includegraphics[scale=0.65]{images/msbr_re_robertson.PNG}
  \caption{MSBR rare earth processing scheme from Robertson et al \cite{robertson_conceptual_1971}.}
   \label{fig:msbr_re}
\end{figure}

\begin{figure}[H]
  \centering
  \includegraphics[scale=0.75]{images/msbr_flows_robertson.PNG}
  \caption{Combined MSBR rare earth and protactinium processing schemes from Robertson et al \cite{robertson_conceptual_1971}.}
   \label{fig:msbr_re_pa}
\end{figure}

To strip gaseous fission products, a bubble generator introduces helium sparging gas in 15 to 20 mil bubbles. 10 percent of the flow is redirected to the bypass loop which contains a gas separator. This gas separator strips the fission products with approximately 100 percent efficiency. The off-gas system also includes an approximately two hour holdup during which noble metals deposit on the fuel salt drain tank surface. Figure \ref{fig:msbr_nm_pa} shows this process, beginning with the number per time, or flux, of the noble gasses in the salt migrating to the graphite voids and helium sparging bubbles. This is followed by the volume holdup into more processing to handle the noble metal fission product cleanup.

\begin{figure}[H]
  \centering
  \includegraphics[scale=0.65]{images/msbr_nm_robertson.PNG}
  \caption{MSBR noble metal flux and off-gas system from Robertson et al \cite{robertson_conceptual_1971}.}
   \label{fig:msbr_nm_pa}
\end{figure}

%Periodic discard of salt at a rate of 0.1 cubic feet per day occurs due to buildup of non-volatile fluorides during fluorination accumulating in the decay tank, such as from zirconium and corrosion product nickel.

%From SaltProc's MSBR example, the removal of each group can be attributed to a particular process within the physical system \cite{rykhlevskii_modeling_2019}. The entrainment separator is used to remove the gases; a nickel mesh is used to remove noble metals, seminoble metals, and volatile fluorides; and various processes given by Robertson are lumped together into the liquid metal process which removes rare earths, discard, and protactinium \cite{robertson_conceptual_1971}. The higher nuclides are removed at such an infrequent rate that they are not included within the SaltProc processes. The MSBR design removes the protactinium with the intent that the protactinium-233 will not absorb neutrons and will instead be fed out of the core to decay into uranium-233, which can then be fed back into the core.


\section{MSR Modeling Approaches}
\label{s:models-methods}

For modeling of MSRs, it is important to model discrete and temporal behavior in core physics.
%The first takes a given fuel composition and analyzes the performance of the reactor given that fuel composition. This may include
With discrete modeling, one may aim to focus on aspects such as DNP movement \cite{fei_molten_2020, shi_gen-foam_2021} or reactor physics, such as kinetics and dynamics \cite{singh_plant-level_2020, cervi_development_2019, aufiero_development_2014, cui_development_2022, singh_dynamics_2017}.
This type of model typically is focused on the order of seconds.

For temporal modeling, the approach involves depleting the fuel, typically focusing on time scales ranging from days to years. Because the online reprocessing of an MSR has such large effects on its fuel composition during a long depletion simulation, these depletion simulations of MSRs typically incorporate a method to simulate online reprocessing.
These simulations can also account for other factors, such as DNP movement, as shown by Zhou et al \cite{zhou_fuel_2018}.
The movement of DNPs can have a noticeable effect on reactor performance by affecting neutron energy spectra and localized reactivity effects, though they do not have an impact when looking at longer time scales \cite{betzler_implementation_2017}.
One method used to approximate spatial dependence of fuel salt without modeling the piping of the core is the scaled flux method \cite{betzler_liquid-fueled_2021}.
Most depletion simulations, however, focus primarily on depletion and do not consider the DNP movement during depletion.

Table \ref{tab:codes_types} shows this in the "DNP" column, which is dominated by models which do not include DNPs.
The table is sorted by the "Reprocessing" column, which signifies the method(s) used in the different MSR models and is also sorted by the publication year in the "Year" column.
The table shows that there are many different MSRs, shown in the "Reactor" column, which have been modeled using a variety of spatial refinement levels, shown in the "Model" column, ranging from unit cells to full 3D models.

A key takeaway from Table \ref{tab:codes_types} is that batchwise reprocessing methods are commonly used to a similar extent as continuous reprocessing methods to handle MSR online reprocessing.
This is important because, as previously explained, the batchwise reprocessing methods are a less accurate way to simulate online reprocessing for a physically continuous scheme.
Therefore, because it is still common for works to use this method, it is important that the differences between continuous and batchwise methods are better understood.


For all of the continuous reprocessing models given in Table \ref{tab:codes_types}, all of them use the "fictitious decay constant" approach, which is shown in Equation \eqref{eq:Bateman_type_one_decay}. For the batchwise models, most use the Steady batchwise approach shown in Equation \eqref{eq:batchwise_rem_eff} and remove a fraction at each depletion step rather than bulk removal at certain steps.

%The "Reprocessing" column of Table \ref{tab:codes_types} shows the mathematical approach taken to the online reprocessing of the work, and does not necessarily represent the reactor reprocessing approach. However, since online reprocessing is considered in the table, offline batchwise reprocessing is not included, such as in the work by Zou et al \cite{zou_transition_2020}. The works shown in the table all use continuous reprocessing schemes, but some approximate the continuous physical process using batchwise methods.

%For modeling of MSRs, there are typically three main types of models which are used. The first is a model which takes a fuel composition and runs without accounting for reprocessing, but accounts for the fuel movement effects, such as DNP drift, shown by Fei et al \cite{fei_molten_2020}. The second is a model which accounts for both fuel movement and reprocessing, such as Zhou et al \cite{zhou_fuel_2018}. The third is a model which does not account for fuel movement but accounts for reprocessing. Models which use the second or third approach are detailed in Table \ref{tab:codes_types}.

%The purpose of this table is to show that batchwise approximations to continuous reprocessing schemes have been implemented historically and recently. These methods have been implemented for a range of various reactors, various model types and dimensions, use various depletion step sizes, and cover various net depletion times. Additionally, tracking delayed neutron precursors is not common in depletion calculations, primarily due to their negligible impact on depletion results.

In Sections \ref{litrev-msr-continuous}, \ref{litrev-msr-batchwise}, and \ref{litrev-msr-mixed}, I will discuss the results of each of the studies through the lens of the current work. That is, I will summarize the effects of using those depletion and reprocessing methods and note what differences may be observed from the literature. Then, in Section \ref{litrev-msr-gaps}, I will comment on the gaps that this work will address.

% Remove scale and step columns from table, include in description of model
\begin{table}[H]
\renewcommand{\arraystretch}{1.25}
\caption{Molten Salt Reactor Models Analyzed with Methods Approximating Online Reprocessing}
\label{tab:codes_types}
\begin{center}
\begin{tabular}{  c | c | c | c | c | c}
 \hline
 Publication & Reprocessing & Model & Reactor & DNP & Year\\
 \hline
 \hline
 Valdez et al \cite{jr_vicente_valdez_modeling_2020} & Continuous & Unit Cell & N/A & No & 2020\\
 Zhuang et al \cite{zhuang_extended_2020} & Continuous & Full 3D & MSFR & No & 2020\\
 Xia et al \cite{xia_development_2019} & Continuous & Full 3D & N/A & No & 2019\\
 Zhou et al \cite{zhou_fuel_2018} & Continuous & Full 3D & MSBR & Yes & 2018\\
 Aufiero et al \cite{aufiero_extended_2013} & Continuous & Full 3D & MSFR & No & 2013\\
 Rodriguez-Vieitez \cite{rodriguez-vieitez_transmutation_2002} & Continuous & Unit Cell & ADNA & No & 2002\\
 \hline

 Rykhlevskii \cite{rykhlevskii_modeling_2019} & Batchwise & Full 3D & MSBR & No & 2019\\
 Betzler et al \cite{betzler_molten_2017} & Batchwise & Unit Cell & MSBR & No & 2017\\
 Ridley and Chvala \cite{ridley_method_2017} & Batchwise & Full 3D & Toy & No & 2017\\
 Ahmad et al \cite{ahmad_neutronics_2015} & Batchwise & Unit Cell & DMSR & No & 2015\\
 Park et al \cite{park_whole_2015} & Batchwise & Full 3D & MSBR & No & 2015\\
 Sheu et al \cite{sheu_depletion_2013} & Batchwise & Full 3D & MOSART & No & 2013\\
 Nuttin et al \cite{nuttin_potential_2005} & Batchwise & Full 3D & MSBR & No & 2005\\

 \hline
 Fiorina et al \cite{fiorina_preliminary_2012} & Mixed & 2D Slice & MSFR & No & 2012\\
 Nagy et al \cite{nagy_parametric_2008} & Mixed & Unit Cell & Toy & Yes & 2008\\
 Merle-Lucotte et al \cite{merle-lucotte_thorium_2007} & Mixed & Full 3D & TMSR & No & 2007\\
 Bauman et al \cite{h_f_bauman_rod_1971} & Mixed & 2D Slice & N/A & Yes & 1971\\

 \hline
\end{tabular}
\end{center}
\end{table}

\subsection{Continuous Reprocessing}
\label{litrev-msr-continuous}

Table \ref{tab:codes_types} shows that many different codes have been used to simulate continuous reprocessing in MSRs. One common aspect between all of them is that they use long depletion step sizes (on the order of months) and long net depletion times (on the order of years).

Valdez et al \cite{jr_vicente_valdez_modeling_2020} demonstrates the efficacy of the updated version of ChemTriton. The version was added after TRITON implemented new tools for continuous material feeds and removals. Valdez et al discusses in depth how the removal rates for isotopes can be calculated using a well mixed approximation and accounting for non-modeled piping by introducing a correcting decay factor. Although it does not account for DNP movement, it does account for neutron important regions without having to model the piping regions.

Zhuang et al \cite{zhuang_extended_2020} discusses an OpenMC extension called OpenMCB-MSR to implement continuous reprocessing in an analysis of the MSFR. The depletion step sizes used are not specified in the work, though they appear to be on the order of decades, with a final depletion time of 200 years.
%with continuous reprocessing step sizes could have been set to above single digit days and still maintained high precision.

Xia et al \cite{xia_development_2019} demonstrates the MOlten salt reactor specific DEpletion Code (MODEC). This code uses continuous reprocessing, and has been used to analyze the Molten Chloride salt Fast Reactor (MCFR) by Liaoyuan et al \cite{liaoyuan_th-u_2021}. The MCFR analysis uses SCALE6.1 for transport and MODEC for depletion.

Zhou et al \cite{zhou_fuel_2018} covers FAMOS (Fuel cycle Analysis code for MOlten Salt reactors). This code uses OpenMC to generate homogenized cross section data followed by the DIF3D diffusion solver along with extensions to model DNP movement, thermal hydraulics, depletion, and continuous reprocessing. Zhou et al analyzes the molten salt fast reactor over 200 operational years and the molten salt breeder reactor for 20 years. The depletion step size is not specifically given in the work, but from analysis of the plots, small time steps seem to have been used at the beginning of cycle and then larger time steps of ~100 days are then used for the rest of the steps.

Aufiero et al \cite{aufiero_extended_2013} displays the Serpent2 newly developed continuous reprocessing functionality. A full 3D simplified core layout of the molten salt fast reactor (MSFR) is used, and continuous reprocessing is used to operate the removal and feed processes. It evaluates the model at various scales, from steady state uranium concentrations at over 60 years to radiotoxicity over $10^8$ years. The step sizes vary as well, though not specifically mentioned, appearing very small at beginning of cycle and expanding to around 10 years at steady state for one of the figures presented.

Rodriguez-Vieitez \cite{rodriguez-vieitez_transmutation_2002} analyzes the Accelerator-Driven Neutron Applications (ADNA) Tier-1 reactor using MCNP. Although the model Rodriguez-Vieitez used does implement continuous reprocessing in the Bateman equation, it is not used for the purpose of fuel cycle analysis and depletion. Rather, the work iteratively works towards the equilibrium fuel composition in order to analyze the effect of varying different reactor parameters, such as geometry.

\subsection{Batchwise Reprocessing}
\label{litrev-msr-batchwise}

The second category of studies summarized in Table \ref{tab:codes_types} use batchwise reprocessing. This method uses depletion step sizes on the order of several days, and uses net depletion times on the order of years.

Rykhlevskii \cite{rykhlevskii_modeling_2019} uses SaltProc to simulate the MSBR. Rykhlevskii uses Serpent2, a full 3D core of the MSBR, three day time steps, and runs for 60 years.
Rykhlevskii calculated the average thorium-232 refueling rate, compared safety parameters at beginning of cycle and equilibrium, and collected other useful data for understanding and further simulating the MSBR.
The batchwise reprocessing in SaltProc includes both the removal and feed processes.

Betzler et al \cite{betzler_molten_2017} uses ChemTriton, a tool designed for SCALE to perform online reprocessing of MSRs. It contains work on multiple reactors, such as the MSBR and the MSRE. The tool iteratively runs SCALE/TRITON over small time steps to simulate continuous reprocessing, and includes enhanced time-dependent feed and separations.
Betzler et al use this tool to optimize the thorium concentration to maintain a critical configuration while maintaining breeding capability, show the impact of fission product removal on core lifetime and fuel utilization, and more.
For the MSBR case, Betzler et al uses three day time, similar to Rykhlevskii and Park et al.
This time step is selected because previous parametric studies showed that it is sufficiently short enough to model the online removal of protactinium \cite{powers_new_2013}.
%The reason for not using shorter time steps is that using shorter time steps increases computational cost while also providing very little improvements to calculated eigenvalue and fuel cycle metrics, discussed by Betzler et al as well as by Powers et al \cite{powers_new_2013}.

Ridley and Chvala \cite{ridley_method_2017} use batchwise reprocessing on a toy MSR using Serpent2 and a custom in-house python library. The feed rates were set to fluctuate in order to maintain criticality while also accounting for density imbalance in Serpent2 by increasing the volume of the material. The full core 3D model was then analyzed using seven day depletion steps for a reactor operating period of ten years.

Ahmad et al \cite{ahmad_neutronics_2015} analyzes single fluid and two fluid denatured molten salt reactors (DMSRs). Ahmad et al use MCNP5 and ORIGEN2 as the codes to perform transport and depletion over 30 years. Although the steps at which fuel is adjusted is at five year intervals, the burnup intervals are set to six months. The reason such large steps were used is because the work analyzes proliferation and resource requirements for the reactors. This means that the neutronics and safety parameters were not the focus of the study. Additionally, only gaseous fission products and fission products removed via sparging are included in the reprocessing scheme. No other chemical extraction takes place.
%This is because the authors wanted the system to be as simple as possible. ORIGEN provides continuous reprocessing capability, however for this work it would not have been necessary since the authors wanted to use a simpler approach for the model \cite{gauld_isotopic_2011}.

Park et al \cite{park_whole_2015} is very similar to the MSBR example in SaltProc, with the primary difference being MCNP6 is used for transportation calculations instead of Serpent2. The depletion was performed using CINDER90 and custom Python scripts. Another difference is that SaltProc uses adjusted values for the removal efficiency rates of xenon, krypton, and protactinium, while the work by Park et al uses the cycle time values from Robertson et al \cite{robertson_conceptual_1971}. This model uses three day time steps and models the fuel composition over 20 years.

Sheu et al \cite{sheu_depletion_2013} analyzes the Molten Salt Actinide Recycler \& Transmuter (MOSART) reactor and uses custom scripts to implement batchwise reprocessing. The codes used are SCALE6 for transport and TRITON for depletion. The step sizes used include 15, 30, and 60 days, in which the authors found that 30 days gave the best results in terms of accuracy and run time balancing for the 30 year simulation.

Nuttin et al \cite{nuttin_potential_2005} analyzes the MSBR using a slightly simplified geometry with MCNP. The reprocessing approach used here is also the same as the approach used in SaltProc, which is to take the inverse of the cycle time to determine the removal rate. Depletion is performed in this work by coupling MCNP to an evolution program and the NJOY code. However, Nuttin et al use ten day steps instead of three day steps that SaltProc uses. It also runs for 100 years instead of 20.

\subsection{Mixed Batchwise and Continuous Reprocessing}
\label{litrev-msr-mixed}

A mixture of batchwise and continuous methods can be implemented if the reprocessing scheme includes physically batchwise processes or if mass balancing is an issue for the model.

Fiorina et al \cite{fiorina_preliminary_2012} is on the Molten Salt Fast Reactor (MSFR) and uses a modified version of the ERANOS-based EQL3D procedure. Fiorina et al make use of continuous reprocessing for non-soluble fission products, but uses batchwise reprocessing for soluble fission products. The batchwise reprocessing was incorporated for simplicity of implementation in the model. The time steps used are shorter at the beginning of cycle and extend out to roughly 30 year steps for a 200 year depletion. Additionally, the authors also performed a radiotoxicity analysis which extends out to one million years.

Nagy et al \cite{nagy_parametric_2008} uses a toy model to go through different reprocessing schemes based on the MSBR and MOSART reprocessing designs \cite{robertson_conceptual_1971, ignatiev_progress_2007}. SCALE-5 XSDRN\cite{noauthor_scale_2005} is used for neutron transport and ORIGEN-S is used for depletion. To account for DNPs, the authors assume that one-third of the delayed neutron precursors decay outside the core.
Using a delayed neutron fraction of 0.0027 for $^{233}$U, the authors claim that 0.0009, or 90 pcm, should be subtracted from the eigenvalue to approximate the impact from the movement of delayed neutron precursors.
Depletion steps are performed every five days, during which continuous reprocessing was implemented for only gaseous fission product removal, while refueling and other fission products used batchwise reprocessing.
The batchwise reprocessing was executed by altering the binary output of ORIGEN.

Merle-Lucotte et al \cite{merle-lucotte_thorium_2007} covers depletion of the Thorium Molten Salt Reactor (TMSR) using MCNP4 \cite{goluoglu_software_1998} for transport and REM for material evolution. Continuous reprocessing is used for gaseous fission product extraction, while the other fission products are processed with batchwise reprocessing.

Bauman et al \cite{h_f_bauman_rod_1971} presents the Reactor Optimum Design (ROD) code for circulating-fuel reactors.
ROD allows users to use continuous or batchwise reprocessing for MSRs. ROD operates by taking a one or two dimensional input and determining the equilibrium fuel composition. This is a useful code for determining a general prediction for a model, but is limited to one or two dimensional problems, a diffusion solver, 15 energy groups, and 200 fission products.
%less precise than more modern codes which offer more detailed geometry and other features.

\subsection{Discussion}
\label{litrev-msr-gaps}

Overall, from these works it can be seen that using continuous reprocessing allows for larger time steps.
These larger time steps allow for decreased computational cost, which enables the works which use continuous reprocessing to simulate results over a longer time period.
However, continuous reprocessing is more complex to implement than batchwise reprocessing.

For codes which do not have continuous reprocessing built-in, authors will find it more efficient to use a batchwise method which can be handled externally rather than implement continuous reprocessing into a code.
This is particularly the case if the authors have a time constraint and need to generate results quickly.
Implementation of continuous reprocessing into an existing code is a work in itself, as shown by authors such as Aufiero et al and Zhuang et al.

Thus, unless every depletion code used in MSR modeling is updated to handle continuous reprocessing, batchwise approximations will continue to be implemented.
Because this approximation is common and is likely to continue in the future, it is of increasing importance to determine the impacts of this approximation.
It is well understood that smaller step sizes are required for batchwise methods to approximate continuous methods, but there has yet to be a full analysis on how the results for an MSR using the two different methods compare.
This is the gap that this work seeks to fill.
I will show how the eigenvalue and various isotopic compositions vary while altering only the reprocessing methods implemented.


%Because the net depletion time is similar to that of continuous reprocessing while implementing much smaller depletion step sizes, this results in a higher computational cost.
% Gaps, best models, etc.

\section{SaltProc}
\label{s:theSaltProc}

\subsection{MSBR Model}

SaltProc comes with a full 3D model of the core of the MSBR created in Serpent2.
The MSBR model provided in SaltProc uses the volume of the primary system, 48.7 cubic meters, for its calculations rather than the 19 cubic meters of the core \cite{rykhlevskii_modeling_2019}.
This is an approximation which allows the depletion simulation to use the correct amount of net material, though it also means that material which would not be irradiated in the piping is present in the core.
The core geometry uses the correct dimensions, which allows cross sections to remain as if the correct core volume were implemented.

\subsection{Reprocessing}
\label{s:SP-repr}

SaltProc is used to handle batchwise reprocessing in this work because it uses Serpent2, which has built-in continuous reprocessing; comes with the MSBR geometry in Serpent2 as an example; and has multiple versions.
Of these different versions of SaltProc, they have slightly different methods in how the depletion data is stored as well as how reprocessing rates are calculated.
However, the two versions of interest are version 0.1 and 0.3, which use different batchwise reprocessing approaches.
These two different approaches are referred to here as the "Bulk" approach and the "Steady" approach.

SaltProc version 0.1 was the first full version of SaltProc publicly released, and provides all the functionality necessary for modeling the MSBR with batchwise reprocessing.
Using this version, Rykhlevskii uses three day depletion steps and does not perform fractional reprocessing at each step, instead performing reprocessing according to the cycle times \cite{rykhlevskii_advanced_2018}.
This approach is termed as the "Bulk" approach to batchwise reprocessing, and is explained in more detail below. This is the same approach used by Gehin and Powers for an analysis of the MSBR \cite{gehin_liquid_2016}.
This version of SaltProc, version 0.1, makes use of the h5py Python package to handle the hdf5 data files which contain depletion data \cite{collette_python_hdf5_2014}. Within the hdf5 file, the atom density of each isotope before and after batchwise reprocessing is recorded as well as different reactor parameters, such as the neutron multiplication factor.

SaltProc version 0.3 includes an example MSBR which users can run as soon as SaltProc is installed, though some of the run parameters such as neutrons per generation and run time have to be altered to generate a more valid model.
This model also uses three day steps, but instead of performing the batch removal only at the cycle time value, this model removes a fraction every three days, which is called the "Steady" approach for batchwise reprocessing.
Additionally, this version of SaltProc uses the PyTables Python package for handling the hdf5 data files. The data files are structured similarly to v0.1 of SaltProc, but instead of atom density they use mass, which makes analysis of the files slightly more user friendly.
